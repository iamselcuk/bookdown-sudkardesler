\documentclass[]{book}
\usepackage{lmodern}
\usepackage{amssymb,amsmath}
\usepackage{ifxetex,ifluatex}
\usepackage{fixltx2e} % provides \textsubscript
\ifnum 0\ifxetex 1\fi\ifluatex 1\fi=0 % if pdftex
  \usepackage[T1]{fontenc}
  \usepackage[utf8]{inputenc}
\else % if luatex or xelatex
  \ifxetex
    \usepackage{mathspec}
  \else
    \usepackage{fontspec}
  \fi
  \defaultfontfeatures{Ligatures=TeX,Scale=MatchLowercase}
\fi
% use upquote if available, for straight quotes in verbatim environments
\IfFileExists{upquote.sty}{\usepackage{upquote}}{}
% use microtype if available
\IfFileExists{microtype.sty}{%
\usepackage{microtype}
\UseMicrotypeSet[protrusion]{basicmath} % disable protrusion for tt fonts
}{}
\usepackage{hyperref}
\hypersetup{unicode=true,
            pdftitle={SUD KARDESLER},
            pdfauthor={Dâr'ül-bedâî' Rejisörü İ.Galib},
            pdfborder={0 0 0},
            breaklinks=true}
\urlstyle{same}  % don't use monospace font for urls
\usepackage{natbib}
\bibliographystyle{apalike}
\usepackage{longtable,booktabs}
\usepackage{graphicx,grffile}
\makeatletter
\def\maxwidth{\ifdim\Gin@nat@width>\linewidth\linewidth\else\Gin@nat@width\fi}
\def\maxheight{\ifdim\Gin@nat@height>\textheight\textheight\else\Gin@nat@height\fi}
\makeatother
% Scale images if necessary, so that they will not overflow the page
% margins by default, and it is still possible to overwrite the defaults
% using explicit options in \includegraphics[width, height, ...]{}
\setkeys{Gin}{width=\maxwidth,height=\maxheight,keepaspectratio}
\IfFileExists{parskip.sty}{%
\usepackage{parskip}
}{% else
\setlength{\parindent}{0pt}
\setlength{\parskip}{6pt plus 2pt minus 1pt}
}
\setlength{\emergencystretch}{3em}  % prevent overfull lines
\providecommand{\tightlist}{%
  \setlength{\itemsep}{0pt}\setlength{\parskip}{0pt}}
\setcounter{secnumdepth}{5}
% Redefines (sub)paragraphs to behave more like sections
\ifx\paragraph\undefined\else
\let\oldparagraph\paragraph
\renewcommand{\paragraph}[1]{\oldparagraph{#1}\mbox{}}
\fi
\ifx\subparagraph\undefined\else
\let\oldsubparagraph\subparagraph
\renewcommand{\subparagraph}[1]{\oldsubparagraph{#1}\mbox{}}
\fi

%%% Use protect on footnotes to avoid problems with footnotes in titles
\let\rmarkdownfootnote\footnote%
\def\footnote{\protect\rmarkdownfootnote}

%%% Change title format to be more compact
\usepackage{titling}

% Create subtitle command for use in maketitle
\providecommand{\subtitle}[1]{
  \posttitle{
    \begin{center}\large#1\end{center}
    }
}

\setlength{\droptitle}{-2em}

  \title{SUD KARDESLER}
    \pretitle{\vspace{\droptitle}\centering\huge}
  \posttitle{\par}
    \author{Dâr'ül-bedâî' Rejisörü İ.Galib}
    \preauthor{\centering\large\emph}
  \postauthor{\par}
      \predate{\centering\large\emph}
  \postdate{\par}
    \date{2019-11-23}

\usepackage{booktabs}
\usepackage{amsthm}
\makeatletter
\def\thm@space@setup{%
  \thm@preskip=8pt plus 2pt minus 4pt
  \thm@postskip=\thm@preskip
}
\makeatother

\begin{document}
\maketitle

{
\setcounter{tocdepth}{1}
\tableofcontents
}
\hypertarget{ozet-abstract}{%
\chapter{Ozet-Abstract}\label{ozet-abstract}}

\hypertarget{perde}{%
\chapter{1. PERDE}\label{perde}}

\begin{verbatim}
 [Vak'a Ada'da ve üç fasıl da aynı salonda geçer:
 Ada'da bir köşkün alt katında bahçeye ve uzaktan denize nâzır bir yemek ve oturma odası,       sağda geniş camlı bir kapı.. Dipte bir kapı ve bir koridor solda merdiven yukarı çıkılan       bir medhal sağ birinci planda bir kapı ]
\end{verbatim}

\hypertarget{birinci-meclis}{%
\section{Birinci Meclis}\label{birinci-meclis}}

\begin{verbatim}
 (Pakize-Mukaddes]
 Perde açılmadan iki kadın kahkası, perde açılırken devam eder. Pakize Mukaddes sahnenin        önünde ayaktadırlar. Pakize'nin elinde mektuplar vardır.\
\end{verbatim}

\textbf{Mukaddes} - (Gülmesi hafifleyerek) Aman yarabbi vallahi pek tuhaf çocuk Pakize Hanım.\\
\textbf{Pakize} - Değil mi? Pek eğlenceli mektup yazıyor. Bu yedinci mektubu yedisi de böyle.. Bundan yirmi gün önce kendisine bir rsmimi göndermiştim bak üçüncü mektubunda bunun için neler diyor (okur) "Süd kardeşim.. Resminiz geldi.. Aman yarabbi beni ne kadar sevindirdiniz.. Matbahdan {[}mutfaktan{]} kendimi kaptan köprüsüne dar attım artık kabıma sığamıyordum. fazla kaynamış süt gibi
kabarıyor, taşıyordum sevincimden bağırıyordum. Ve istiyordum ki hrkes benim sevincime uysun.. Artık başım geminin bacası gibi dumanlıydı, kalbime makine dairesinin bütün gürültüsü dolmuştu. Artık öyle istifi bozuk gemi gibi yan yatıp düşünmüyordum. Demek artık kimsesiz değilim diyordum. Ana yok, babam yok.. Onların da kimsesi yokmuş. Görüyorsunuz ya ben kimsesizlerin kimsesizi idim. Şimdi artık bir süt kardeşim var.. Yirmi beş sene evvel birbirine aynı süt ve meme ile birleşen dudaklarımız acaba yakın bir günde.. Aman kardeşim sakın bu sözüme fenâ bir manâ vermeyiniz.. Size karşı kardeşten başka.\\
\textbf{Mukaddes} - (Gülerek) A-a bu adetâ bir aşk mektubu..\\
\textbf{Pakize} - Ya onun gibi birşey. Pek eğlenceli yazıyor.. Zavallı Yaşar'cık\\
\textbf{Mukaddes} - Herhalde süt ninenin oğlu pek şen bir çocuk..\\
\textbf{Pakize} --Evet öyle galibâ!..\\
\textbf{Mukaddes} - Galibâ mı?.. Süt kardeşini tanımıyor musun..\\
\textbf{Pakize} - Hayır kardeşim tanımıyorum. Süt ninem öldüğü zaman ben altı yaşında idim. Onun yüzünü hayâl meyâl hatırlıyorum. Oğlu Yaşar da benim kadar bir bebekti.. Tam o zamanda merhûm babam memûriyetle Manisa'ya gitmek üzere idi. Yaşar'ı Dâr'ül-hayr'ın aşcısı Ömer Ağa isminde biri istemiş. Babam da süt ninemin hemşehrisi olan bu adamı kırmamak için çocuğu vermiş.. Aradan yirmi sene geçtiği halde şu tesâdüfe bakın süt ninemin çocuğu merhûm Ömer Ağa'nın evlatlığı Yaşar'cık asker oluyor. Hem de kocam gibi o da bahriyeye intisâb oluyor ve sonra günün birinde arayıp sorarak bu adrese anneme hitâben mektup yazıyor..\\
\textbf{Mukaddes} - Ey sonra?..\\
\textbf{Pakize} - Tabii annem ölmüş olduğundan bu mektuba ben cevap yazıyorum sonra artık muntazaman mektuplaşıyoruz. Zavallı çocuk bana hayatından askerliğinden bahseder hep aynı eğlenceli uslûbla.. O kadar ki belki yüzünü görsem tanıyamayacağım bir çocukla gâyet samîmî iki dost olduk.\\
\textbf{Mukaddes} - Pekiyi ama sizde de onun resmi yok mu?\\
\textbf{Pakize} - Var ama çocukken hep beraber ailece çektirilmiş bir resim. Düşün yirmi senelik solgun bir kağıt.. Yaşar bir buçuk senedir Trabzon'da imiş. Gemileri Karadeniz sahillerinde karakol yaparmış Son mektubunda nihâyet bir hafta kadar mezûniyet alabilidiğini ve Seyr-ü Sefain'in Akdeniz Vapuru'yla geleceğini bildiriyor. Bakalım Akdeniz ya bugün ya yarın gelecekmiş..\\
\textbf{Mukaddes} - A Akdeniz mi? Öyle ise Pakize'ciğim. Gözün aydın süt ninenin oğlu İstanbul'da..\\
\textbf{Pakize} - Ne gibi?\\
\textbf{Mukaddes} - Ne gibi olacak.. Akdeniz geleli dört saat oluyormuş. Sabahleyin erkenden bizim çamaşırcı Fatma söyledi.. O da oğlunu bekliyor..\\
\textbf{Pakize} - Oo öyle ise hiç şüphe yok gelmiştir. Bugn olmazsa yarın sabah muhakkak buraya gelir.\\

\hypertarget{ikinci-meclis}{%
\section{Ikinci Meclis}\label{ikinci-meclis}}

\begin{verbatim}
 [Evvelkiler-Zeliha-Terzi]
\end{verbatim}

\textbf{Zeliha} - Hanımefendi.. Beyefendinin esvabını getirdiler..\\
\textbf{Pakize} - Peki buraya getirsinler.. (Zeliha çıkar)\\
\textbf{Mukaddes} - Necdet Bey'in askerliği daha bitmedi mi kuzum.\\
\textbf{Pakize} - Kocamın mı? Daha iki ayı var.. Onunki askerlik değil mirasyedilik kardeşim.. Hergün İstinye'de oturan miralayını motorla Heybeli'ye götürüp getirmekten başka bir iş yaptığı yok; motordan, otomobilden anlar diye bahriye erkanından miralay Rasim Bey'in maiyetine verdik bu rahatını da dayıma medyûndur {[}borçludur{]} ya.. Yüzünü bile görmediği damadına uzaktan yardım etti (giren terzi kızına) Ha bonjur matmazel beyin sivil kostümünü mü getirdiniz? Buraya bırakın. Faturası da yanınızda mı?\\
\textbf{Terzi} - Evet efendim (uzatır) Efendim Beyefendi kostümünü bir def'a görmek isterlerse\ldots{}\\
\textbf{Pakize} - Beyefendi evde yok kızım bakar. Şayet bir kusur bulursa artık lütfen kendileri getirirler. Haydi yahut vaktiniz varsa biraz bekleyin yarıma kadar gelir..\\
\textbf{Kız} - Peki efendim.. (çekilir)\\
\textbf{Mukaddes} - (Pakize'nin terziye söz söylerken aldığı şiveye dikkat ederek) Ne o Paki{[}ze{]} galibâ yine kocanla aranız şeker renk?\\
\textbf{Pakize} - Nasıl da bildin Mukaddes'ciğim.. Bu adama karşı içimde yenilmez bir kin var..\\
\textbf{Mukaddes} - A.. Aman yarabbi o nasıl söz kardeşim..\\
\textbf{Pakize} - Evet bir aydan beri..\\
\textbf{Mukaddes} - Kocana karşı, kin mi? Peki ama böyle kin besleyecek kadar ne yaptı da..\\
\textbf{Pakize} - (Keserek) Ne yapacak? Evlendiğinin senesinde her erkek karısına ne yaparsa onu.. Beni aldattı.. Hem de bir kaldırım çiçeğine..\\
\textbf{Mukaddes} - (Hayretinden gözlerini açıp fırlar) Ne? Necdet Bey seni aldattı mı?~(Ağlayacak gibi) Aman yarabbi şu başıma gelen felakete bakınç Ah herşey aklıma gelirdi de bu gelmezdi işte.. Ah böyle inci gibi bir tazenin üstüne ha hem de sokak şırfıntılarıyla.. Ah..\\
\textbf{Pakize} - (Mukaddes'in ani teessür ve ağlamasına şaşarak) Aman kardeşim yüreğine yazık nafile yere ne ağlıyorsun?..\\
\textbf{Mukaddes} - Ah Pakize'ciğim.. Nasıl ağlamam.. Ya benimki de beni aldatırsa ben ne yaparım (Ağlamaya devam eder)\\
\textbf{Pakize} - Hayır hayır ağlama.. Benim için niye kendini üzüyorsun? Eğer bir gün seninki de seni aldatırsa o vakit ağlarsın. Hem hayır niye ağlayacaksın.. Niye ağlayacağız? Erkeklerin hepsi bir kadının bir damla gözyaşına değmez. Onları aynı silahla yola getirmeli..\\
\textbf{Mukaddes} - (Gözlerini silerek) Ama bilmezsin Pakize'ciğim sanki kendi kocam üstüme hiyânet etmiş gibi içime dokundu eyy.. Peki ne oldu nasıl oldu da Necdet Bey'in hıyânetine vâkıf oldun.. Elinle mi yakaladın?..\\
\textbf{Pakize} - Ah nerede! Elimle tutsaydım hiç olmazsa hıncımı alırdım. Tutamadım fakat elimdeki isbât gâyet kuvvetli!..
(Dışarıdan Necdet'in sesi)\\
\textbf{Necdet} - (Dışarıdan) Gel kızım.. Gel. Bir bakalım ne olur ne olmaz. Belki potları vardır.\\
\textbf{Pakize} - Hah.. İşte Necdet.. Aman belli etmeyelim.. (Eliyle işaret)\\
\textbf{Mukaddes} - A tabii kardeşim.. İşte (Susar kaşlarını çatar)\\

\hypertarget{ucuncu-meclis}{%
\section{Üçüncü Meclis}\label{ucuncu-meclis}}

\begin{verbatim}
 [Evvelkiler- Necdet- Roza-Zeliha]
\end{verbatim}

\textbf{Necdet} - (Bahriye neferi, ellerinde yağlı eldivenler\ldots) Aç aç matmazel.. (Hanımları görünce askerâne bir selamla) Hanımlar, merhabâ\ldots(Kendi kendine güler, onların abûs çehrelerini görünce susar ve karısının o halini tabiî bularak) Ha.. (Mukaddes'e teveccühle) Efendim.. (Elini uzatır)\\
\textbf{Mukaddes} - (Necdet'in uzattığı ele bakar.. Yarım bir dudak büküşle döner ve çok nazlı, dargın bir şive ile) Teşekkür ederim Necdet Bey, safâ geldiniz..\\
\textbf{Necdet} - \textless\textgreater.. Siz de safâ geldiniz Mukaddes Hanım, nasılsınız..\\
\textbf{Mukaddes} - Teşekkür ederim\\
\textbf{Necdet} - Efendim.. Ha, oh oh.. Kemal Bey ne alemde?\\
\textbf{Mukaddes} - Zevcim mi? İyidir kimbilir o da nerelerde?\\
\textbf{Necdet} - Tabiî kalemde değil mi efendim.. (Karısına) Eyy.. Pakize'ciğim (Pakize onun uzattığı eli reddeder)Ya.. Yine mi? (Her ikisine birden) Latîfe mi ediyorsunuz Allahaşkınıza?.. Ben girince niye böyle, döndünüz kaldınız canım? Ha anladım mutlaka beni çekiştiriyordunuz öyle mi? Mukaddes Hanım.. (Onların hallerine kızarak) Eyy iyi vallahi yine poyraz esiyor galibâ!!!..\\
\textbf{Terzi Kızı} - (Bu hale güler) Hah..Hah..\\
\textbf{Pakize} - (Yüksek sesle) Matmazel.. Burada komedi oynanmıyor. Lütfen terbiyenizi takınınız.. (Kocasına) Siz de çoluğa çocuğa gülünç oluyorsunuz.. Biraz daha ciddi olunuz.\\
\textbf{Necdet} - (Ciddi) Pekala! Pekala! (Terziye) Ha.. Bu mu matmazel ?.. Fena değil. Şöyle durun bakayım.. (Kızın arkasına ceketi koyar muayene eder)\\
\textbf{Mukaddes} - (Yavaşca Pakize'ye) Kardeşim ben gideyim.. Bu dakikada hiç bir erkek yüzü görmeye tahammül edemeyeceğim.\\
\textbf{Pakize} - Hakkın var Mukaddes'ciğim güle güle (şapka)\\
\textbf{Necdet} - (Terzi ile meşgul) Ha bakın şuraları galibâ burası da biraz düşük.. Durun bakayım şöyle..\\
\textbf{Pakize} - (Kocasına hiddetle) Beyefendi matmazel manken değil ceketinizi kendi sırtınızda muayene ediniz. Matmazel verin o ceketi Zeliha'ya, Zeliha götür o kostümü beyin odasına haydi bakalım! (Zeliha çıkar)\\
\textbf{Necdet} - An'allah ma'as-sabirin.. Peki matmazel siz de gidiniz teşekkür ederim. eğer bir kusuru varsa ben kendim mağazaya getiririm.\\
\textbf{Terzi} - Pek efendim\ldots{} Mersi\ldots{} (Çıkar)\\
\textbf{Mukaddes} - (Gitmek üzere) Allahaısmarladık Pakize'ciğim, bedbaht kardeşim!\\
\textbf{Necdet} - (Beraber) An'allah ma'asabirin\\
\textbf{Pakize} - Güle güle kardeşim.\\
\textbf{Mukaddes} - Sizi beklerim iki gözüm\ldots{} Beni unutmayın..\\
\textbf{Pakize} - Hay hay gelirim kardeşim güle güle\ldots{}\\
\textbf{Necdet} - (Mukaddes'e) Gidiyor musunuz Mukaddes Hanım?..\\
\textbf{Mukaddes} - Evet efendim.\\
\textbf{Necdet} - Affedersiniz efendim. Müsâadenizle bir şey soracağım\ldots{}\\
\textbf{Mukaddes} - Buyurun\ldots{}\\
\textbf{Necdet} - Sizi biraz değişmiş görüyorum. Sakın birşeye müteessir olmayınız.\\
\textbf{Necdet} - Sizi biraz değişmiş görüyorum. Sakın birşeye müteessir olmayınız.\\
\textbf{Necdet} - (Şaşırarak) Ya!.. Söyleyiniz de bu teessürün yarısını ben alayım.\\
\textbf{Mukaddes} - Siz mi?.. Hım\ldots{} Bir erkek\ldots{}\\
\textbf{Pakize} - (Mukaddes'in bir poy kırmasına mani olmak için) Size ait değil beyefendi.. Bu kadar mütecessis olmayınız.. Biz kadınca dertleşiyorduk..\\
\textbf{Necdet} - (Askerce bir selam) O halde affedersiniz.. Kemal Bey'e selam Mukaddes Hanım\ldots{}\\
\textbf{Mukaddes} - Söylerim.. Allahaısmarladık Pakize'ciğim. (Çıkar)\\
\textbf{Pakize} - gule gule kardesim\\

\hypertarget{dorduncu-meclis}{%
\section{Dördüncü Meclîs}\label{dorduncu-meclis}}

\begin{verbatim}
 [Necdet-Pakize-Zeliha)
\end{verbatim}

\textbf{Necdet} - (Karısına yaklaşarak) Pakize bu nedir, Mukaddes Hanım bana dargın mı kuzum?\\
\textbf{Pakize} - Hayır hayır hemen üstünüze almayınız\ldots{} Kabahatli olduğunuz ne kadar belli..\\
\textbf{Necdet} - Ne demek?\\
\textbf{Pakize} - Tabiî baksanıza!.. Hemen alınıyorsunuz! (Sükût)\\
\textbf{Necdet} - (Yaklaşarak) Peki karıcığım sen niye böylesin? Hâlâ mı Pakize hâlâ mı.. Bir ay oldu..\\
\textbf{Pakize} - Rica ederim bana yaklaşmayınız..\\
\textbf{Necdet} - Demek el'ân hırsınız geçmedi.. Seni kinci seni..\\
\textbf{Pakize} - Hım!!..Kin mi? Size mi? Değmez ki.. İndimde o kadar ehemmiyetiniz yo ki nah işte şu resim gibi..\\
\textbf{Necdet} - Resim gibi mi?\\
\textbf{Pakize} - Evet bir resim bir gölge.. Siz yoksunuz..\\
\textbf{Necdet} - Yok muyum? Öyle bir varım ki (sarılmak ister) karıcığım.\\
\textbf{Pakize} - Ricâ ederim.\\
\textbf{Necdet} - Barışalım artık ya.. (Zeliha'nın geldiğini görerek çekilir)\\
\textbf{Pakize} - Sus hizmetçi.. Ne var Zeliha?\\
\textbf{Zeliha} - Hanıefendi bir ricâm varda.. Onu ricâ edeceğim..\\
\textbf{Necdet} - Ne imiş?\\
\textbf{Pakize} - (Kocasının sesini bastırarak) Ne imiş bakalım?\\
\textbf{Zeliha} - Bir hafta izin isteyecektimde..\\
\textbf{Pakize} - Bir hafta izin mi.. Neden?\\
\textbf{Zeliha} - Şey nişanlım annesinin evine gelmişde oraya gideceğim.\\
\textbf{Pakize} - Nereden gelmiş?\\
\textbf{Zeliha} - Askerden..\\
\textbf{Necdet} - Senin nişanlın mı var kız?\\
\textbf{Zeliha} - Ya.. Evet efendim. Ben..\\
\textbf{Pakize} - Bana söyle.. Eyy?..\\
\textbf{Zeliha} -Evet efendim nişanlım var.. Ama iki de çocuğum var..\\
\textbf{Pakize} - Çocuğun mu var?\\
\textbf{Zeliha} - Evet ikiz.\\
\textbf{Necdet} - Nişanlın? Çocuğun.. Nikâhtan evvel, hem de ikiz, bârî bir olsaydı değil mi Pakize?\\
\textbf{Zeliha} - Ama efendim nişanlım askerden gelince hemen nikâhlanacağız beni hâlâ seviyor.\\
\textbf{Pakize} - Anlaşıldı.. Pekala.. Pekala..\\
\textbf{Zeliha} - İzin verirseniz bu akşam gideceğim..\\
\textbf{Pakize} - Bana bak kızım.. bir hafta izin olmaz sonra bize yemeği kim pişirecek. Aşcı kadın gelsin, ondan sonra..\\
\textbf{Zeliha} - Efendim aşcı kadın yarın gelecek..\\
\textbf{Pakize} - Peki peki bakalım git sonra görüşürüz. (Kapı çalınır) Git bak kim geldi.\\
\textbf{Zeliha} - Peki ama efendim?\\
\textbf{Pakize} - (Sert) Sana ne diyorsam onu yap..\\
\textbf{Zeliha} - Peki efendim.(Çıkar)\\
\textbf{Necdet} - Zavallı kız sen de amma paylıyorsun ha.. Baksana nişanlısının çocukları da varmış. Hem de ikiz..\\
\textbf{Pakize} - Evet, aşk meselelerinde beyefendinin yüreği yufkadır. Değil mi?\\
\textbf{Necdet} - Ama Pakize'ciğim!\\
\textbf{Pakize} - Kâfî!.. Evimizde bu kızdan başka kimse kalmadı. Kapıyı açmaktan başka bir işe yaramaya 12 yaşındaki Fatma'nın elinde mi kalacağız? Göndermeyeceğim işte.\\
\textbf{Necdet} - Canım pekala aman..\\
\textbf{Pakize} - Sus kızdırma beni..\\

\hypertarget{besinci-meclis}{%
\section{Beşinci Meclîs}\label{besinci-meclis}}

\textbf{Zeliha} - Efendim elektrikçi gelmiş\ldots{}\\
\textbf{Pakize} - Hah geldi mi söyle gelsin..\\
\textbf{Lami} - Bonjur efendim bonjur..\\
\textbf{Necdet} - Bonjur Lami Bey elektrik instalasyonunun planını yapmak için geldiniz değil mi?\\
\textbf{Lami} - Evet efendim; iki def'adır geliyorum. Sizi bulamıyorum. İyi oldu bu tesâdüf.. Ki siz de,,\\
\textbf{Pakize} - Affedersiniz beyim. Bu işler için benimle görüşeceksiniz efendi ile değil\ldots{} Oturunuz efendim.\\
\textbf{Lami} - Peki efendim (Sükût)\\
\textbf{Pakize} - Sizi rahatsız ettik affedersiniz.\\
\textbf{Lami} - Beis yok efendim.. Mağazamda adamım var.. Şimdi efendim cereyanıbeldeden alacağız değil mi. Bu hus3usta lâzım gelen muâmelât-ı resmiye bitti değil mi efendim?\\
\textbf{Necdet} - Bu işler bitti efendim (Kalkarak)\\
\textbf{Lami} - Pek güzel efendim. Yukarı küçük köşke de.. elektrik alacaksınız değil mi efendim?\\
\textbf{Necdet} - Hayır efendim.\\
\textbf{Pakize} - evet efendim. Alacağız. Lami Bey. Bu evde yalnız benim reyim sorulur. Efendi burada hiç, hiç, hiçtir.\\
\textbf{Necdet} - Oo Pakize ricâ ederim.. Çok oluyorsun artık..\\
\textbf{Pakize} - Susunuz beyefendi.. Evet siz burada solda sıfırdan başka birşey değilsiniz (Necdet kendini hiddetle salıncak sandalyeye atarken Lami'e) Ne diyordum efendim? Evet her iki daireye de elektrik alacağız..\\
\textbf{Lami} - Ya.. Hay hay efendim. Bundan kolay yok hanımefendi.\\
\textbf{Necdet} - Sonra her odaya bir zil düğmesi istiyorum.\\
\textbf{Lami} - Ya.. Hay hay bundan kolay birşey yok beyefendi.\\
\textbf{Pakize} - (Şiddetle kalkarak) Ricâ ederim Lami Bey zil mil istemez. Otel odaları gibi.. İstemiyorum. Hem bende bir mecmua var size göstermek için sakladım, orada elektiriğe dair çok zarîf resimler var, gideyim getireyim. (Çıkar)\\

\hypertarget{altinci-meclis}{%
\section{Altıncı Meclîs}\label{altinci-meclis}}

\begin{verbatim}
 [Necmi-Lami]
\end{verbatim}

\textbf{Lami} - (Yerinden kalkıp Necdet'in yanına gelerek) Affedersiniz efendim birşey soracağım..\\
\textbf{Necdet} - Buyurun.\\
\textbf{Lami} - Bu elektrik instalasyonu hakkında kimin sözünü dinşeyeceğim, sizin mi yoksa hanımefendinin mi?\\
\textbf{Necdet} - Kimin mi? (vakfe {[}durma/durak yeri{]}) Lami Bey evli misiniz?\\
\textbf{Lami} - Evet efendim\ldots{} Üç seneden beri\ldots{}\\
\textbf{Necdet} - İyi\ldots{} Karınızı aldatıyor musunuz?\\
\textbf{Lami} - Karımı mı? Bilmiyorum efendim\ldots{} Daha..\\
\textbf{Necdet} - Ha.. Onu aldattığınız zaman -zira günün birinde mutlaka aldatacaksınız- sakın yakalanmayın.\\
\textbf{Lami} - Vallahi.. (Gülerek) Bir parça şaşırdım da.. Ne söyleyeceğimi bilemiyorum.\\
\textbf{Necdet} - Anlatayım. Lami Bey, patlıyorum tahammülüm kalmadı şu anda bir sırdaşa, derdimi dökecek bir arkadaşa muhtacım (Lami kalkar) Bir dakika beni dinleyin. Vaktinizin geçeceğinden korkarsanız ücretinize birşey ilave ederiz.\\
\textbf{Lami} - Öyle olsun efendim. (Oturur)\\
\textbf{Necdet} - Affedersiniz. Öteki sandalyeye oturur musunuz? Zira yanınıza şöylece oturursam daha iyi, daha rahat olacak..\\
\textbf{Lami} - Hay hay byurun efendim (Yerleşir)\\
\textbf{Necdet} - Hah şimdi Lami Bey bakın ben altı senedir evliyim. Bu altı sene zarfında karımı aldatmak hatırımdan geçmiyordu. Fakat geçen ay nasılsa elimden bir kaza çıktı.Topu topu bir def'acık karımı aldattım. Bu macerayı size anlatıvereyim. İstinye'den motorla yalnız olarak aşağıya doğru son süratle iniyordum. Tam Arnavutköyü burnunu döneceğim sırada karşıma birdenbire zarif bir futa (?) çıkmasın mı? Beyaz ketenler giymiş.. Kolları buraya kadar (İşaret) kesik, gür lepiska saçlara güneş çarpmış pırıl pırıl yanıyor.. Canlı bir tehlike üstüme geliyor. Aman.. Gözüm kamaşmıştı.. Herşeyden evvel makinemi durdurtmak lazımdı.. Hemen istop ettim dümenimi sancağa aldım fakat ne olsa vakit geçmişti.. Küt! Futasının kıçına çarpmayayım mı?\\
\textbf{Lami} - Aman yarabbi ne kontakt!\\
\textbf{Necdet} - Evet tekneler arasında\ldots{}\\
\textbf{Lami} - Sonra\\
\textbf{Necdet} - Futa bir tığ gibi hafif, belki on metre ileriye kaymasıyla beraber alabora olması da bir oldu..\\
\textbf{Lami} - Aman!\\
\textbf{Necdet} - Evet aman!.. Ve tabiî ilk işim bir hamlede havuza düşen bir kelebek gibi deniz içinde çırpınan bu kadını kurtarıp motora almak oldu..\\
\textbf{Lami} - Ey sonra..\\
\textbf{Necdet} - Sonra.. Aygın baygın hanımı futasıyla beraber Bebek'teki yalısına götürdük.. Meğerse nâzeninim pek meşhur bir yeni tüccarın kapatması değil miymiş.. Koca yalıda kimseler yok.. Bir hizmetçi, bir o, bir de ben.. Bana bir izzet ikrâm..\\
\textbf{Lami} - Oh.. Oh..\\
\textbf{Necdet} - Kadın soyundu, kurundu.. Giyindi.\\
\textbf{Lami} - Sizin yanınızda mı?\\
\textbf{Necdet} - Evet.. Ama aramızda paravan vardı. Ne ise.. Çay konyak çay konyak birini içip ötekini tokuşturmaya başladık.\\
\textbf{Lami} - (Gülerek) Oh kontakt kontakt üstüne!.\\
\textbf{Necdet} - Evet kadehler arasında böyle iki saat geçti.. Siz benim yerimde olsanız ne yapardınız?\\
\textbf{Lami} - Ne mi yapardım? Kendimi cereyan-ı elektr.. cereyan-ı tabiîye bırakırdım.\\
\textbf{Necdet} -Gayet tabiî biz de kendimizi akıntıya kaptırdık fakat kahpe kapatma bırakır mı beni.. O gün sarhoşlukla kaptırdığım adresime üç gün sonra bir mektup.. Hem de buraya.. Eve..
\textbf{Lami} - Aman.. ~
\textbf{Necdet} - Demeğe kalmadan mektubumu karım okur -zira karım mektuplarımın hepsini açar- işte azizim o günden beri işkence başladı İki saatlik bir rüya hayatından sonra evimde otuz günden beri kâbûs geçiriyorum ne dersiniz? (Sükût)\\
\textbf{Lami} - Beyefendi müşterimsiniz halinize acırım.\\
\textbf{Necdet} - Teşekkür ederim\ldots{} Ama benim halimi tehvîn etmez {[}kolaylaştırmaz/hafifletmez{]}. Beni affediniz. Ben ne yapayım bana ne tavsiye edersiniz? Bir plan da bunun için düşünün\ldots{} ~\\
\textbf{LAMİ} - Vallahi beyefendi. Bugün yalnız elektrik instalasyonuna dair bir plan yapmaya geldim. Başka bir gün gelir bu sizin sevda instalasyonu için istediğiniz planı da yapmaya çalışırım.\\
\textbf{Necdet} - Öyle olsun.. Teşekkür ederim.. İşte karım (İkisi de kalkarlar)\\

\hypertarget{yedinci-meclis}{%
\section{Yedinci Meclîs}\label{yedinci-meclis}}

\begin{verbatim}
 [Evvelkiler-Pakize (girer)-sonra Pakize]
\end{verbatim}

\textbf{Pakize} - (Elinde mecmua ile) İşte Lami Bey kataloğu getirdim. Fakat siz yukarıki köşkü gezmiş miydiniz?\\
\textbf{Lami} - Hayır efendim..\\
\textbf{Pakize} - Şu halde evvel orayı bir görseniz.\\
\textbf{Lami} - Hay hay efendim.. (Necdet'le beraber çıkmaya hazırlanır)\\
\textbf{Pakize} - Rica ederim beyim Necdet Bey'in size söyleyeceği şeylere zerre kadar ehemmiyet vermemenizi tavsiye ederim. Burada benden başka kimsenin arzusu ve emri olamaz..\\
\textbf{Lami} - Pek güzel hanımefendi.\\
\textbf{Necdet} - Elimden bir kaza çıkacak benim!\\
\textbf{Lami} - Beyefendinin bana yol göstermesine lütfen müsaade eder misiniz efendim?\\
\textbf{Pakize} - Göstersin..\\
\textbf{Necdet} - Bahçeye çıkın sola doğru yürüyün geliyorum efendim..\\
\textbf{Lami} - Peki efendim (Çıkarken çantasıı oturduğu yerde unutur)\\
\textbf{Necdet} - (Şiddetle karısına dönerek) Bana bak Pakize sâhî söylüyorum sen artık fazla gidiyorsun bu derece olmaz.\\
\textbf{Pakize} - Ricâ ederim.. (Lami'nin çıktığı tarafı gösterir)\\
\textbf{Necdet} - Yok.. Artık ben ricâ ederim Pakize. Bunlar ne demek oluyor izâh etmeli (Kapı zili)\\
\textbf{Pakize} - Kapı çalınıyor sonra daha sonra konuşuruz\\
\textbf{Necdet} - Hayır şimdi..\\
\textbf{Zeliha} - Küçük hanım küçük hanım..\\
\textbf{Pakize} - Ne var Zeliha..\\
\textbf{Zeliha} - Beklediğiniz geldi..\\
\textbf{Necdet} - Kim? Hanımın beklediği kim?\\
\textbf{Pakize} - Hele Zeliha'yı tekdîr etmeyiniz. İyi söyledi beklediğim geldi..\\
\textbf{Zeliha} - Evet efendim ya o geldi.\\
\textbf{Pakize} - Koş getir. Yaşar'dır, süt kardeşim..\\

\hypertarget{sekizinci-meclis}{%
\section{Sekizinci Meclîs}\label{sekizinci-meclis}}

\begin{verbatim}
 [Evvelkiler-Şemsi]
\end{verbatim}

\textbf{Şemsi} - (Kapının önünde gözükür elinde torba tam bahriyeli) Dışarıda bekleyemedim affedersiniz. Ben Yaşar'ım. Pakize Hanım'ın süt kardeşi.\\
\textbf{Pakize} - Nasıl Yaşar siz misiniz?\\
\textbf{Şemsi} - Evet süt kardeşiniz.. Karadeniz Sevâhil Kumadanlığı maiyetinden Aytaş Karakol Gemisi aşcısı Yaşar..\\
\textbf{Pakize} - Yaşar siz ha!..\\
\textbf{Şemsi} - Evet ben.. Siz de süt kardeşim Pakize Hanım'sınız değil mi?. Kalbim de öyle söylüyordu.\\
\textbf{Pakize} - Ne güzel çocuk..\\
\textbf{Şemsi} - (Elindekini yere bırakarak) Müsaade eder misiniz?\\
\textbf{Pakize} - A.. N demek kardeşim (Şemsi Pakize'yi öpmeye başlar)\\
\textbf{Necdet} - Oo.. bu sütü bozuk bu kardeş hâlâ bırakmadı (Öksürür)\\
\textbf{Şemsi} - Zaten içeriye girer girmez sizi tanıdım. Gönderdiğiniz resim daima yanımda. Buradan ayırmıyorum. (Çıkarır resmi öper)\\
\textbf{Necdet} - Vay, vay, vay resim de gönderilmiş ha!.. (Öksürür)\\
\textbf{Şemsi} - (Necdet'e dönerek) Merhaba bahriyeli.. Bu da kim?.\\
\textbf{Necdet} - Süt hemşirenizin zevciyim.. Necdet..\\
\textbf{Şemsi} - Ya bizim enişte ha.. Hem bir meslekten ha.. Öpüşelim süt enişteciğim! (Necdet'i öper)\\
\textbf{Necdet} - Kirpi gibi de batıyor\ldots{}\\
\textbf{Şemsi} - (Pakize'ye) Eyy süt kardeşim beni hatırlayabilidiniz mi?\\
\textbf{Pakize} - Hayâl meyâl Yaşar.\\
\textbf{Şemsi} - Hayâl meyâl mi.. Bakın ben sizi derhal tanıdım sanki yirmi sene evvel bıraktığım gibi. Mini mini iken şuralarda ne koşar ne oynardık alt alta üst üste. (Çoşup koşar yine öper) Ah benim cici kardeşim yine kavuştuk.. Gelin çocukluğumuzdaki gibi yine öpüşelim!\\
\textbf{Necdet} - Eyy Yaşar Efendi.. Yeter ya.. Şimdi artık çocuk değilsiniz büyüdünüz değil mi?\\
\textbf{Şemsi} - Pekala, darılmayın enişte bey ben onun kardeşiyim..\\
\textbf{Necdet} - Anlaşıldı ya.. Ne ise ne vakit dönüyorsunuz bakalım?\\
\textbf{Şemsi} - İznim bir hafta demek ki on gün izinliyim!\\
\textbf{Necdet} - İyi hesap..\\
\textbf{Pakize} - On gün ha.. Ne saadet.. Burada kalırsınız..\\
\textbf{Şemsi} - Ma'al-memnuniye.\\
\textbf{Necdet} - Dur bakalım hanım.. Telaş etmeyin çocuk belki daha serbest bulunmak ister.. Belki İstanbul'da gideceği yer vardır..\\
\textbf{Şemsi} - Benim kimsem yok ki bir tanecik süt kardesimden mâadâ..\\
\textbf{Pakize} - Zavallı Yaşar'cık, öyle ya onun başka kimi var? (Şemsi'ye) Burası senin evin Yaşar.. İznin bitinceye kadar burada kal, ye, iç, keyfine bak..\\
\textbf{Şemsi} - Fena olmaz.. Teşekkür ederim kardeşim..\\
\textbf{Necdet} - Ne iyi, bunlar kendi kendilerine gelin güvey oluyorlar..\\
\textbf{Pakize} - (Zeliha'ya) Zeliha Yaşar'ı balkonlu odaya yerleştirirsin.\\
\textbf{Necdet} - Balkonlu oda mı? Orası benim odan.. Ben nerede yatarım.\\
\textbf{Pakize} - Siz arka odada kalırsınız.\\
\textbf{Necdet} - Hayır biliyorsun, arka oda rutubetlidir Pakize..\\
\textbf{Şemsi} - Fenâ mı? Kendinizi denizde farz ediniz (Pakize'ye) Yalnız aksilik nerde? Benim değişecek üstüm başım da yok ne çamaşır ne esvap\\
\textbf{Pakize} - Var var. Kardeşim siz de zevcimin boyundasınız. Onun esvapları size gelir.. Hele iyi oldu tam bu günde onun terziden yeni esvapları geldi..\\
\textbf{Şemsi} - Bakın ne tuhaf tesâdüf!..\\
\textbf{Pakize} - Zeliha beyin yeni esvaplarını süt kardeşime verirsin..\\
\textbf{Necdet} - A.. Yok.. İşte bu olmaz benim yeni esvabım..\\
\textbf{Pakize} - Siz de susun (Şemsi'ye) Kardeşim çamaşırınız var mı?\\
\textbf{Şemsi} - Var ama, üstümde! Başka da yok..\\
\textbf{Pakize} - Zeliha beyin çamaşır dolabından alırsın olmaz mı?\\
\textbf{Necdet} - Şimdi de çamaşırlarım.. Ey sonra süt kardeşe biraz da kolonya istemez mi?\\
\textbf{Şemsi} - (Zeliha'ya) Traştan sonra biraz kolonya verirseniz..\\
\textbf{Necdet} - Pakize diyorum.\\
\textbf{Pakize} - Susunuz diyorum. Süt kardeşimin nesi var.. Gül gibi çocuk..\\
\textbf{Necdet} - Pekala. Biz de diken demedik yalnız..\\
\textbf{Pakize} - Kâfî.. Senelerden beri deniz üstünde askerlik eden böyle bir kahraman her şeye layıktır. Siz dışarıya! Elektrikçi sizi bekliyor.\\
\textbf{Necdet} - Öyle ya ben dışarıya değil mi? Dur sen..(Çıkmak üzere)\\
\textbf{Şemsi} - (Necdet çıkarken önleyerek) Enişteciğim bana darılmadınız ya.. Gelin öpüşelim..\\
\textbf{Necdet} - Hayır artık öpüşme yeter.. Yaşar Efendi (Çıkar)\\
\textbf{Şemsi} - Nesi var. Yoksa buraya gelir gelmez havayı mı değiştirdim..\\
\textbf{Pakize} - Yok canım siz ona ehemmiyet vermeyin siz şuraya oturun bakalım, rahat edin. Bakın burada buzlu şurup var içer misiniz?\\
\textbf{Şemsi} - Hastaya çorba mı soruyor synuz? Hararetim var kardeşim.. Elbette içerim.\\
\textbf{Pakize} - Alın.. Siz biraz nefes alın bu esnâda biz de odanızı hazırlarız. (Şemsi'nın bıraktığı torbayı alıp Zeliha'ya verir.)\\
\textbf{Şemsi} - Hay Allah razı olsun sizden.. (Şezlonga oturup sallanırken) İnsan böyle de yaşarmış ha..\\
\textbf{Pakize} - Evet bazan böyle yaşayanlar hayatlarının lıymetini bilmiyorlar. Haydi Zeliha alık alık bakma yürü önden (Çıkarlar)\\

\hypertarget{dokuzuncu-meclis}{%
\section{Dokuzuncu Meclis}\label{dokuzuncu-meclis}}

\begin{verbatim}
 [Şemsi-Lami]
\end{verbatim}

\textbf{Şemsi} - Eyy resm-i kabul mükemmeldi (Düşünür) Adam sen de bir haftalık beylik beyliktir.~(Şurubunu çeker)
\textbf{Lami} - (Gözükerek) Çantamı nereye koymuşum? Ha işte.. (Şemsi'yi görerek bağırır) Açç Hayırdır inşaallah.. Şemsisen burada ne arıyorsun..\\
\textbf{Şemsi} - (Kalkarak) Vayy Lami..\\
\textbf{Lami} - Şemsi'ciğim (Kucaklaşma) Ne vakit geldin? Burada ne arıyorsun?\\
\textbf{Şemsi} - (Kendi) Vay canına ( Lami'ye) Sus.. Beni ismimle çağırma!\\
\textbf{Lami} - Ne demek.. Yani Şemsi demeyecek miyim?\\
\textbf{Şemsi} - Sus diyorum. Ben burada Şemsi değilim. Fatih'li elektrikçi Şemsi değilim. Ben Yaşar'ım.\\
\textbf{Lami} - Ulan yoksa askerden mi kaçtın?\\
\textbf{Şemsi} - Yok canım. Bir buçuk senedir deniz üstünde imanım gevriyor.\\
\textbf{Lami} - Eyy?\\
\textbf{Şemsi} - Eyy'si.. Bizim gemide bön bir arkadaş vardı. Geminin aşcısı Yaşar. Pakize Hanım'ın süt kardeşi imiş.\\
\textbf{Lami} - Ne.. Bizim yeni müşteriler..\\
\textbf{Şemsi} - Ya.. Bu Yaşar iki aydan beri süt hemşireşi Pakize Hanım'a benim vasıtamla mektuplar yazar dururdu. İnsan tanımadığı bir kadınla mektuplaşınca tuhaf bir heyecan içinde kalıyor. Sanki mektuplaşan Yaşar değil bendim. Bir gün hanımın resmini istedik. Resim geldi. Nefis. incecik. gül gibi, deniz kızı gibi bir kadın.\\
\textbf{Lami} - Ay desene resminden deniz kızına tutuluverdin.\\
\textbf{Şemsi} - Hem de nasıl.. Mecnun gibi. Gece gündüz onu düşünüyorum. Hem düşünebiliyor musun bir buçuk sene süren bir mahrumiyetten sonra..\\
\textbf{Lami} - Şemsi!\\
\textbf{Şemsi} - Uzatmayalım. Geçen hagta Yaşar da ben de onar gün izin kopardık. O zaman Leyla'mı görebilmek için delice bir fikre kapıldım. Kendimi Yaşar'ın yerine koyup buraya gelmek vaktinden bir bir hafta çalmak..\\
\textbf{Lami} - Eyy peki ama Pakize Hanım süt kardeşini tanımıyor muymuş?\\
\textbf{Şemsi} - Yirmiseneden beri birbirlerini görmemişler..\\
\textbf{Lami} - Oh oh! Yaşar bu becayişe razı oldu mu?\\
\textbf{Şemsi} - İster istemez. Allem ettim kallem ettim. Zaten yufka olan gönlünü yumuşattım. Hem o kadar ki; işi beraberce pişirdik.. İzin tezkeresini de, süt hemşiresinin fotoğrafını da bana verdi. Buraya geldim. Berikiler de yuttu.\\
\textbf{Lami} - Eyy.. Ulan ya karın! Sen evliydin zannederim.\\
\textbf{Şemsi} - Karım.. Karım Fatih'te.\\
\textbf{Lami} - Zavallı kadın da seni bekler durur değil mi?\\
\textbf{Şemsi} - Hayır zira izin aldığımı kendisine yazmadım. On günlük iznimin yarısını süt kardeşimin yanında geçirmeye karar verdim. Bizimkne de süvarimle aram açıldığını bundan dolayı bana izin vermeyeceğini yazdım. Oyun nasıl?\\
\textbf{Lami} - Acıklı!.. Vicdansız şey!\\
\textbf{Şemsi} - Ah Lami'ciğim karımla süt kardeşimin arasındaki farkı bilsen.. biri örekinden bin kat güzel?\\
\textbf{Lami} - Kim karın mı?\\
\textbf{Şemsi} - Bırak karımı Allah'ı seversen. Gelecek hafta da ona gider gönlünü alırım. Şimdi sen bana bunlara dair malumat ver. Madem ki müşterilerinmiş!\\
\textbf{Lami} - Eyy Şemsi'ciğim talihin varmış. Tam zamanında aralarına düştün.\\
\textbf{Şemsi} - Neden yoksa araları mı açık.\\
\textbf{Lami} - Sorma kocası süt kardeşini aldatmış..\\
\textbf{Şemsi} - Nasıl?\\
\textbf{Lami} - Nasıl mı? Evvela denize sonra ağına düşürdüğü bir kadınla..\\
\textbf{Şemsi} -Ne süt kardeşim de bunu öğrenmiş öyle mi?\\
\textbf{Lami} - Evet hanımın mektubunu ele geçirerek.\\
\textbf{Şemsi} - Aman ne saadet gördün mü? Şimdi işim iş.. Lakin bu kadar güzel bir kadının kocası olup da onu aldatmak..\\
\textbf{Lami} - Haydi oradan Hint horozu. Ulan sanki sen kendi karını aldatmıyor musun!\\
\textbf{Şemsi} - Affedersin. Evvela benim karım süt kardeşim kadar güzel değil. Saniyen bir kdın ne kadar güzel olursa olsun kocaları onu aldatmak ezeli bir adet. Sonra herif burada oturmuş mirasyediler gibi askerlik ediyor. Ben bir buçuk senedir Karaddeniz dalgalaroı arasında bocalanıyor{[}um{]}. Bu kadarcık bir mükafatım olmasın mı?\\
\textbf{Lami} - (Gülerek) Hay tilki hay. Ne ise neme lazım. Yalnız dikkat et yakayı ele vermeyesin. Ben senin yerinde olsam bıyoıklarımı kestirirdim. Ey sahi yanağına ne olmuş senin?\\
\textbf{Şemsi} - Üç ay evvel talim esnasında düştüm yanağım sıyrıldı. Hatta karıma da yazmamıştım.\\
\textbf{Lami} - Olur şey değil be.. Ne acar şeymişsin.\\
\textbf{Şemsi} - Sus!.. Süt eniştem geliyor. Birbirimizi tanımıyoruz ha!..\\

\hypertarget{onunucu-meclis}{%
\section{Onunucu Meclis}\label{onunucu-meclis}}

\begin{verbatim}
 [Evvelkiler-Necdet-Pakize-Zeliha]
\end{verbatim}

\textbf{Necdet} - Lami Bey çantanızı buldunuz mu?\\
\textbf{Lami} - Evet efendim. Hanımefendinin verdiği kataloğa bakıyordum. Yarın sabah gelip yukarı köşkün planını yapacağım.\\
\textbf{Necdet} -Pekala demek şimdi gidiyorsunuz.\\
\textbf{Lami} - Evet efendim.\\
\textbf{Necdet} - Buyrun güle güle (Beraber çıkarlar)\\
\textbf{Şemsi} - (Yalnız) Eyy Şemsi sen budala değilsen Yaşar rolünde yaşarsın. Ben seni tanırırm sen budala bir herif değilsinç Binaenalyh Yaşar rolünde yaşayacaksın..\\
\textbf{Pakize} - (Girerek) Kardeşim odanız hazır.\\
\textbf{Şemsi} - Ah size nasıl teşekkür edeyim bilmem ki kardeşim..\\
\textbf{Pakize} - Burası sizin eviniz Yaşar. İsterseniz gelin sizi odanıza götüreyim. (Çıkarlar)\\
\textbf{Necdet} - (Gelir) Hayır Pakize sevgili süt kardeşinizi Zeliha çıkarsın!\\
\textbf{Pakize} - A.. Ne için?\\
\textbf{Necdet} - Ben biraz sonra çıkacağım. Kal. Sana birşey söyleyeceğim.\\
\textbf{Pakize} - Pekala\ldots{} Zeliha Yaşar'ı götür. Süt kardeşime ne lazımsa verir istirahatını temin edersin\ldots{}\\
\textbf{Zeliha} - Merak etmeyiniz efendim. (Yaşar'a) Buyurun çavuş efendi.\\
\textbf{Şemsi} - (Çıkarken) Lami'nin hakkı varmış. Tam zamanında düşmüşüm. (Zeliha ile çıkarlar)\\

\hypertarget{onbirinci-meclis}{%
\section{Onbirinci Meclis}\label{onbirinci-meclis}}

\begin{verbatim}
 [Pakize-Necdet]
\end{verbatim}

\textbf{Pakize} - Haydi bakalım ne söyleyecekseniz söyleyin..\\
\textbf{Necdet} - Evvela san şunu söyleyeyim ki: Yabancıların yanında bana fena muamele ediyorsun.\\
\textbf{Pakize} - Size daha iyi muamele etmek elimden gelmez. Beni aldatmamalı idiniz.\\
\textbf{Necdet} - Ben seni aldatmadım. Kendimi bahriye uğruna feda ettim.\\
\textbf{Pakize} - Ne?\\
\textbf{Necdet} - Boğulmaktan kurtardığım sefil kadın yazdığı mektupta neler diyordu unuttun mu? Eğer gelmezseniz sizi gazetelerle teşhir ederim. Bahriyeli Necdet Bey bana tasallut diye skandal çıkarırım diyordu. Ben de ne yapayım mesleğimin şerefine siyaneten..\\
\textbf{Pakize} - Bu rezaleti yaptınız. Beni aldattınız değil mi? Aman yarabbi! Utanmasanız bunun için bir de liyakat madalyası isteyeceksiniz.\\
\textbf{Necdet} -Liyakat değil ama. Herhalde tahlisiye madalyasına layıktım.
\textbf{Pakize} - Utanmadan bir de alay ediyorsunuz değil mi? Düşünmüyorsunuz ki burada beni müdafa edecek kimsem olmadığı için siz iki kat kabahatlisiniz. Akrabamdan sağ kalan yegane hamim dayım senelerden beri İstanbul'a gelmiyor.\\
\textbf{Necdet} - Evet yegane haminiz Gazanfer Kaptan. Evlendiğimiz zaman bile lütfen düğünümüzde hazır bulunmaya tenezzül etmediler.\\
\textbf{Pakize} - Düğünümüz olduğu zaman dayım KArdeniz Filosu'nda
mühim bir vazife ifa ediyordu. Zavallı adam beni hâlâ mesud zannediyor. Geçen gün ona yazmayı, bana karşı irtikab ettiğiniz hiyaneti bildirmeyi düşündüm. Fakat babam yerindeki adamı inkisar-ı hayale uğratmak neye yarardı. Sana karşı yapılacak şey, birçok kadınlar gibi, karşıma çıkan ilk erkeğe teslim olmaktı ama\ldots{}\\
\textbf{Necdet} - Pakize çıldırdın mı?\\
\textbf{Pakize} -Allah göstermesin!\\
\textbf{Necdet} - Ha.. Şöyle biraz nefes alayım..\\
\textbf{Pakize} - Bu kadar adileşmeyeceğim en muvafık fırsatı bekleyeceğim.\\
\textbf{Necdet} - Ne\\
\textbf{Pakize} - Evet. Fırsat aldatılmış kadınların silahıdır.\\
\textbf{Necdet} - Pakize deliliğin lüzumu yok.\\
\textbf{Pakize} - Sonra {[}yine{]} evvelki gibi şen şakrak, muti olacağım. Siz de evin içinde erkek sırasına geçecek hakimiyeti alacaksınız. Bu suretle ödeşeceğiz.\\
\textbf{Necdet} - (Hiddet içinde) Buna müsaade edeceğimi zannediyorsan\ldots{}\\
\textbf{Pakize} - İstersen\ldots{} Fakat o gün kadar indimde bir gölde, bir hiçten başka birşey değilsiniz. Benim için yoksunuz\ldots{}\\
\textbf{Necdet} - (Hala mütehavvir {[}hiddetli{]} Ya öyle mi? Peki\ldots{} (Güya müteessir) Yoksa ne yaparım. Zaten ben beş gün için Cumhuriyet Kruvazörü ile Marmara'ya top talimlerine gideceğim.. Bu sabah enrini aldım. Sana sana söylemeyi unututtum. Hem de şimdi çantamı alıp gideceğim..\\
\textbf{Pakize} - (Lakayd) Ya pek ala uğurlar olsun.\\
\textbf{Necdet} - Vedalaşmayacak mıyız? Beni bir defa öpmez misin?\\
\textbf{Pakize} - Sizi öpmek mi? Sonra. İntikamımı aldıktan sonra..\\
\textbf{Necdet} - Peki Pakize.. Fakat pişman olacaksın..\\
\textbf{Pakize} - Acaba?\\
\textbf{Necdet} - Görürüz.. Allahaısmarladık\\
\textbf{Pakize} - Güle güle..\\
\textbf{Necdet} - (Çıkarken) Hay Allah topunuzu birden (Sesi kaybolur. Bir kapunun şiddetle kapandığı işitilir.\\
\textbf{Pakize} - (Yalnız. Zile basarak) Gudur bakalım Necdet Bey. Simdi sıra senin.\\

\hypertarget{on-ikinci-meclis}{%
\section{On ikinci Meclis}\label{on-ikinci-meclis}}

\begin{verbatim}
 [Pakize-Zeliha-Seha]
\end{verbatim}

\textbf{Zeliha} - Zili mi çaldınız küçük hanım?\\
\textbf{Pakize} - Evet Zeliha! Yaşar'a ne lazımsa verdin ya?\\
\textbf{Zeliha} - Tabii efendim. Benden ustura istedi. Beyefendininkini verdim.\\
\textbf{Pakize} - Pek iyi ettin (kapı zili) Git.Kim sorarasa evde yok dersin ha\ldots{}\\
\textbf{Zeliha} - Peki efendim..\\
\textbf{Pakize} - (Yalnız) hay Yaşar hay. (Biraz teessüfle) Ah.. Fakat süt kardeşim.. Hem de bir aşcı\\
\textbf{Zeliha} - (Girerek) Bir hanım geldi. Sizi yok diyeyim mi?\\
\textbf{Pakize} - Kimmiş? Komşulardan mı?\\
\textbf{Zeliha} - Hayır efendim.\\
\textbf{Pakize} - İsmini söyledi mi?\\
\textbf{Zeliha} - Hayır efendim.\\
\textbf{Pakize} - Peki buraya al.~(Gizli) Kimbilir ne için gelmişlerdir.\\
\textbf{Zeliha} - (Dışarıdan) Evet efendim. Aşağıdalar. Buraya buyurun..\\
\textbf{Seha} - Mersi kızım.\\
\textbf{Pakize} -( Seha'yı görüp sevibçle haykırarak) A\ldots{} Seha.. Seha'cığım.\\
\textbf{Seha} - (Sevinçle) Cicim\ldots{} (Kucaklaşırlar) (Zeliha Şemsi'nin tarafından çıkar)\\
\textbf{Pakize} -Aman yarabbi Seha'cığım sen nereden çıktın. Böyle ahiretten mi geliyorsun?\\
\textbf{Seha} - Benim güzel Pakize'ciğim görüyor musun altı senelik ayrılıktan sonra yine kavuştuk..\\
\textbf{Pakize} - Sahi birbirimizi görmeyeli altı sene oldu değil mi?\\
\textbf{Seha} - Ya mektepten çıktığımızdan beri\ldots{} Mektepten çıktıktan sonra hepimiz çil yavrusu gibi dağıldık. İşte yine toplanıyoruz.\\
\textbf{Pakize} -Aman yarabbi ne mesudum. otursana cicim.\\
\textbf{Seha} - Ben iki senedir İstanbul'dayım. Daha önce düşün tamam dört sene Edirne'de kaldım.\\
\textbf{Pakize} - Niçin?\\
\textbf{Seha} - Babmın işlerinden dolayı.\\
\textbf{Pakize} - Ey şimdi nerede oturuyorsun?\\
\textbf{Seha} - Fatih'de\\
\textbf{Pakize} - İki senedir İstanbul'da olasın da bir kere bana gelmeyesin.. Hain seni..\\
\textbf{Seha} - Vallahi kardeşim.. Böyle deme. İki senedir evliyim, kaynana yanında oturuyorum. Girdiğime, çıktığıma karışıyorlar.. Bir fırsatını bulamadım. Ama yemin ederim hatırımdan çıkmıyordun. İşte yine evvel seni arayan ben oldum.\\
\textbf{Pakize} - Sahi öyle. Pekala, adresimi nasıl buldun?\\
\textbf{Seha} - Pek kolay. Senin Ada'da oturacağını biliyordum. babanın ismiyle köşkünüzü sordum, salık verdiler. Arabacı da sizi biliyor. Adamcağız senin evlendiğini ve kocanın da Necdet Bey isminde zengin bir genç olduğunu söyledi.\\
\textbf{Pakize} - Ya\ldots(Parmağındaki yüzüğe bakarak) Nihayet sen de evlendin ha..\\
\textbf{Seha} - Evet iki seneden beri.\\
\textbf{Pakize} - Kocan kim?..\\
\textbf{Seha} - Şemsi isminde bir elektrik mühendisi. fakat şimdi asker..Hem de bahriyede\\
\textbf{Pakize} - Oo.. Benimki de; Necdet de bahriyeyi istemişti.\\
\textbf{Seha} -İstanbul'da mı?\\
\textbf{Pakize} - Evet..\\
\textbf{Seha} - Ah bahtiyarsın Pakize: Şükür et. Kocan yanı başında hem de belki hiç sıkıntısız, tehlikesiz askerlik yapıyor.\\
\textbf{Pakize} - Sıkıntısı, teklikesi yok mu?\\
\textbf{Seha} - Değil mi ya?\\
\textbf{Pakize} - Ne ise. Sen öyle bil. (Kalkarak) Seha'cığımşurup ister misin?\\
\textbf{Seha} - Hay hay..\\
\textbf{Pakize} - (Şurubu hazırlarken) Evvela senin kocandan bahsedelim. Bir sevgi neticesinde mi evlendiniz..\\
\textbf{Seha} - Evet bidayette öyle idi.. Fakat sonra askere gidince bu sevgi bir aksülamele uğradı. Gözden ırak olan gönülden ırak olurmuş..\\
\textbf{Pakize} - A\ldots{} Niye Seha'cığım..\\
\textbf{Seha} - İlk zamanlar her hafta muntazamanmektup yazardı. Fakat sonraları.. İki haftada bir. Daha sonraları ayda bir yazmaya başladı.. tasavvur et Pakize'ciğim: gemiye gittiğinden beri bir defa olsun izin alıp gelmedi..\\
\textbf{Pakize} - Oo.. Vah kardeşim..\\
\textbf{Seha} - Bir hafta evvel aldığım son mektubunda da suvarisiyle arası açık olduğu için istediği izni reddetiklerini yazıyordu. Düşün, üzüntüden ne yapacağımı bilmiyorum. Nihayet kendi kendime Bahriye Vekaleti'ne müracaata karar verdim..\\
\textbf{Pakize} - Pek güzel bir fikir..\\
\textbf{Seha} - Fakat benim vekalette tanıdığım kimse yok. halbuki hatırımda kaldığına nazaran senin bi dayın olacak.. gazanfer Kaptan, pek nüfuzlu bir kaptandır derdin. Acaba onun vasıtasıyla..\\
\textbf{Pakize} - Ah kardeşim bu şimdi benim de hatırıma geldi ama.. Dayım burada olsaydı ne iyi olurdu. O şimdi kaymakam oldu fakat kimbilir nerededir? Ankara'da mı, İzmir'de mi, Ünye'de mi? Nerededir? O bir yerde durur mu ki.\\
\textbf{Seha} - Ya vah vah! Halbuki bütün ümidim bunda idi.. (Gördün mü? Bir kere (Düşünerek) Ah keşke ona açıverseydim.\\
\textbf{Pakize} - Kime?\\
\textbf{Seha} - Buraya gelirken vapurda yanı başımda büyük rütbeli ihtiyar bir kaptan vardı.. Ona..\\
\textbf{Pakize} - (Gülerek) İhtiyar bir kaptan mı? Ne söylüyorsun Seha..\\
\textbf{Seha} - Güzel dinç bir ihtiyar. Vapurda bilhassa yanı başımda oturdu. Aman Pakize'ciğim, daha Sarayburnu'nu bile dönmemiştik, bir de baktım başını çevirmiş tuhaf tuhaf bana bakıyor. Kıpkırmızı oldum. Önüme bakmaya başladım. Bu sefer de adamcağızın iç çekmelerini işitiyordum. Adamcağız garç garç bir iki defa yutkunduktan sonra, bana dönüp ``sıkılıyorsunuz galiba hanımefendi. Buyurun şu gazeteyi. Eğlenirsiniz.'' dedi. Resimli gazete uzattı.. ~
\textbf{Pakize} - Kabul ettin mi?\\
\textbf{Seha} - A.. Hiç kabul eder miyim kardeşim? Teşekkürle reddettim. Fakat şimdi pişman oluyorum. Vah vah ..\\
\textbf{Pakize} - Dur bakalım üzülme kardeşim. Seha'cığım belki bir çaresi bulunur. Dayım tanıdığı bir başkasını bulalım. Şemsi Bey'e bir izin temin edebiliriz.\\
\textbf{Seha} - Ah kardeşim bunu yapabilirsen her ikimiz de sana ne kadar minnettar oluruz.\\
\textbf{Pakize} - Gitmek mi ?Bunca senelik {[}dizgi hatası{]} Aman Seha'cığım vazifem; sen akşam buradasın ya..\\
\textbf{Seha} - A.. İmkanı yok gitmeliyim..\\
\textbf{Pakize} - Gitmek mi? Bunca senelik hasretten sonra kırk yılda bir beraberce bir gece geçireceğiz. Reddedersen vallahi darılırım..\\
\textbf{Seha} - Hayır ama Pakize'ciğim, kaynanam. Bilmem ki..\\
\textbf{Pakize} - Aman sen de Seha'cığım. Koca kadınsın. Kırk yılda bir bir arkadaşına gece yatısına gitmeye hakkın yok mu?\\
\textbf{Seha} - Hayır onun için değil. Belki merak ederler diye.. Sizde telefon var mı?
\textbf{Pakize} - Hayır daha almadık.\\
\textbf{Seha} - Şu halde bana müsaade edecekin Pakize'ciğim. Ben Heybeli'ye kadar gidip geleceğim. İki üç saate kadar gelirim.\\
\textbf{Pakize} - A.. Niçin?\\
\textbf{Seha} - Görümcem orada oturuyor. Kaynanamın bir işi için onlara uğrayacaktım. Madem ki akşam inmeyeceğim, oradan da Fatih'e telefon ederim. Eczahaneden gelmeyeceğimi söylesinler.\\
\textbf{Pakize} - Sen bilirsin Seha'cığım.. (Latife ile) Kaçma ha..\\
\textbf{Seha} - A.. Kabil mi Pakize'ciğim?.. (Öperek) Şimdilik Allahaısmarladık.\\
\textbf{Pakize} - (Seha ile ayrılacağı sırada) Seha'cığımDüşün bakalım mektepten şahadetnamelerimiz alacağımız bahçede ne konuşmuştuk!\\
\textbf{Seha} - (Düşünerek) Vallahi\ldots{}\\
Pakize Hatırlayamadın değil mi? Ondan sık sık bahsediyorduk\\
\textbf{Seha} - (Gülerek) Ondan sık sık bahsettiğimiz için..\\
\textbf{Pakize} - (Gülüşme) Hayır unuttun mu? O gün bana ne dedin? ``Evlendiğim zaman kocam beni aldatırsa ben de ilk fırsatta omu aldatacağım.'' demedin mi?\\
\textbf{Seha} - Ha evet kısasa kısas değil mi?\\
\textbf{Pakize} -Yine o fikirde misin?\\
\textbf{Seha} - Tabii. Hem bunu yalnız bir intikam usulü değil; adeta bir vazife telakki ediyorum. Eğwer bütün kadınlar bu usulü tatbik edebilselerdi, bugün karısını aldatan kocalar azalırdı.. bak mesela ben Şemsi'yi çıldırasıya seviyorum. Fakat en ufak bir hıyanetini yakalayacak olursam, ona karşı kullanacağım silah budur.\\
\textbf{Pakize} - Teşekkür ederim Seha'cığım. Bu sözlerin son tereddütlerimi izale etti.\\
\textbf{Seha} - Ne.. Sakın kocan seni..\\
\textbf{Pakize} - Aldatıyor kardeşim.. Evet altı senelik karı kocalıktan sonra.. Bir mektubunu yakaladım ki..\\
\textbf{Seha} - Vah Pakize'ciğim desene ki bedbahtsın\ldots{}\\
\textbf{Pakize} - Evet fakat intikamımı alacağım. Seha senin nasihatını tutacağım..\\
\textbf{Seha} - A.. Yok sana nasihat diye söylemedim. Aman Paki{[}ze{]} çok düşün ha\ldots{}\\
\textbf{Pakize} - Sen olsan bu kadar düşünür müydün?\\
\textbf{Seha} - Ben başka mesele.. Şimdi senin başında..\\
\textbf{Pakize} - Ben düşüneceğim kadar düşündüm kardeşim..\\
\textbf{Seha} - Ne ise düşün de yine konuşuruz Pakize'ciğim. Şimdilik Allahaısmarladık.\\
\textbf{Pakize} - Seni geçireyim.. Bilsen ne kadar düşündüm. Bundan başka birşey düşünmüyorum. Şimdi yalnız fırsat bekliyorum. (Son cümle çıkarken kaybolur.)\\

\hypertarget{on-ucuncu-meclis}{%
\section{On Üçüncü Meclis}\label{on-ucuncu-meclis}}

\begin{verbatim}
 [Pakize-Şemsi]
\end{verbatim}

(Bir kaç saniye sonra Şemsi gözükür. Arkasında Zeliha söz söylemektedir.) (Şemsi tamamen traş olmuş, sivil giyinmiştir. Elinde serpuş vardır.)

\textbf{Şemsi} - Sizin beyin esvabı sanki bana göre yapılmış o kadar iyi geldi ki.. Al kızım.. Şunu da pencereye as biraz hava alsın.. (Serpuşu astıktan sonra kendisini hayran hayran seyreden Zeliha'ya bakıp gülerek) Bu kılıkla da hiç gemi aşcılarına benzemiyorum değil mi, Zeliha?\\
\textbf{Zeliha} - Ooo.. Hayır efendim hiç..\\
\textbf{Pakize} - (İçeriden seslenerek) Zeliha! Zeliha!\\
\textbf{Zeliha} - Aaa.. Küçük hanım (Süratle koşarak) Efendim..\\
\textbf{Pakize} -Neredin sen? Yaşar indi mi?\\
\textbf{Zeliha} - Evet efendim burada (Tamam bu sözde Pakize sahneye girer) (Şemi'yi yeni kılıkla görüp şaşırarak)\\
\textbf{Pakize} -A.. Siz misiniz? Vallahi tanıyamayacaktım,
Yaşar.. Siz ha..\\
\textbf{Şemsi} - Evet ben Yaşar. Biraz evvelki nefer kıyafetinden çıkınca beni tanımakta güçlük çektiniz değil mi?\\
\textbf{Pakize} - Hakikaten öyle. Hele bıyıklarınızı kestikten sonra..\\
\textbf{Şemsi} - Bu yeni elbise ile bütün bütün değiştim..İğrenilmeyecek konuşulacak bir adam oldum değil mi?\\
\textbf{Pakize} - Hayır onu demek istemiyordum..\\
\textbf{Şemsi} - Ne deseniz haklısınız. Bir gemi aşcısının giyinip kuşandıktan sonra bu derece değişmesine ben de olsam şaşardım.\\
\textbf{Pakize} - Sahi soracaktım.. Askere girmezden evvel asıl mesleğiniz de aşcılık mıydı?\\
\textbf{Şemsi} - Yoo..\\
\textbf{Pakize} - Pekiyi nasıl oldu da?..\\
\textbf{Şemsi} - (Bir pot kırmamak için) Ha.. Yani bilirsiniz ya.. Valideniz merhûm beni.. Şey.. Hastahanesinin aşcıbaşısı..\\
\textbf{Pakize} - Hastahane mi ya?.. Mektep.. Dar'ül-hayır..\\
\textbf{Şemsi} - Öyle ya.. Hastahane demişim.. Dar'ül-hayıra aşcı Ömer Ağa'nın yanına evlatllık olarak vermiş. Bizim babalığın yanında mutfak işlerini biraz öğrenmiştim. Sonra gemide en kolay iş olarak bunu bulduğum için mutfağa inmeyi kendşm istedim. Askerliğim bitinceye kadar bu yağlı meslekte kalmaya ahd ettim. Yoksa mutfak bemin midemi doyursa bile dimağımı doyuramayacağı için..\\
\textbf{Pakize} - (Birden sözünü keserek resmi bir şive ile) Rica ederim Yaşar Bey otursanıza.. Rahatsız olmayınız..\\
\textbf{Şemsi} - Teşekkür ederim efendim. Fakat benimle niçin böyle birden bire resmi oldunuz, kardeşim?\\
\textbf{Pakize} - Çünkü.. Nasıl söyleyeyim bilmem.. Demin karşımda sadece bir nefer, bir asker görmüştüm.. Bir askerin karşısında kendimi daha ziyade emniyette hissediyordum. Halbuki şimdi..\\
\textbf{Şemsi} - Şimdi sivil olarak gördüğünüz halde kendinizi daha mı az emniyette hissediyorsunuz?. Bana gelince asker, sivil sizin karşınızda aynı şeyi hissediyorum.. (İskemlesini yanaştırarak) asker veya sivil Trabzon'dan buraya kadar yalnız sizi görmek için gelen Yaşar'dan başka birşey değilim.. Yaşar, çocukluk arkadaşınız Yaşar.. İki aydan beri kendisine en hararetli, en samimi mektupları yazdığınız, resminizi gönderdiğiniz Yaşar.. O andan itibaren sizin için hayatını fedaya hazır olan..\\
\textbf{Pakize} - (Birden silkinerek) Rica ederim..\\
\textbf{Şemsi} - (Cümlesini süratle itmam ederek {[}tamamlayarak{]}) hazır olan süt kardeşiniz Yaşar!\\
\textbf{Pakize} - (Önüne bakarak) yaşar Bey bu tarzda konuşmayınız..\\
\textbf{Şemsi} - Bey mi? Oldu mu ya? sadece Yaşar.. Süt ninenizin oğlu Yaşar.. Son mektubunuzda ne diyordunuz? Senelerden sonra karşınıza çıkan size hayatınızın en mesut safhası olan çocukluğunuzu hatırlatan, küçüklük arkadaşınız, yegane dostunuz değil miyim?\\
\textbf{Pakize} - hay hay\ldots{} samimi dostum, süt kardeşimsiniz..(Bunu söylerken gayrı iradi olarak elini uzatmıştır.)\\
\textbf{Şemsi} - (Pakize'nin elini alıp kalbine götürerek) Ben de ben de sizin en samimi dostunuzum.\\
\textbf{Pakize} -(Değişik bir sesle) Yaşar elimi bırakınız benim..\\
\textbf{Şemsi} - (yavaşca bırakırken içine bakarak) Peki affedersiniz.. Ooo! Aman Yarabbi! eliniz..\\
\textbf{Pakize} - Ne var bir şey mi gördünüz?\\
\textbf{Şemsi} - Evet bir şey gördüm. Keşke görmeseydim..\\
\textbf{Pakize} - Merak ettiriyorsunuz..\\
\textbf{Şemsi} - hayır.. Merak etmeyin.. Çok garip. Keşke görmeseydim. (Elini bırakır)\\
\textbf{Pakize} - (Elini uzatarak) ne var? Kuzum ne gördünüz..\\
\textbf{Şemsi} -Ne mi gördüm. Elinizde öyle şeyler okudum ki..\\
\textbf{Pakize} - Nasıl? Okundunuz mu? El falından anlıyor musunuz?\\
\textbf{Şemsi} - (Gülümseyerek) Eh bir parça.. Eğer fala inanırsanız size oradan muhim şeyler haber verebilirim.\\
\textbf{Pakize} - (Gülerek) Acaba!.. Haydi bakalım..\\
\textbf{Şemsi} - Şüphe mi ediyorsunuz?. Peki.. Hayır sağ değil sol elinizi verin.. (Elini alır içine bakarak) İşte şu hat saadeti gösterir. Bakınız ne kadar kısa. Şuradan kesilmiş. Saadetiniz devam etmemiş. Siz artık mesut değilsiniz. Çünkü zevciniz sizi aldatıyor..\\
\textbf{Pakize} - (tam hayretle) Nasıl bildiniz?..\\
\textbf{Şemsi} - Eliniz söylüyor. Doğru değil mi?\\
\textbf{Pakize} - Doğruysa bile size ne?\\
\textbf{Şemsi} - Durun, durun, işte işte deniz üstünde bir kaza! Yayılmış {[}Dizgi hatası-Bayılmış{]} güzel bir kadın kocanızın kolları arasında ah.. eğer bir pertavsız {[}büyüteç{]} olsa size kocanızın ismini de, kadının gözlerinin rengini de söyleyebilirdim.\\
\textbf{Pakize} - Aman Yarabbi! Siz falcı mısınız acaba?\\
\textbf{Şemsi} -Bazan falcı da olurum. Durun bakayım. Ooo.. Kokulu bir mektup (Elini koklar gibi ağızına sürerek) evet bu eliniz titreyerek mektubun bir parçasını yırtıyor.. Herşeyi biliyorsunuz. Kocanızı affetmediniz. hakkınız da ver. Çünkü sizi aldatmak..\\
\textbf{Pakize} - (elini çekerek) Yeter Yaşar..\\
\textbf{Şemsi} - Affedilmez bir cinayet. Sonra kıymet bilmez koca o adi kadının etkilerine kapılmış vaktiyle aynen size söylediği tatlı cümleleri, yeminleri sayıp dökerken; uzaklarda, denizaşırı ufuklarda zavallı bir adam gecelerini sizi düşünmekle geçiriyordu.\\
\textbf{Pakize} - Yaşar neler söylüyor sunuz?\\
\textbf{Şemsi} - Evet zavallı bir adam.. El yazılarını kalbi çarparak tekrar tekrar okur, resminiz öpücüklere boğardı.\\
\textbf{Pakize} - Yaşar Bey..\\
\textbf{Şemsi} - Pakize Hanım.\\
\textbf{Pakize} - Yaşar Bey süt kardeşimsiniz.\\
\textbf{Şemsi} - Evet süt kardeşim. Süt kardeşciğim sizi seviyorum.\\
\textbf{Pakize} - (Mübhem {[}belli belirsiz{]} tekrarlar) Ben seviyor.. (Şemsi'ye) Süt kardeşinizi bu tarzda nasıl seviyorsunuz? Günahtan korkmuyor musunuz?\\
\textbf{Şemsi} - Kimbilir beni size bu kadar yaklaştıran şey de annemin verdiği sütlerdir. Sonra ne yalan söyleyeyim ben sizin bütün bütün süt kardeşiniz değilim!\\
\textbf{Pakize} - (Kalkarak) Nasıl?\\
\textbf{Şemsi} - (Devamla) Ben sizin üvey süt kardeşinizim!..\\
\textbf{Pakize} - Ne demek?\\
\textbf{Şemsi} - Evet üvey! Eğer ben de sizin annenizden süt emseydim o vakit öz süt kardeş olurdum. Hem sonra mürur-u zaman var. O bir çocukluk hatırası idi. Görüyorsunuz ya? Sizi seviyorum. Yazdığınız her mektup bu sevgiyi kıvılcımladı. Ah! Sizin bu derece bedbaht olduğunuzu bilseydim bütün bunları size evvelden yazardım.\\
\textbf{Pakize} - Keşke yazsaydınız. Size cevap vermezdim bu günahı da işlemezdik..\\
\textbf{Şemsi} - PakizeHanım kalbiniz emin olsun: Bu günah değil.. Sevgiye susamış ruhların arasına ne süt ne yoğurt girer.\\
\textbf{Pakize} - Aman Yarabbi!..\\
\textbf{Şemsi} - emin olun yarın öbür gün gemiye gittiğim zaman beni orada teselli edecek şey yalnız bu sevgidir. Bırakınız da hayatımın en güzel hatırasını alayım..\\
\textbf{Pakize} - (Dalgın fakat mütehayyic {[}heyecanlanmış{]}) Aman yarabbi! Süt kardeşimle..\\
\textbf{Şemsi} - Hayır.. Süt kardeşinizle değil, gönül arkadaşınızım. Bir an için unutun bu süt kardeşliğini\\
\textbf{Pakize} - (Aklına kocası gelerek) İşte fırsat. (Şemsi'ye) Peki süt kardeşim.\\
\textbf{Şemsi} - Ah süt kardeşim sizi seviyorum. (Zaten kolları arasında tuttuğu Pakize'yi alır dudaklarından öper bu puse mütekabil bir inilti içinde devam eder.)\\

\hypertarget{on-dorduncu-meclis}{%
\section{On Dördüncü Meclis}\label{on-dorduncu-meclis}}

\begin{verbatim}
 [evvelkler - Gazanfer Kaptan]
\end{verbatim}

\textbf{Kaptan Kaymakam} - (Dipten gözükür. Onları bu halde görünce gür bir sesle bağırır) Lâşkâ {[}halatı boşa al{]}\ldots{} (Şemsi Pakize derhal ayrılır.) Ha sizi tehlikeli bir vaziyette yakaladım çocuklar.\\
\textbf{Pakize} - (Hayretle bağırır) A.. Dayım!\\
\textbf{Şemsi} - Vay bir kaymakam {[}yarbay{]}. (Sivil olduğunu unutarak askeri bir vaziyet alır)\\
\textbf{Ka{[}ymakam{]}} - Şu anda aklınızdan geçmiyordum. Ya, sürprizim tam zamanında oldu. Ne donakaldınız orada? (Kollarını açarak) Kızım Pakize'm gel bakayım. Dayını böyle mi karşılayacaksın bakayım\ldots{}\\
\textbf{Pakize} - (Kollarının arasına koşarak) Dayıcığım..\\
\textbf{Ka{[}ymakam{]}} - Evladım..\\
\textbf{Şemsi} - Dayısı mı? Vay canına\ldots{}\\
\textbf{Ka{[}ymakam{]}} - Öp bakayım dayını.. Hadi hadi kocan müsaade eder. (Şemsi'ye) öyle değil mi sevgili damadım?\\
\textbf{Şemsi} - (Rica ederek) Kocası mı? Eyvah! Kocası mı olduk?\\
\textbf{Pakize} - (Kendi) Ne?! Kocam zannediyor\ldots{}\\
\textbf{Ka{[}ymakam{]}} - (Pakize'nin değişik mahcup halini görerek) Ey ne var ya?\\
\textbf{Pakize} - A.. Bir şeyim yok dayıcığım. Sizi birdenbire görmek beni şaşırttı da.\\
\textbf{Ka{[}ymakam{]}} - Adam sen de. İşte kavuştuk. Öp bakayım da bir daha dayıcığını. (Pakize onu öper) Hah\ldots{} (Onu bir daha öperek) Oh biricik kızım! Kocan kadar ateşli öpemiyorsun değil mi? Sizi gidi külhaniler sizi. Demin şurada sımsıkı maçuna olmuştunuz. Lakin sevgili damadım karınıza bir sarılışınız var. Aranızdan su sızmayacağına bahse girerdim. Tam sapına kadar beraber gelmişsiniz. (Elini uzatarak) Sizinle nihayet tanıştığıma emin olun pek memnunum. Bilseniz düğününüzde bulunamadığıma ne kadar müteessir oldum.\\
\textbf{Şemsi} - (Sürat-i intikal ile {[}durumu hemen kavrayarak{]}) Ah.. Ya ben? Bütün gün süt\ldots{} Süt dökmüş kedi gibi sessiz duran Pakize'ye ``Hanım dayınız gelmeyecek mi?'' diyerek sizi dilimden düşürmedim. Nihayet ikimiz de üzüldük\ldots{}\\
\textbf{Ka{[}ymakam{]}} - Yavrularım\ldots{} Geldim.. Mesud olduğunuzu gözümle gördüm. ama heyhât bu sefer yanınızda uzun müddet kalamayacağım. Yarın sabah Ankara'ya gideceğim. Daha İstanbul'a gidip kendime öte beri düzeceğim..\\
\textbf{Pakize} - (Geniş nefes alarak) Ya\ldots{} Böyle hemencecik mi dayıcığım?\\
\textbf{Ka{[}ymakam{]}} - Evet yavrum\ldots{} vekalet acele çağırmış. Verilecek pek mühim raporlarım var. Eh, Allah bilir ama yine bana uzun bir sefer görünüyor.. Galiba vapur teslimine gideceğiz.\\
\textbf{Pakize} - Ya vah vah..\\
\textbf{Şemsi} - Ya vah vah..\\
\textbf{Ka{[}ymakam{]}} - Bu sefer Karaburun'dan gelirken kendi kendime İstanbul'a varınca mutlaka Ada'ya gidip çocukları öpeceğim diye ahd ettim. Burada bir öğle yemeğinizi yer, hemen kaçarım.\\
\textbf{Şemsi} - Ya.. Ne güzel fikir değil mi Pakize'ciğim?\\
\textbf{Pakize} - A tabii tabii.\\
\textbf{Ka{[}ymakam{]}} - Fakat ben boyuna kendimden bahsettim. Biraz da size daie konuşalım canım. (Oturarak) Bir defa ikinizin mesut olup olmadığınızı sormak fazla. Çünkü sizin gibi altı senelik evliler dört günlük nişanlılar gibi sarmaş dolaş olunca pek belli ki..\\
\textbf{Pakize} - Aman dayıcığım..\\
\textbf{Ka{[}ymakam{]}} - Haydi canım! Kızaracak ne var deli kız.? Bilakis hoşuna gitsin. Harpten sonra böyle kocalar ender oldu.. (Şemsi'ye) Zaten göreyim seni delikanlı, istersen onu mesut etme; sonra benim elimden zor kurtulursun..\\
\textbf{Şemsi} - (Ne söyleyeceğini şaşırarak) Teşekkür ederim.. Şey tabii beyefendi.\\
\textbf{Ka{[}ymakam{]}} - Kızım her türlü saadete layıktır. O ismet ve namusun timsalidir.\\
\textbf{Şemsi} - Hakkınız var beyefendi.\\
\textbf{Ka{[}ymakam{]}} - Beyefendi, beyefendi! Gemide miyiz be! Dayı diyemiyor musunuz? Dayı\ldots{}\\
\textbf{Şemsi} - (Gülerek) Hakkınız var dayı bey, hakkınız..\\
\textbf{Ka{[}ymakam{]}} - Hah şöyle eyy. Az kaldı unutuyordum. Sizin askerlik daha bitmedi mi? Yine liman dairesinde misiniz?\\
\textbf{Şemsi} - (Şaşırarak) efendim?\\
\textbf{Pakize} - (Atılarak) Evet dayıcığım. tabii yine orada. Size de yazmıştım ya\ldots{}\\
\textbf{Ka{[}ymakam{]}} - Pek güzel pek güzel! Fakat tuhaf şey. pakize'nin bana gönderdiği bahriye elbiseli resimle sizin aranızda iyice fark var.\\
\textbf{Şemsi} - Evet o vakitlr şimdiki arasında fark var.. Ben zaten çabuk şişmanlar, bazan da derhal zayıflarım\ldots{} Gününe göre..\\
\textbf{Ka{[}ymakam{]}} - (Gülerek) Gününe göre mi? Hah hah pek tuhafsınız\ldots{}\\
\textbf{Pakize} - (Kalkıp mukalemeyi {[}konuşmayı-sohbeti{]} değiştirerek) Dayıcığım şurup içer misiniz?\\
\textbf{Ka{[}ymakam{]}} - (Şemsi'ye) Şurup değil ama şarap varsa\ldots{}\\
\textbf{Şemsi} - Şarap var değil mi, Pakize ?..\\
\textbf{Pakize} - Var ya.. Vereyim dayıcığım..\\
\textbf{Ka{[}ymakam{]}} - Eh pekala. Ben de sıhhatinize içkinizi içerim..\\

\hypertarget{on-besinci-meclis}{%
\section{On Beşinci Meclis}\label{on-besinci-meclis}}

\begin{verbatim}
 [Evvelkiler-Necdet-Zeliha]
\end{verbatim}

\textbf{Necdet} - (Arkasından gelen Zeliha{[}'ya{]} hiddetle{]} Sna gitmeyeceğim diyorum. Kızım al şunu (Şapkasını verir) Bana biraz sıcak su hazırla.. Limana telefon ettim. yarın sabah ilk vapurla inerim.\\
\textbf{Ka{[}ymakam{]}} - (Şarabı içip bardağı iade ile) Oh fena değil bu şarap. Nereden böyle..\\
\textbf{Necdet} - (Kaymakamı görerek) Vay bir kaymakam. (Gayrıı iradi olarak pencerenin yanında asılı duran serpuşu giyer. Hazır ol vaziyeti alır)\\
\textbf{Şemsi} - İşte şapa oturduk!\\
\textbf{Ka{[}ymakam{]}} - Bu neci?\\
\textbf{Pakize} - Bu şeyci dayıcığım\ldots{} Süt ninemin oğlu Yaşar\ldots{}\\
\textbf{Necdet} - Bu ne diyor?\\
\textbf{Ka{[}ymakam{]}} -sizin yaninizdami\\
\textbf{Şemsi} - Evet.\\
\textbf{Pakize} - (Kocasına gider süratle) Bana uy yoksa bittik.\\
\textbf{Necdet} - Ne?..\\
\textbf{Pakize} - (Dayısının yanına gelerek) Dayıcığım Yaşar izinle gelmiş. Ay Taş gemisinde aşcı.\\
\textbf{Necdet} - Şimdi de aşcı olduk..\\
\textbf{Ka{[}ymakam{]}} - (Amirane) Yaklaş bakalım aşcı başı\ldots{}\\
\textbf{Şemsi} - (Kendi) Benim başlığı almış..\\
\textbf{Ka{[}ymakam{]}} - (Okuyarak) Aytaş'tansın ha.. Ver elini bakayım.\\
\textbf{Necdet} - Elim pek piste efendim.\\
\textbf{Ka{[}ymakam{]}} - Zarar yok uzat.\\
\textbf{Necdet} - (Bileğini indirerek) Buyurun efendim\\
\textbf{Ka{[}ymakam{]}} - Eyy? Doğru Trabzon'dan mı geliyorsun?\\
\textbf{Necdet} - Galiba.\\
\textbf{Ka{[}ymakam{]}} - Ne demek galiba?\\
\textbf{Pakize} - (Necdet'e işaretle) Söylesene Yaşar. Bu sabah gelmedin mi? (Dayısına) Kusuruna bakma dayıcığım. Sizi görünce şaşaladı.\\
\textbf{Necdet} - (Şiddetle) Evet evet Trabzon'dan geliyorum efendim. Doğruca Trabzon'dan.\\
\textbf{Ka{[}ymakam{]}} - Belli leş gibi kokuyorsun.\\
\textbf{Necdet} -Leş gibi mi?\\
\textbf{Ka{[}ymakam{]}} - (Necdet'i iterek) Şu hale bak! Bu da gemici ha?.. Ayakta durmaya mecali yok. Böylelerini de gemide ne diye kullanırlar bilmem ki. (Şemsi'yi göstererek) Halbuki bak şuna: Topuz gibi delikanlı!\\
\textbf{Necdet} - Şimdi bayılacağım.\\
\textbf{Pakize} - Bir bardak daha ister misiniz, dayıcığım?\\
\textbf{Ka{[}ymakam{]}} - Hay hay kızım (Necdet'e) Söyle bakalım geminiz karakola çıkıyor mu?\\
\textbf{Necdet} - karakola hayır efendim.\\
\textbf{Pakize} - (Atılarak) Dayıcığım yemekten evvel biraz istirahat istemez misiniz?\\
\textbf{Şemsi} - Yorgunsanız benim odama buyurun.\\
\textbf{Necdet} - Odana mı?\\
\textbf{Ka{[}ymakam{]}} - Dinlenmek değil ama bir parça elimi yüzümü yıkaraım\\
\textbf{Pakize} - (Dışarı bağırarak) Zeliha! Zeliha!\\
\textbf{Şemsi} - Zeliha! Zeliha!\\
\textbf{Necdet} - Herif evi gibi emrediyor.\\
\textbf{Zeliha} - Efendim.\\
\textbf{Pakize} - Zeliha dayımı tuvalet odasına götür.\\
\textbf{Zeliha} - Peki efendim.\\
\textbf{Ka{[}ymakam{]}} - Haydi bakalım kızım düş önüme (Çıkar)\\

\hypertarget{on-altinci-meclis}{%
\section{On Altıncı Meclis}\label{on-altinci-meclis}}

\begin{verbatim}
 [Kaymakamdan maada- evvelkiler]
 
\end{verbatim}

\textbf{Şemsi} - (Onlar çıkarken) Ben galiba doğrudan doğruya eve gitmediğime pişman olacağım\\
\textbf{Necdet} - (Başındaki serpuşu atarak) Of.. Gitti.. Şimdi siz bu oyunu bana izah eder misiniz?\\
\textbf{Pakize} - (Mülayimetle) Hay hay azizim.\\
\textbf{Necdet} - ey bu kaymakam dayın değil mi?\\
\textbf{Pakize} - Öyle ya .. Birdenbire geldi Yaşar'ı da sen zannetti.\\
\textbf{Necdet} - Ben yani kocan?\\
\textbf{Pakize} - Evet..\\
\textbf{Necdet} - Ya.. Ben ben de..\\
\textbf{Şemsi} - Sizi de beni yani süt kardeşini zannetti.\\
\textbf{Necdet} - Niçin doğrusunu söylemediniz?\\
\textbf{Pakize} - Niçin mi? Söyleyemezdim çünkü tam içeriye girdiği sıra süt kardeşimle öpüşüyordum.\\
\textbf{Necdet} - (Hiddetle) Nasıl?..\\
\textbf{Şemsi} - (Hiddetle) Necdet Bey burada bir kabahatli var..\\
\textbf{Pakize} - (Necdet'i göstererek) O da budur.\\
\textbf{Necdet} - Ben miyim?\\
\textbf{Şemsi} -Sizsiniz..\\
\textbf{Pakize} - Sensin ya.. Bütün bunlar senin başının altından çıktı. O kabahati işlemeye idin bu karışıklıklar olmazdı.. Size vaad etmemiş miydim? İlk fırsatta intikamımı alacağım diye.\\
\textbf{Necdet} - Eh demek bu fırsat..\\
\textbf{Pakize} - (Şemsi'yi gösterek) Fırasat da budur.\\
\textbf{Necdet} - Siz misiniz?\\
\textbf{Şemsi} -(Tefehhürle {[}övünerek{]}) Benim.\\
\textbf{Necdet} - Desenize tam zamanında gelmişim.. (Hiddetle) evvela siz aşcıbaşı Yaşar Efendi\ldots{} İlk işiniz buradan arabanızı çekmek olsun.. Hem derhal..\\
\textbf{Pakize} - hayır. Ben buna müsaade etmem.\\
\textbf{Şemsi} - Süt kardeşim müsaade etmiyor.\\
\textbf{Necdet} - (Kendini kaybedercesine) Pakize diyorum!\\
\textbf{Pakize} - Dayım onu sizi zannediyor. Mümkün mü? Zaten o bu akşam gidiyorç Üçümüzün de selameti için Yaşar'ın akşama kadar benim kocam olarak kalması lazımdır.\\
\textbf{Necdet} - Ben de kapı mandalı olacağım öyle mi? Katiyen! Anlıyor musun? Katiyen! Dikkat et sonra ona herşeyi söylerim.\\
{[}\textbf{Pakize} -{]} Bana hıyanet ettiğini duyarsa seni seni affetmez. Ben de ona karşı aldatılmış bir zevce vaziyetinde kalamam. Bunu bil..\\
\textbf{Şemsi} - (Necdet'e) NMüsaade edin de en bi-taraf hükmü ben vereyim.\\
\textbf{Necdet} - (Şiddetle) Siz susun bakalım bulaşıkcı efendi\ldots{}\\
\textbf{Şemsi} - Peki bana göre hava hoş. Siz düşünün.\\
\textbf{Necdet} - (Karar vererek) Ne olursa olsun. Herşeyi kabul ederim, bu vaziyeti asla\ldots{}\\
\textbf{Pakize} - Pekala. Sonra başına gelecekleri de düşün.\\
\textbf{Necdet} - Bundan başka daha ne gelebilir.\\
\textbf{Pakize} - İşte dayım\ldots{}\\

\hypertarget{on-yedinci-meclis}{%
\section{On Yedinci Meclis}\label{on-yedinci-meclis}}

\begin{verbatim}
 [Evvelkiler- Kaymakam- Zeliha]
\end{verbatim}

\textbf{Necdet} - Kaymakam Bey size birşey söylemek isterim\ldots{}\\
\textbf{Pakize} - (Sağda) Ben de size birşey söylemek isterim dayıcığım.\\
\textbf{Necdet} - Affedersiniz evvela ben söyleyeceğim..\\
\textbf{Ka{[}ymakam{]}} - Eyy dur bakalım sen aşcıbaşı! Sana müsaade ettiğim zaman söylersin! Haydi alabanda marş!\\
\textbf{Necdet} - Fakat müsaade ediniz Kaymakam Bey..\\
\textbf{Ka{[}ymakam{]}} - Ne? Bana cevap mı veriyorsun. Sen hiç talim ve terbiye görmedin galiba! Burada evinde olmadığını unutuyorsun. Haydi bakalım. Aö başı aşcıbaşı! Şöyle gelin damadım yaklaşsın da Pakize..\\
\textbf{Şemsi} - Peki dayı bey\ldots{}\\
\textbf{Ka{[}ymakam{]}} - Sevgili yavrularım\ldots{} Beni dinleyin. Ben tekaüd edildiğim günden itibaren Karaburun'daki çiftliğe çekilip oturacağım. Pek muhtemeldir ki bu akşam sizi öpüp ayrıldıktan sonra bir daha birbirimiz görmeyelim.\\
\textbf{Pakize} - Ne söylüyorsunuz dayıcığım?\\
\textbf{Ka{[}ymakam{]}} - Kesme yavrum\ldots{} Söyleyim de.. Dört sene evvel küçük çiftliğimde bulunan simli kurşun madenine bu sene şirket talip oluyor. Beş yüz bin lira veriyor.\\
\textbf{Necdet} - (Yüksek sessle) Yarım milyon\ldots{}\\
Kaymakam - Susun arkadan (Pakize'ye) Çiftliğimden maada bu servetimi ikiye taksim etmiş, bir hissesini sana, bir hissesini İzmir'de oturan öteki teyzen Hatice'ye terk etmeye karar vermiştim. (Ayağa kalkarak) İstanbul'a gelmeden evvel onlara da uğrayıp bir akşam kaldım. Ah onların hayatı ile sizinki arasında müthiş bir fark var. Burada kucaklaşma, öpüşme, sevişme, saadet; orada her dakika dırıltı, gürültü eksik değil\ldots{} Kocası Hatice'yi aldatıyor. Hatice de kocasını. İşte bunun için vasiyetimde Hatice'ye metelik bırakmayacağım. Hepsini size bırakacağım.\\
\textbf{Necdet} - (Kendini kaybedercesine sevinerek) Oh yaşa\ldots.\\
\textbf{Ka{[}ymakam{]}} - Sana ne oluyor aşcıbaşı? Sus bakalım! (Pakize'ye) Ey Pakize'm buna ne dersin?\\
\textbf{Pakize} - (Mahcup) dayıcığım, güzel dayıcığım bu kadar lütfunuza ne diyeceğimi bilemiyorum. Pek minnettarım.\\
\textbf{Ka{[}ymakam{]}} - Siz ne dersiniz damadım?\\
\textbf{Necdet} - (Yine dalarak) Ne diyeceğim? Allah razı olsun derim.\\
\textbf{Ka{[}ymakam{]}} - Buna da mı karıştınız Yaşar Efendi?\\
\textbf{Şemsi} - Dayı beyciğim bilmem ki size nasıl teşekkür etmeli\ldots{}\\
\textbf{Ka{[}ymakam{]}} - O teşekkür nafile yavrum. Beni memnun etmek isterseniz daima sevişin\ldots{}\\
\textbf{Necdet} - Oh oh bari vasiyeti de ben yazayım.\\
\textbf{Ka{[}ymakam{]}} - (Devamla) Sizden bundan başka birşey istemem. E artık vasiyetten ölümden daha ziyade bahsetmeyelim. İnşaallah bu dediğim servete mümkün olduğu kadar geç sahip olursunuz\ldots{} (Pakize'ye) Haydi bakalım. Bana söyleceğin şey ne imiş\ldots{}\\
\textbf{Necdet} - Hiç. Hiç efendiö.\\
\textbf{Ka{[}ymakam{]}} - Yine mi? Bu sizin süt ninenizin oğlu da mutfak sineği gibi herşeye burnunu sokuyor..\\
\textbf{Necdet} - Affınızı rica ederim kaymakam bey. Bana soruyorsunuz zannettim de\ldots{}\\
\textbf{Ka{[}ymakam{]}} - Kes! Yeter! Pakize, senin söyleyeceğin ne imiş bakalım..\\
\textbf{Pakize} - Oo dayıcığım ehemmiyetsiz birşeydi. Biraz evvel söyleyecektim ama bilmem, unuttum.\\
\textbf{Ka{[}ymakam{]}} - Ya sen? Yaşar? (Necdet dalgınlıkla Şemsi'ye bakmaktadır) Yaşar! Yaşar, diyorum dalga geçme..\\
\textbf{Necdet} - Ha? Ben ni kaymakam bey..\\
\textbf{Ka{[}ymakam{]}} - Evet sen. Demin birşey söylemek için müsaade istemiştin. Müsaade ediyorum. Haydi söyle şimdi..\\
\textbf{Necdet} -Ben mi kaymakam bey? Bir yanlışlık var efendim. Benim size söyleyecek hiçbir şeyim yok.\\
\textbf{Ka{[}ymakam{]}} - (Şiddetle Şemsi'ye) Bu heri alimallah bunak galiba..\\
\textbf{Şemsi} - Galiba dayı bey..\\
\textbf{Ka{[}ymakam{]}} - Sen aklını ocağa mı düşürdün be herif\ldots{} (Bu esnada Zeliha elinde bir takım kağıtlar vardır- ağlar gibi burnunu çeker.)\\
\textbf{Ka{[}ymakam{]}} - Nen var kız?\\
\textbf{Necdet} - (Kendi) Eyvah bu kız bir pot kırmasın. (Pakize'ye işaretle)\\
\textbf{Zeliha} - Ben mi efendim. Şu kağıtları bahçey götürüp {[}a{]}tacağım..\\
\textbf{Ka{[}ymakam{]}} - Ey peki bunun için mi ağlıyorsun?\\
\textbf{Zeliha} - hayır efendim. Deminden küçük hanımı görmek için bir kaç gün izin istedim de o da evde kimse kalmayacak diye bana izin vermemişti. Çünkü aşcı da nişanlısına gitti. Şimdi ben de nişanlıma gitmek istiyorum da küçük hanım bir türlü izin vermediği için..\\
\textbf{Ka{[}ymakam{]}} - Ey peki kes zırlamayı. (Pakize'ye) Pakize'ciğim bu kıza üç gün izin vermiyor musun?\\
\textbf{Pakize} - A.. Dayıcığım reddetmemiştim, bu akşam gitme diye söylemiştim.. İzin vermez olur muyum?\\
\textbf{Zeliha} - (Elindeki kağıtları bırakır. Sevinçle yüzü güler) Ah, sahi mi küçük hanımcığım\ldots{}\\
\textbf{Necdet} - Haydi artık durma git.\\
\textbf{Şemsi} - Hemen koş. Durma haydi..\\
\textbf{Şemsi} -(Kızı önleyerek) Haydi işte durma çık. Çabuk çık.\\
\textbf{Necdet} - (Apar) Aşcıbaşı Yaşar bize yemek yapar (Yüksek) Yaşar.\\
\textbf{Necdet} - (Kaymakam için düşünerek Şemsi'ye bakarak) Kaymakam gitse de evimi çorbaya çeviren şu aşcıyı bir def etsem..\\
\textbf{Ka{[}ymakam{]}} - Yaşar buraya gel..\\
\textbf{Şemsi} - Baksana buraya Yaşar. kaymakam bey birşey emrediyorlar.\\
\textbf{Necdet} - Ha affedersiniz efendim.\\
\textbf{Ka{[}ymakam{]}} - (karşısında resm-i tazimde bulunan Necdet'e) Aşcı kadın izinli imiş. Hizmetçi de gitti. Onların yerin bana öğleye güzel bir omlet yapmalı..\\
\textbf{Necdet} - Ben mi efendim?\\
\textbf{Ka{[}ymakam{]}} - Hayır ben. Sen ne biçim aşcısın be?\\
\textbf{Necdet} - Affedersiniz
efendim. Omlet.. Omlet değil mi efendim?\\
\textbf{Şemsi} - Yaşar biraz dalgındır efendim (Yaşar'a) Haydş bakalım Yaşar. Süt kardeşinin evi senin evin demektir. Sıva kolları.\\
\textbf{Pakize} - Evet evet.. Haydi bakalım kardeşim, dayıma güzel bir omlet yap..\\
\textbf{Ka{[}ymakam{]}} - Miskin mendebur, haydi bakalım. Topla şu kağıt parçalarını yerden..\\
\textbf{Şemsi} - Yaşar goster kendini, sen aşcı değil misin?..\\
\textbf{Necdet} - Ben aşcıyım ama gemide\ldots{}\\
kaymakam - Ne armut adam şu yarabbi.. Gemini çok seviyorsan şimdi oraya gönderirim ha..\\
\textbf{Pakize} - Haydi, haydi Yaşar.\\
\textbf{Şemsi} - Haydi, haydi Yaşar.\\
\textbf{Ka{[}ymakam{]}} - Çabuk diyorum koş miskin herif.. (Birden kağıtları toplamakla meşgul Necdet'in arkasına bir tekme vurur.)\\
\textbf{Necdet} - (Süratle çıkarken) Koşuyorum efendim.. Aman yarabbim! Ben şimdi mutfakta ne halt ederim..\\

Perde iner

\hypertarget{perde-1}{%
\chapter{2. PERDE}\label{perde-1}}

\hypertarget{birinci-meclis-1}{%
\section{Birinci Meclis}\label{birinci-meclis-1}}

\begin{verbatim}
 [Gazanfer- Şemsi-Pakize- sonra Necdet] 

 [Perde açılmadan evvel, üç darbeyi müteakkip evvela (zil**) sonra çatal bıçakların             tabaklara vurulmasından mütevellid sesler, sonra evvela kaymakamın saniyen Şemsi'nin,          salisen Pakize'nin sesleri geliyor. Yaşar, Yaşar, Yaşar... Sonra perde açılır.]

 (Perde açıldığı zaman ortada yemek masası, halka nazaran [ortada] Pakize, sağda kaymakam,      solda Şemsi oturmuşlardır)
\end{verbatim}

\textbf{Pakize} - Yaşar, Yaşar, Yaşar\ldots{} ~\\
\textbf{Kaymakam} - Yaşar, Yaşar, Yaşar\ldots{}\\
\textbf{Şemsi} - Yaşar, Yaşar, Yaşar\ldots{}\\
\textbf{Yaşar} - (İçeriden) Efendim,efendim.\\
\textbf{Kaymakam} - Ha.. Çocuklar omletin kokusu çıktı. Neredesin oğlum?
(Necdet kıpkırmızı, ter içinde, aşcı önlüğü,elinde tencere girer)\\
\textbf{Pakize}, \textbf{Şemsi} - (Tencereyi görerek) A.. Yaşar.\\
\textbf{Necdet} - Geldim efendim, omlet hazır.\\
\textbf{Ka{[}ymakam{]}} - Bu ne? Omleti tencerede mi pisirdin?\\
\textbf{Pakize} -Mutfakta tava bulamadın mı Yaşar?\\
\textbf{Necdet} - Ha.. Tava.. Efendim? Evet tava aradım. Bulamadım.\\
\textbf{Ka{[}ymakam{]}} - Bunu sofraya böyle tencere ile mi getirirler oğlum?\\
\textbf{Necdet} - efendim bakit kaybolmasın diye.. Şey ettim. İsterseniz götüreyim, bir tabağa boşaltayım..\\
\textbf{Ka{[}ymakam{]}} - (Necdet tencereyi götürürken) Dur , dur\ldots{} Gel, zararı yok. (Necdet tencereyi sofraya koyar) Beş dakikadır sana mütemadiyen bağırıyoruz..\\
\textbf{Necdet} - Nasıl beş dakika efendim? Belki on beş dakika\ldots{}\\
\textbf{Ka{[}ymakam{]}} - Ey.. Peki neden koşup gelmiyorsun?\\
\textbf{Necdet} - Efendim buradaki ocaklara elim alışmamış.. Gemi..Ocaklarına benzemiyordu..\\
\textbf{Ka{[}ymakam{]}} - Kes.. Aç bakalım tencereni! (Tencerenin kapağını Necdet açar)\\
\textbf{Şemsi} - (Tencereye bakarak) Omlaeti nasıl yaparsınız Yaşar Efendi!\\
\textbf{Necdet} - Nasıl mı? Her aşcı gibi bir tencereye \ldots{} Yahut tavaya yağ, tuz, biber koyarım. Üstüne yumurtaları kırar ocağa koyarım.\\
\textbf{Şemsi} - Hepsi bu mu?\\
\textbf{Necdet} - Hayır.. Yani.. Şey.. İşte ne lazımsa koyar pişiririm.\\
\textbf{Ka{[}ymakam{]}} - Bu kabuklar ne? Yumurtanın kabuklarını da beraber mi pişirdin?\\
\textbf{Necdet} - Buna emin misiniz kaymakam bey? Ay onun içinde kabuk da mı var?\\
\textbf{Ka{[}ymakam{]}} - Ne demek emin miyim? ({[}tencereyi göstererek{]}) Bunlar ne?\\
\textbf{Necdet} - (Bakarak) A.. Sahi! (Gülerek)..İşte bu tuhaf..\\
\textbf{Ka{[}ymakam{]}} - (Hiddetle) Bir de tuhaf buluyorsun ha?\\
\textbf{Necdet} -Şey affedersiniz.. Tuhaf demişim, şaşırdım.. Bu dalgınlıktan dolayı affınızı rica ederim. Telaşla içine düşürmüşüm, alayım efendim. (Elini uzatır)\\
\textbf{Ka{[}ymakam{]}} - Sen hiç şenlik görmedin mi be adam. Tencere el sokulur mu? Haydi kaldır şunu ortadan.. Çabuk ol..
(Necdet tencereyi alır)\\
\textbf{Pakize} - Sonra ne varsa getir..\\
\textbf{Necdet} - Külbastıyı mı? Peki..\\
\textbf{Ka{[}ymakam{]}} - Buraya gel..\\
\textbf{Necdet} - Efendim.\\
\textbf{Ka{[}ymakam{]}} - Külbastın nasıl bari, kanlı kanlı isterim ha!..\\
\textbf{Necdet} - Kanlı kanlı efendim.. (Kendi kendine) Görürsünüz kanlıyı.. Eğer topunuzu zehirlemezsem bana da işte öylesine (Çıkar)\\
\textbf{Ka{[}ymakam{]}} - Garaibden bir aşcı?\\
\textbf{Şemsi} -Bir parça vurdum duymazdırda.\\
\textbf{Pakize} - (Kalkarak) Zannederim biraz mutfağa baksa{[}m{]} iyi olacak\\
\textbf{Ka{[}ymakam{]}} - Yok canım efendim otur Allah aşkına bırak şu hımbılı başının çaresine baksın.\\
\textbf{Şemsi} - Biraz daha zeytin almaz mısınız, sevgili dayıcığım?\\
\textbf{Ka{[}ymakam{]}} - Ma'al-i memnuniye\\
\textbf{Şemsi} - (Pakize'ye) Siz efendim?\\
\textbf{Ka{[}ymakam{]}} - Siz, siz.. Oldu mu ya.. Bu merasim de ne oluyor. bakın yavrularım bunu sevmedim işte.. Şimdi, İstanbul'da karı koca birbirine böyle mi hitap ediyorlar. A-Ah.. Siz, kaba düşüyor haydi şimdi derhal birbirinize sen diye hitap ediniz..\\
\textbf{Şemsi} - Hayhay efendim bana göre hava hoş.. (Pakize'ye) Ey sen almaz mısın?\\
\textbf{Pakize} - (Biraz mahcup) Sen bilirsin.. Teşekkür ederim.\\
\textbf{Ka{[}ymakam{]}} - Hele şükür. Ey.. Çocuklar, şimdi söyleyin bakalım.. Madem ki biz bizeyiz.. Yolcudan ne haber?\\
\textbf{Pakize} - (İkisi de birşey anlamayarak) Hangi yolcudan?\\
\textbf{Şemsi} - Yolcudan mı efendim!\\
\textbf{Ka{[}ymakam{]}} - Canım hiç bklediğiniz küçük bir yolcu yok mu?\\
\textbf{Pakize} - (Mahcup) A.. Hayır dayıcığım!\\
\textbf{Ka{[}ymakam{]}} - Ne? Amma yaptınız ha.. Öyle mi damadım? Hiç mini mini Türk bahriyeleisi yetiştirmek istemiyor musunuz?..\\
\textbf{Şemsi} - İstemez olur muyum, dayı bey..Benim de istediğim bu ama..\\
\textbf{Ka{[}ymakam{]}} - Demek buna iktidarınız yok ha?\\
\textbf{Şemsi} - İktidarım mı yok? Hayır efendim öyle değil.. Bana göre hava hoş.. Canıma minnet ama..\\
\textbf{Ka{[}ymakam{]}} - Nasıl? (Pakize'ye) Demek kabahat sende, öyle mi kızım?\\
\textbf{Pakize} - (Pek mahcup) Çocuk olmasını arzu etmek kafi değildir dayıcığım.\\
\textbf{Ka{[}ymakam{]}} - (Şemsi'ye) Demek kabahat sende..\\
\textbf{Şemsi} - Aman efendim, şey efendim..\\
\textbf{Ka{[}ymakam{]}} - Hele durun çocuklar ben bunun en keskin çaresini biliyorum. Size söyleyim.. Bizim Selimiye'nin çarkçıbaşısı söyledi.. Baharlı yemekler yeyin.. Biberli çorba.. Mayonezli ıstakoz! Birçok yemiş.. Biraz da şarap hiç kaçmazmız!\\
\textbf{Şemsi} - Acaba?\\
\textbf{Ka{[}ymakam{]}} - Acabası yok damadım.. Tecrübe edin.. İsterseniz bu akşam size kulüpte bir yemek yedireyim (Şemsi'ye) Şöyle adamakıllı iştiha ile yenmiş bu tarzda bir akşam yemeği neler yapmaz.. Neler ha Necdet Bey? Akşam döner dönmez de bana mini mini bir yeğen hazırlamak üzere kolları sıvarsınız hah hah hah!\\
\textbf{Şemsi} - Hay hay dayıcığım hay hay!\\
\textbf{Kaymakam} - (Pakize'ye) Ey sen ne diyorsun bu fikre Pakize?\\
\textbf{Pakize} - (Gittikçe mahcup ve kıpkırmızı) Dayıcığım.\\
\textbf{Ka{[}ymakam{]}} - Ay.. İyi vallahi.. Altı senelik karı kocalıktan sonra hala cahillik mi.. (Şemsi'ye) Haydi bakayım damadım.. Karını öp bakayım.. Şu kırmızı yanaklarından.. Benden sıkılmayın be.. ben müsaade ediyorum yavrularım.\\
\textbf{Şemsi} - Aman dayıcığım öpeyim.. İstediğiniz kadar; istediğiniz kadar.\\
\textbf{Ka{[}ymakam{]}} - Adamlar.. Şuraya bakın.. Kumrular gibi.. Kumrular gibi..\\
\textbf{Şemsi} - (Öperek) Karıcığım.. Sevgili karıcığım..\\
\textbf{Necdet} - (Elinde büyücek bir tabakla girer ve bu hali görünce haykırır) Ahh! (Tabağı düşürür)\\
\textbf{Şemsi} - Vay..\\
\textbf{Pakize} - Eyvah\\
\textbf{Ka{[}ymakam{]}} - (Hiddetle) Bu ne?\\
\textbf{Necdet} - Elimden kaydı efendim..\\
\textbf{Ka{[}ymakam{]}} - Beceriksiz, armut ağa..\\
\textbf{Şemsi} - Gözünüz etrafta olacağına önünüze bakamaz mıydınız?\\
\textbf{Pakize} - (Kalkarak) Aman Yarabbi.. Vah, vah, vah..\\
\textbf{Necdet} - karımı öpmekten sizi men ederim..\\
\textbf{Şemsi} - Dayınız emretti.. Yoksa ona hakikati mi söylememi isterdiniz..\\
\textbf{Necdet} - (Şiddetle) O.. Hayır.. Hayır..\\
\textbf{Ka{[}ymakam{]}} - Bırak kızım.. Bırak toplasın. Ulan Yaşar dua et talihine..Alemde sen benim emrimde olmalıydın ki..\\
\textbf{Şemsi} - O vakit dünyanın kaç bucak olduğunu anlardın..\\
\textbf{Necdet} - Kaymakam bey.. Emin olun ki pek müteessirim.. Fakat ne yapım olan oldu.. Halbuki külbastılarım da pek iyi olmuştu.. Kanlı kanlı..\\
\textbf{Şemsi} - Haydi, haydi.. gevezeliği bırakta toplayın şunları haydi çabuk..\\
\textbf{Necdet} - Emir de ediyor.. Babasının evi gibi bir de emir ediyor ha..\\
\textbf{Şemsi} - San topla şunları diyorlar..\\
\textbf{Necdet} - Peki ama ne bağırıyorsunuz?\\
\textbf{Ka{[}ymakam{]}} - Ne diyor, karşılık mı veriyor?\\
\textbf{Necdet} - Hayır kaymakam bey.. Birşey demedim efendim.. Başüstüne dedim.\\
\textbf{Pakize} - Ah dayıcığım yemek yiyemediğiniz için ne kadar mahcubum..\\
\textbf{Şemsi} - Yaşar'a söyle de gitsin aşağıdan börek alsın gelsin bari..\\
\textbf{Pakize} - Sahi hakkın var..\\
\textbf{Necdet} - Oh.. Senli benli görüşüyorlar..\\
\textbf{Ka{[}ymakam{]}} - Yok canım.. Yok canım..\\
\textbf{Necdet} - Demek börekçiye gitmeyeceğim öyle mi?\\
\textbf{Ka{[}ymakam{]}} - Hayır.. Sen kahveleri hazırla..\\
\textbf{Şemsi} - Kahveleri olsun iyi pişirmeli, anlıyor musun?\\
\textbf{Necdet} - Kaymakam bey. Bundan emin olabilirler\ldots{} Kahvem iyidir. Ah nasıl bir tehlikeden kurtulduklarını bilseler.. (Çıkar)\\
\textbf{Pakize} - Vallahi cidden ben mahcubum..\\
\textbf{Şemsi} - Ya ben, ya ben..\\
\textbf{Ka{[}ymakam{]}} - Adam sen de, ne ehemmiyeti var.. (Pakize'nin verdiği yemiş için) Teşekkür ederim. Zaten dünyada ehemmi,yetli yalnız birşey vardır. O da : Sevgi.\\
\textbf{Şemsi} - Ah dayı bey ne güzel söylediniz.. Ne yüksek fikir Yarabbi.. (Pakize'ye) İşitiyor musun sevgilim? İşitiyor musun karıcığım?\\
\textbf{Pakize} - Evet, evet azizim.\\
\textbf{Ka{[}ymakam{]}} - Ben bile bu yaşta sevgiyi herşeyin fevkinde tutuyorum. Onsuz edemem. Şu anda zihnimi meşgul eden birşeyi size söylemeden duramayacağım azizim.\\
\textbf{Ka{[}ymakam{]}} - Buraya gelirken vapurda ta yanı başımda mini mini, (tombul tombul) bir hanım oturmuştu..\\
\textbf{Şemsi} - Eyy.. Dayı bey desenize siz de güzellere..\\
\textbf{Ka{[}ymakam{]}} - Güzellere meftunum ya.. Ne zannettiniz damadım.. Ah.. Yirmi yaş küçülmek için neler vermezdim bugün..\\
\textbf{Pakize} - Kompliman istiyorsunuz galiba dayıcığım..\\
\textbf{Şemsi} -Zaten kırkından fazla göstermiyorsunuz ki..\\
\textbf{Ka{[}ymakam{]}} - Yok canım sahi mi?\\
\textbf{Şemsi} - İşte.. Ancak..\\
\textbf{Ka{[}ymakam{]}} - (Şemsi'ye) Ne ise.. Bugün bir türlü cesaret edip bu hanıma açılamadım..\\
\textbf{Şemsi} - Oo.. hata etmişsiniz dayı bey.. Cesaret etmlei açılmalı idiniz.. Sonra keka.. Gel keyfim gel..\\
\textbf{Ka{[}ymakam{]}} - O da buraya adaya çıktı ama.. Sonra iskelede gözümden kaybettim (tombul tombul) birşey idi..\\
\textbf{Pakize} - Nasıl (tombul tombul) mu?\\
\textbf{Şemsi} - Siz de (tombul tombu)l mu seviyorsunuz dayı bey?\\
\textbf{Ka{[}ymakam{]}} - Ya sorma azizim..\\
\textbf{Pakize} - Ada'ya nı çıktı dediniz?\\
\textbf{Ka{[}ymakam{]}} - Evet.\\
\textbf{Pakize} - Durun bakayım.. Elinde beyaz bir şemsiye var mıydı?\\
\textbf{Ka{[}ymakam{]}} - Evet..\\
\textbf{Pekize} - Beyaz bir şapka..\\
\textbf{Ka{[}ymakam{]}} - Evet..\\
\textbf{Pakize} - Lacivert bir esvap..\\
\textbf{Ka{[}ymakam{]}} - Evet.. Evet..\\
\textbf{Pakize} - Mavi gözlü, beyaz tenli bir kadın.. Balık etinde ne güzel\\
\textbf{Ka{[}ymakam{]}} - Ta kendisi..\\
\textbf{Şemsi} - Acaba?\\
\textbf{Pakize} - Dayıcığım bu hanımı tanıyorum dersem ne dersiniz?\\
\textbf{Ka{[}ymakam{]}} - Aman sahi mi Pakize?\\
\textbf{Pakize} - Tabii.. Pek iyi tanırım.. Altı seneden beri görüşmediğim mektep arkadaşım, Seha.. Seha Şemsi Hanım..\\
\textbf{Ka{[}ymakam{]}} - (Vecd içinde)
Seha mı? (Şemsi'nin boğazında kalır) Ne oluyorsunuz Necdet Bey?\\
\textbf{Şemsi} - Birşey değil.. Su içerken boğazımda kaldı da..\\
\textbf{Necdet} - (Gelir) İşte kahveler..\\
\textbf{Şemsi} - (Kendi) Karım burada.. Birbirlerini de tanıyorlar ha?\\
\textbf{Pakize} - (Necdet'e) Yaşar likör çıkar.\\
\textbf{Necdet} - Peki efendim. (Kendi) Allah için tuttuğum mesleğe diyecek yok..\\
\textbf{Ka{[}ymakam{]}} - (Sofradan kalkarak) Peki ama Pakize.. Nasıl bildin bakalım?\\
\textbf{Pakize} - Nasıl mı? Pek sade. Deminden buraya beni görmeye gelmişti..\\
\textbf{Şemsi} - (Kendi) Buraya mı gelmiş?\\
\textbf{Ka{[}ymakam{]}} - Sonra sana hepsini anlattı öyle mi? Ah bu kadınlar.\\
\textbf{Pakize} - Evet. Vapurda vapurda dinç güzel bir ihtiyar kaymakam görmüş..\\
\textbf{Ka{[}ymakam{]}} - Dinç güzel dedi ha!!\\
\textbf{Pakize} - Hasılı işte bu manada birşey.. Bir sigara almaz mısınız dayıcığım..\\
\textbf{Ka{[}ymakam{]}} - Hay aksi iş (Şemsi'ye) Hakkınız varmış damadım.. Cesaret etmeli açılmalı imiş.. Sonra da keka.. Gel keyfim gel.. Ha?\\
\textbf{Şemsi} - Oo.. Bu biraz fazla olmaz mı ya?\\
\textbf{Ka{[}ymakam{]}} - Fazla mı olur, ne gibi?\\
\textbf{Şemsi} - Şeyy.. Hayır.. Cümlemi bitiremedim dayı bey: Bu kadar çekingenlik fazla olmaz mı? diyecektim..\\
\textbf{Ka{[}ymakam{]}} - Ha.. Şöyle..\\
\textbf{Pakize} - (Şemsi'ye) Sigara?\\
\textbf{Şemsi} - Hay hay..Süt..Şey.. Sevgili karıcığım.. Peki ama Ada'ya ne yapmaya gelmiş bu.. Seza Resmi Hanım\\
\textbf{Ka{[}ymakam{]}} - Seha Şemsi Hanım\\
\textbf{Şemsi} - Seha Şemsi Hanım?\\
\textbf{Pakize} - Beni görmeye ve kabilse dayımın sayesinde bir izin koparmaya gelmiş..\\
\textbf{Ka{[}ymakam{]}} - Yok canım sahi mi Pakize..\\
\textbf{Şemsi} - Hay aksi şeytan hay..\\
\textbf{Pakize} - Düşünün zavallı kocası Şemsi Bey bir buçuk seneden beri daha bir defa izinli gelememiş.. Galiba suvarisi ile mi ne arası açılmış..\\
\textbf{Ka{[}ymakam{]}} - Oo.. Bu mu?.. Bu ise pek kolay birşey.. Ankara'ya gider gitmez hemen yarın vekaletten bir kolayını buluruz..\\
\textbf{Şemsi} -(Telaşla) Aman dayıcığım sakın böyle birşey yapmayınız..\\
\textbf{Ka{[}ymakam{]}} - Niçin?\\
\textbf{Şemsi} -Hayır birşey için değil.. Yani malum a tavsiyeler daima göze batar çirkin görünür.\\
\textbf{Ka{[}ymakam{]}} - Ben tavsiye falan etmeyeceğim.. O çocuk zaten bunu hak etmiş (Pakize'ye) Sen bu hanıma söyle.. Benim gözüme baksın ben bu işi yaparım..\\
\textbf{Şemsi} - (Kendi) İşte şimdi bundan sonra benim başıma gelenler..\\
\textbf{Pakize} - Bunu kendisine bizzat da söyleyeniz dayıcığım. Zira Seha akşama doğru buraya gelecek.. Akşam da bizde kalacak.\\
\textbf{Şemsi} - (Yüzünün ne renk aldığını belli etmemek için elleriyle yüzünü kapayarak) Ne.. Akşama burada mı?.\\
\textbf{Pakize} - Ama ben kendisini birkaç gün bırakmayacağım..\\
\textbf{Ka{[}ymakam{]}} - (Neşeli) Ey Vallahi buna diyecek yok işte.. Onu da yemeğe götürürüz.. (Şemsi'ye) Olmaz mı yeğenim.. Siz karınızla kol kola ben de bu hanımla kol kola deniz kenarında bir piyasa ederiz.\\
\textbf{Şemsi} - (Zorla gülmeye çalışarak) Ya!..Ne güzel fikir hay hay dayı bey.. Ne güzel fikir..(Kendi) Oof Yarabbi burada insan boğulacak..\\
\textbf{Necdet} - (Elinde likor tepsisiyle) İşte likörleri getirdim.\\
\textbf{Ka{[}ymakam{]}} - (Tan bu esnada kahvesinin ilk yudumunu içer) Hay Allah cezanı versin.. Bu nasıl kahve? Zehir gibi. Hani bunun şekeri..\\
\textbf{Necdet} - A.. Ben de kendi kendime ``Mutlaka bu kahvenin birşeyi eksik ama'' diyordum.. Bakın şekeri eksikmiş.\\
\textbf{Pakize} - Aman Yarabbi şeker koymayı unutmuş..\\
\textbf{Ka{[}ymakam{]}} - Ne eşek şey.. Ben ömrümde böyle aşcı görmedim.\\
\textbf{Necdet} - Yeniden yapayım kaymakam bey yeniden.\\
\textbf{Ka{[}ymakam{]}} - Hayır, hayır.. Sade benim içinse lüzum yok.. Zaten benim pek vaktim yok..eğer telefon varsa Kasımpaşa'ya Havuzlar İdaresi'ne telefon etmek isterdim..\\
\textbf{Necdet} - Beyoğlu sekiz dokuz sekiz.\\
\textbf{Ka{[}ymakam{]}} - Vay sen Havuzlar İdaresş'nin telefon numarasını nereden biliyorsun\\
\textbf{Necdet} - Suvariden işitmiştim efendim.. Hatırımda kalmış.. Pek kuvvetli hafızam vardır beyefendi..\\
\textbf{Pakize} - Bizde telefon yok dayıcığım.. Yaşar sizi eczahaneye götürsün. Haydi Yaşar..\\
\textbf{Ka{[}ymakam{]}} - Haydi bakalım.. Düş önüme bozuk dümen, çürük tekne.\\
\textbf{Necdet} - (Çıkarken kendi kendine) Oh!. Bunlar da hediyesi.. (Çıkarlar)

\hypertarget{ikinci-meclis-1}{%
\section{İkinci Meclis}\label{ikinci-meclis-1}}

\begin{verbatim}
 [Şemsi- Pakize- Necdet]
\end{verbatim}

\textbf{Şemsi} - (Kendi) Şimdi aklını başına topla, şaşırma..Şimdi karım akşama gelecek.. Saat üç herhalde ricat için ferah ferah iki saatim var..\\
\textbf{Pakize} - Ne o.. Yaşar, düşünceli, endişeli duruyorsunuz?\\
\textbf{Şemsi} - Ben mi?\\
\textbf{Pakize} - Evet.. Demin Seha Şemsi Hanım ismini telaffuz ettiğimden beri bir tuhaf duruyorsunuz?\\
\textbf{Şemsi} - Yok.. Sizi temin ederim..\\
\textbf{Pakize} - Ben bunun sebebini pekala keşfediyorum.. Sehanın burada bulunması bizin aramıza bir hail koyacağından ve sonra intikamımı almaktan vazgeçeceğimden korkuyorsunuz değil mi? Fakat kalbiniz müsterih olsun, korkmayınız azizim. Kocamdan intikam almak için sizinle birleşmemi tavsiye eden kendisi oldu.\\
\textbf{Şemsi} - Ya.. Bunu size o mu tavsiye etti?\\
\textbf{Pakize} - Ya.. Seha ile ben öteden beri izdivaç hususlarında hemfikirizdir. Daha demin bana ``Aldatılmış bir kadının birinci vazifesi kocasına aynen mukabele etmektir.'' diyordu.\\
\textbf{Şemsi} - Bunu dedi ha?\\
\textbf{Pakize} - Evet kısasa kısas!.. Dişe diş, göze göz.. İşte bunun için ben de vazifeme başlıyorum..Şimdi herşeyi unuttum.. Kocama aynen mukabele için en zararsız fırsatı kullanıp bu vazifemi ifa etmekten başka birşey düşünmüyorum, herşeyi unuttum.. hatta sizinle süt kardeşi olduğumuzu bile hatırıma getirmiyorum.\\
\textbf{Şemsi} - Şu anda hatırımda ondan başka birşey yok. Deminden beri sizinle süt kardeş olduğumuzu.. Ve bir dakikalık bir zaaf içinde süt kardeşime karşı ne büyük bir hata işlediğimi düşünüyorum.\\
\textbf{Pakize} - Nasıl? Nasıl? Demek şimdi..\\
\textbf{Şemsi} - Evet evet şimdi deminki hisler altında değilim canım efendim. Size karşı hissettiğim şeyler ne olursa olsun.. Buraya geldiğim zaman beni o kadar nezaketle kabul eden.. süt enişteme karşı hıyanette bulunmak istemiyorum..\\
(Necdet dipte gözükür ve onlara gözükmeyerek paravananın arkasından muhavereyi dinler) Ben Necdet Bey'i pek çabuk anladım.. Zevcinizin altından bir kalbi elmas gibi temiz ve yüksek bir ruhu var, süt kardeşim. (Bu sözlerin üzerine Necdet'in yüzü parlar) Oo.. Şimdi bana vereceğiniz cevabı biliyorum. O size hıyanet etti diyeceksiniz değil mi? Evet! Bu bir hakikattir.. Fakat o bugün bundan kalbiyle nedamet etmiş bulunuyor.. Onu bu hıyanetinden dolayı mahkum ediniz!.. fakat bu hükmünüzü icra etmeyiniz. Fiilen ihanet etmeyiniz.\\
\textbf{Necdet} - Aman yarabbi, bu ne iş..\\
\textbf{Şemsi} - Düşününüz intikamın daha sırası değildir. Pakize Hanım ne kadar haklı olursanız olunuz ona karşı hissettiğiniz adaveti {[}düşmanlığı{]} unutun.. Ve izdivaç denilen ulvi arkadaşlık namına ona eliniz uzatın..\\
\textbf{Necdet} - (Elinde mendil teessür ve heyecandan ağlayarak meydana çıkar) Ah.. Yaşar.. Bu ne güzel, ne yüksek, ne ulvi sözler canım..\\
\textbf{Pakize} - Ah Necdet..\\
\textbf{Necdet} - Her şeyi işittim. Heyecanımdan ağlıyorum, kusuruma bakmayın. (Hala ağlayarak Şemsi'ye yaklaşır) Yaşar Yaşar ben sizin hakkınızda çok fena fikirler besliyordum. Sizi layıkıyla tanımamıştım.. Öz ve ahad {[}tek{]} kardeşim olsa bu kadar iyi söylemezdi (Coşarak) Bırakın size sen diye hitap edeyim. Müsaade eder misin?\\
\textbf{Şemsi} - Sen bilirsin kardeşim..\\
\textbf{Pakize} - (Kendi) Ne çabuk can ciğer oldular..\\
\textbf{Necdet} - (Şemsi'nin elini sıkarak) vallahi Yaşar'cığım bei ağlattın sen ömür bir herifmissin be!..\\
\textbf{Şemsi} - Sen de Necdet'ciğim sen de olur heriflerden değilmişsin!\\
\textbf{Necdet} - Bana bak gel öpüşelim seninle..\\
\textbf{Şemsi} - Şimdi ben sana dipeceğim..\\
\textbf{Pakize} - (Kendi) Şimdi de öpüşüyorlar..\\
\textbf{Necdet} - (Pakize'nin tarafına geçerek) Ey Pakize'ciğim.. (Pakize hiç cevap vermeyerek elindeki tepsinin üzerine bardakları, tabakları koymakla meşgul olur) Pakize (Bir vakfe) Cevap vermiyor musun?\\
\textbf{Pakize} - (Baridane {[}soğuklukla-soğukça{]}) Şimdi işim var.\\
\textbf{Şemsi} - Size yardım edeyim kardeşim..\\
\textbf{Pakize} - Lüzum yok teşekkür ederim.\\
\textbf{Necdet} - Zararı yok, yine iş gör. Bu bana cevap vermene mani olmaz ki. Sen yalnız beni affettiğini söyle yeter.. yaşar'ın dediği ulvi arkadaşlığı bozmakta ısrar etmezsin zannederim. Sonra dünya üzerindeki ulvi arkadaşların hali ne olur. (Pakize daima işiyle meşgul birşey söylemeden ve hiç bozmadan tepsiyi alır ve evvla Necdet'e sonra Şemsi'ye istifafkar {[}küçümseyici{]} nazarlar fırlatarak çıkar.)\\
\textbf{Şemsi} - (Kendi) Ah bu başka (masanın örtüsünü örtmekle meşgul)\\
\textbf{Necdet} - Hişt Yaşar Yaşar'cığım sen şuna yine söyle! Deminki gibi yine söyle.. Başladığın işi ikmal et..\\
\textbf{Şemsi} - Merak etme Necdet'ciğim.. Sen bana bırak üzülme..\\
\textbf{Necdet} - Aman kardeşim, ocağına düştüm.\\
\textbf{Şemsi} - Merak etme yanmazsın (Kendi) Şimdi marifet vaktiyle buradan sıvışmaktadır.(Yüzüstünde kalan bazı şeyleri alarak çıktığı koridordan çıkar)\\

\hypertarget{ucuncu-meclis-1}{%
\section{Üçüncü Meclis}\label{ucuncu-meclis-1}}

\begin{verbatim}
 [Necdet-Kaymakam]
 (Necdet kendine bir bardak şarap koyarak)
\end{verbatim}

\textbf{Necdet} - İnsan bazı adamları ilk görüşte bir türlü anlayamıyor. Hay Yaşar hay. tam sütü halis bir çocuk. İzni bitip gemisine giderse her hafta şokola göndereceğim.. (Bir koltuğa kurulur ayaklarını başka bir sandalyeye koyar elindeki süze süze içmeye başlar) Öff şurada biraz dinlenirim ya. Mutfakta ocakların karşısında piştim.\\
\textbf{Ka{[}ymakam{]}} - (Girdiğin{[}de{]} Necdet'i görmemiştir) Bu iş de bitti. (Necdet'in lakayd ve müsterih yukarıda şarap içtiğini görerek) Oh oh.. Maşaallah. Yaşar efendi. Adam bozma istifini.\\
\textbf{Necdet} - (Hızla kalkarak) Al bakalım.. Gelsi..\\
\textbf{Ka{[}ymakam{]}} - Necdet Bey {[}baskı hatası{]} ne zıkkımlanıyordun?.. Orada.\\
\textbf{Necdet} - Bir bardak şarap lütfettilerde onu içiyorum, kaymakam bey..\\
\textbf{Ka{[}ymakam{]}} - Şarap içiyor? karşımda şarap içiyor.. Sıkılma olmadıktan sonra.. (Birdenbire bağırarak) Bırak elinden şunu! Damadım karısıyla sana iyi yüz vermişler. Ama bu kadar laubalilik benim hoşuma gitmez..\\
\textbf{Necdet} - Affedersiniz kaymakam bey.\\
\textbf{Ka{[}ymakam{]}} - (Sigara tabakasını çıkaraır boş olduğunu görerek) Hay aksi şeytan tütünüm de kalmamış.\\
\textbf{Necdet} - (Süratle hareket ederek) Durun kaymakam bey. Şimdi takdim ederim.. (necdet cebinden bir deste anahtar çıkarır biriyle büfenin gözünü açar ve kaymakama sigara paketini uzatır)Buyurun efendim.\\
\textbf{Ka{[}ymakam{]}} - Bu ne anahtarlar sende mi duruyor?\\
\textbf{Necdet} - Vay canına tabii..\\
\textbf{Ka{[}ymakam{]}} - Kimseye danışmadan istediğin kadar çıkarıp alıyorsun ha.. Ulan seni bu halde gören ev sahibi zanneder. Keyfine buyruk yaşıyorsun. Hep bunlara Necdet bey mi müsaade ediyor.\\
\textbf{Necdet} - evet kaymakam bey Necdet Bey müsaade ediyor. Ben gelir gelmez yaşar senin suratın hoşuma gidiyor,işte şarap şurada sigara burada. nah işte anahtarlar dedi. Çünkü siti temiz çocuktur.\\
\textbf{Ka{[}ymakam{]}} - (hayretle) Aşkolsun ona..\\
\textbf{Necdet} - Ah efendim ben de Necdet Bey'i seviyorum. Necdet Bey çok iyi bir adam.\\
\textbf{Ka{[}ymakam{]}} - Evet ama bu iyiliğini suistimal ediyorsun 8Nevdet'in suratına dikkatle bakarak) Sahi senin suratın da şirinmiş..\\
\textbf{Necdet} - Ya öyledir kaymakam..\\
\textbf{Ka{[}ymakam{]}} - Biraz aptacasın ama.. eyy nereleri gezdin bakalım söyle.\\
\textbf{Necdet} - (Kendi) Eyvah..\\
\textbf{Ka{[}ymakam{]}} - Cevap versene be.\\
\textbf{Necdet} - Şeye gitmiştik efendim. Tabii biliyorsunuz Aytaş çok gezmiş. Hep Karadeniz'de işte şey limanı ile şey feneri arasında mekik dokuduk.\\
\textbf{Ka{[}ymakam{]}} - (Hiddetle) Şeyiyle şey limanıyla.. İsim söyle be herif.\\
\textbf{Necdet} - Söyleyeyim efendim. Gemim esas olarak en ziyade Trabzon önünde bulunur. Sonra iki günde bir Samsun'a İnebolu'ya, Kastamonu , Hanya, Konya, Balıkesir, Eskişehir'e giderdik.\\
\textbf{Ka{[}ymakam{]}} - Ne, ne, ne Balıkesir, Eskişehir mi? Sen ne çalıyorsun\\
\textbf{Necdet} - (Kendi) Eyvah!.. (Aşikar) Hayır efendim müsaade buyurun gemimiz İzmir'e gelince bazan izin alır oralara da giderdim demek istemiştim.\\
\textbf{Ka{[}ymakam{]}} - Ha şöyle..\\
\textbf{Necdet} - (Kendi) Aman Yarabbi çocukken coğrafya imtihanında bu kadar terlememiştim.\\
\textbf{Ka{[}ymakam{]}} - (Kendi kendine gülerek) Bıraksam bütün Anadolu'yu çorba gibi karıştıracak. Sen galiba gemide mutfaktan başka birşey bilmiyorsun. Ocağın karşısında beynin erimiş. Haydi bakalım koş, benim kasketle eldivenlerim bastonun nerede ise bul getir.\\
\textbf{Necdet} - Başüstüne kaymakam bey\ldots{} (çıkar kendi kendine) Ah şu adam çiflikten gitse ne sevineceğim.\\

\hypertarget{dorduncu-meclis-1}{%
\section{Dördüncü Meclis}\label{dorduncu-meclis-1}}

\begin{verbatim}
 [Kaymakam- sonra Seha]
\end{verbatim}

\textbf{Ka{[}ymakam{]}} - (Necdet'in arkasından) Alemde sen benim emrimde olmalıydın ki Ben sana Hanya'yı Konya'yı gösterirdim.\\
\textbf{Seha} - (Girer) Aman.. Yarın dönüşümde uğrarım.\\
\textbf{Ka{[}ymakam{]}} - (Kalkarak) Bir hanım.. (Evvela birdenbşre Seha'yı tanıyamaz)\\
\textbf{Seha} - (Hayretle) A.. İhtiyar kaptan burada..\\
\textbf{Ka{[}ymakam{]}} - (Mütehayyir derhal tanır) Aman vapurdaki piliç (Yüksek sesle hitap ederek) Buyursunlar Seha Şemsi Hanım..\\
\textbf{Seha} - Nasıl efendim? Benim ismimi biliyor muydunuz?\\
\textbf{Ka{[}ymakam{]}} - Biraz evvel isminizi hemşire-zadem Pakize'den duydum öğrendim.\\
\textbf{Seha} - A.. Demek kaymakam Gazanfer sizsiniz, öyle mi?\\
\textbf{Ka{[}ymakam{]}} - Evet kulunuz..\\
\textbf{Seha} - A.. Estağfurullah.. Teşerrüf ettim efendim.\\
\textbf{Ka{[}ymakam{]}} - O şeref benim olsun.. Cıvanım\\
\textbf{Seha} - Bakınız ne garip. Benim de buğun Ada'ya gelmektten maksadım\\
\textbf{Ka{[}ymakam{]}} - Evet onu da biliyorum efendim. Pakize'den onun da haberini aldım.\\
\textbf{Seha} - Aman Yarabbi vapurda bana resimli gazeteniz vermek istediğiniz zaman almamıştım. Bu kabalığıma şimdi nadimim..\\
\textbf{Ka{[}ymakam{]}} - (Zen-perest {[}kadın düşkünü-zampara{]} tavırla yaklaşarak) Evet kaşlarınız çattınız ve incecik sesinizle ``Teşekkür ederim'' demiştiniz. Ne zararı var.. Siz şu kısmete bakın: Meğerse burada birleşecekmişiz. Buyurunuz oturunuz efendim.\\
\textbf{Seha} - Teşekkür ederim. Pakize nerede efendim.\\
\textbf{Ka{[}ymakam{]}} - Buralarda idi. Şimdi gelir. Oturun da bekleyelim.\\
\textbf{Seha} - (Oturur. Kaymakamda bir sükut. kaymakam Seha'ya bakar, bıyık burar. Sonra gülümseyerek içini çeker. seha bunu görür önüne bakar)\\
\textbf{Ka{[}ymakam{]}} - (Yanı başındaki masadan bir gazete alıp ona vererek) Eğer içiniz sıkılırsa şu gazeteyi alın eğlenirsiniz..\\
\textbf{Seha} - (Kahkaha ile) Hayhay efendim. (Alır) Tıpkı vapurdaki gibi söylediniz.\\
\textbf{Ka{[}ymakam{]}} - Ya.. Tıpkı vapurdaki halleri de geçirmeye başladım.. O vakit reddetmiştiniz, görüşemedik. Şimdi kabul ediyorsunuz demek ki görüşebiliriz.O vakitki mahrumiyetimi şimdi telafi edeceğim. Ah hanımefendi ben çok sıkılganımdır. )Elini kalbine götürmek ister) Bakınız nasıl atıyor..\\
\textbf{Seha} - A okadar fazla değilsiniz..\\
\textbf{Ka{[}ymakam{]}} -Allahaşkına müsaade buyurun zira.. Hani.. Ayağıma gelen bir kısmet gibi, bir hazine gibi burada karşıma çıkmanız bende yelkenleri suya indirdi. Siz.. Siz yok mu..\\
\textbf{Seha} - A.. Rica ederim kaymakam bey. Pakize size benden bahsetmiş ama herşeyi tamam söylememiş. Ben kocamka dargın değilim. Zevcim beni sever ben de zevcimi severim.\\
\textbf{Ka{[}ymakam{]}} - İyi ya meleğim. Siz yine sevişin..\\
\textbf{Seha} - Kaymakam bey müsaade edin..\\
\textbf{Ka{[}ymakam{]}} - Peki.. Sakın kusuruma bakmayın Seha Hanım sizi burada görür görmez rüya zannettim.. Rüyada bir adam gibi de yaptığımı bimiyordum. Şimdi uyandım. Kendime geldim affedersiniz. Bitti.\\
\textbf{Seha} - (Yanına gider) Ama bana darılmayın da.\\
\textbf{Ka{[}ymakam{]}} - Ooo rica ederim. Bu bir sağanaktı geçti. Siz darılmayın da\\
\textbf{Seha} - Estağfurullah..\\
\textbf{Ka{[}ymakam{]}} - Ben darılmadığımı ispat için bugün buraya gelmekten maksadınız ne ise onu temin edeceğim.. Oldu mu?\\
\textbf{Seha} - Yani zevcim izinli gelebilecek değil mi?\\
\textbf{Ka{[}ymakam{]}} - Tabii.. Zevciniz şimdiden izinli demektir.\\
\textbf{Seha} - Kaymakam bey bilseniz beni ne kadar sevindirdiniz. Nasıl size teşekkür edeyim bilmem.\\
\textbf{Ka{[}ymakam{]}} - Nasıl mı? Bu akşam benimle beraber kulüpte yemek yiyerek..\\
\textbf{Seha} - (Mütereddid= Sizinle mi? Yalnız olarak mı?\\
\textbf{Ka{[}ymakam{]}} -Hayır Necdet Bey'le Pakize beraber canım.\\
\textbf{Seha} - Öyle ise ma'al-i-memnnuniye efendim gelirim. (Seha eldivenlerini çıkarır.)\\
\textbf{Ka{[}ymakam{]}} - Ha şöyle.. Teşekkür ederim.\\
\textbf{Seha} - Aman kaymakam bey şimdi Pakize Hanım'ın zevci Necdet Bey'i görmeyi ne kadar istiyorum.\\
\textbf{Ka{[}ymakam{]}} - Oo.. Pek nazik bir çocuk.. Göreceksiniz.. Hem de Onlar da sizin gibi mesut bir çift birbirleri için çıldırıyorlar.\\
Seha- (Kendi) Vay demek ki kocasını affetmiş..\\
\textbf{Ka{[}ymakam{]}} - (Cebinden bir not defteri çıkararak masada) Eyy şimdi Şemsi Bey'in künyesini söyleyin bakayım.\\
\textbf{Seha} - Ne gibi efendim.\\
\textbf{Ka{[}ymakam{]}} - Yani babasının ismi.. Nereli.. Hangi gemidedir?\\
\textbf{Seha} - Babasının ismi Nuri.\\
\textbf{Ka{[}ymakam{]}} - (Yazarak) Şemsi Nuri. İstanbul'lu değil mi?\\
\textbf{Seha} - Evet efendim.. Gemisi Trabzon'da, Aytaş'tan.\\
\textbf{Ka{[}ymakam{]}} - Aytaş'tan mı? Ooo kocanız da bizim deniz kurdu aşcı Yaşar'ın gemisindenmiş..\\
\textbf{Seha} - Aşcı Yaşar kim?\\
\textbf{Ka{[}ymakam{]}} - (Gülerek) Yaşar mı.. Bir antika. Bizim Pakize'nin süt kardeşi. Bu sabah Trabzon'dan gemisinden bir hafta izinle gelmiş!..\\
\textbf{Seha} - Aman ne iyi, ne saadet.. Bana Şemsi'ye dair havadis verir. (O sırada bir elinde kasket diğerinde baston Necdet görünür)\\
\textbf{Ka{[}ymakam{]}} - Hah işte kendisi. {[}İti an{]}, çomağı hazırla..\\

\hypertarget{besinci-meclis-1}{%
\section{Beşinci Meclis}\label{besinci-meclis-1}}

\begin{verbatim}
 [Evvelkiler- Necdet]
\end{verbatim}

\textbf{Necdet} -(Seha'ya bakarak) Bu kadın da kim?\\
\textbf{Ka{[}ymakam{]}} - Yaklaş bakalım Yaşar (Necdet birinci plana ilerler Seha oturur) Hanıma cevap ver.\\
\textbf{Seha} - Aşcıbaşı siz de Aytaş Gemisi'ndesiniz öyle mi?\\
\textbf{Necdet} - Evet hanımefendi.\\
\textbf{Seha} - Ey.. Amana çabuk söyleyin.. Şemsi Bey nasıl? Gemi arkadaşınız Şemsi iyidir İnşaallah\ldots{}\\
\textbf{Necdet} - (Sıkılmış bir halde) Al bakalım bir bela daha.\\
\textbf{Ka{[}ymakam{]}} - Ne duruyorsun? Suali anlamadın mı?\\
\textbf{Necdet} - evet kaymakam bey.. (Seha'ya) Şemsi midediniz?\\
\textbf{Seha} - Evet Şemsi Nuri. Şemsi Nuri.\\
\textbf{Necdet} - (Zihninde arar gibi) Şemsi Nuri..\\
\textbf{Seha} - Canım siyah bıyıklı.. Uzun saçlı.\\
\textbf{Necdet} - Yanlışınız var. Bizim gemide saçları uzun Nuri Bey yok.\\
\textbf{Seha} - Canım hani suvarisi ile arası açık.\\
\textbf{Ka{[}ymakam{]}} - Sizin suvarinin ismi ne bakayım.\\
\textbf{Necdet} - Bilmem.\\
\textbf{Ka{[}ymakam{]}} - (Kalkarak) Ne demek? Suvarinin, yüzbaşının, amirlerinin isimlerini bilmiyor musun?\\
\textbf{Necdet} - Hayır efendim. Hiç sormadum. Hem isimleri bir türlü aklımda tutamam.\\
\textbf{Ka{[}ymakam{]}} - Peki ama daha demin bana Havuzlar İdaresi'nin telefon numarasını söyledin. Hani hafızam kuvvetlidir diye övünüyordun?\\
\textbf{Necdet} - Evet rakamlar için! Fakat isimlere gelince mümkün değil hatırında tutamam\\
\textbf{Ka{[}ymakam{]}} - Olur numara değil be..\\
\textbf{Seha} - Canım düşünün bakalım.. Elbette tanıyacaksınız. Bir geminin içinde olur da arkadaşınızı tanımaz olur musunuz?\\
\textbf{Necdet} - Oo tabii görsem tanırım. Görür görmez ``Vay sensin ha!'' der boynun asarılırım ama ismine gelince..\\
\textbf{Ka{[}ymakam{]}} - Bırakın şunu hanımefendi.. Selamın aleyküm salak\\
\textbf{Necdet} -Aleyküm-es-selam efendim.\\
\textbf{Ka{[}ymakam{]}} - Kır dümeni bakayım salak. Haydi sen mutfağa marş..\\
\textbf{Necdet} - Başüstüne kaymakam bey. (Çıkar)\\

\hypertarget{altinci-meclis-1}{%
\section{Altıncı Meclis}\label{altinci-meclis-1}}

\begin{verbatim}
 [Kaymakam- Seha- sonra Şemsi]
\end{verbatim}

\textbf{Ka{[}ymakam{]}} - Şimdi müsadenizle ben karşıya kadar gidip geleceğim. Hep hep iki saat ya sürer yasürmez. Kusuruma bakmazsınız değil mi?\\
\textbf{Seha} - Rica ederim kaymakam bey. Buyurun tekrar buyurun teşekkür ederim..\\
\textbf{Ka{[}ymakam{]}} - Aman canım bir hiç için mi?..(Şemsi girer)\\
\textbf{Şemsi} - (Girerek) Her tarafta birisi var meydanı boş bulamıyorum ki sivişayim..\\
\textbf{Ka{[}ymakam{]}} - (Onu görerek) hah işte damat. (Seha'ya) Onu size takdim edeyim.\\
\textbf{Şemsi} - (Karısını görünce taş kesilir) Karım!çç Bozmamalı yoksa yandım. (Bütün cesaretini toplayarak durur)\\
\textbf{Ka{[}ymakam{]}} - (Takdim ile) Necdet Bey.. Seha Şemsi Hanım..\\
\textbf{Şemsi} - (Cesaretle hiç bozmayarak) Şerefyabım hanımefendi..\\
\textbf{Seha} - (Birdenbire hayretle haykırarak) Ah!..\\
\textbf{Şemsi} - (Aynı zamanda) Eyvah..\\
\textbf{Ka{[}ymakam{]}} - Neniz var Seha Hanım?\\
\textbf{Seha} - Siz Necdet Bey misiniz?\\
\textbf{Şemsi} - Evet efendim Necdet İbrahim..\\
\textbf{Ka{[}ymakam{]}} - Altı seneden beri hemşire-zadem Pakize'nin kocası..\\
\textbf{Seha} - (Kendi) Ne benzerlik Yarabbi.. Aynı ses.. Aynı gözler.. Müthiş şey..\\
\textbf{Şemsi} - (Kaymakama) Hanıma ne oldu birdenbire..\\
\textbf{Seha} - (Kaymakama) Beni mazur görün kaymakam bey fakat bayılacağım. Yarabbi Necdet Bey o kadar fevkalade bir surette kocam Şemsi'ye benziyor ki\\
\textbf{Ka{[}ymakam{]}} - Yok canım.\\
\textbf{Şemsi} - Acaip..\\
\textbf{Seha} - Yani o kadar ki eğer yüzünüzdeki yara izi olmasa ve bıyığınız da olsa sizin için Şemsi diye yemin ederim.\\
\textbf{Ka{[}ymakam{]}} - Bu derece ha?..\\
\textbf{Şemsi} - (Kendi) Aman yüzümün yarasını mektupta yazmadığım işe yaramış, hele bıyıkları traş etmem..\\
\textbf{Seha} - Aman Yarabbi şimdi çıldıracağım. Sizi temin ederim ki pek mütehayyicim. eğer sizi burada kaymakam bey vasıtasıyla tanımamış olsaydım, başka yerde tesadüf etmiş olsaydım bilmem..\\
\textbf{Ka{[}ymakam{]}} - (Neşeli) Şüpheniz olmasın Seha Hanım.. Necdet Bey Pakize'nin kocası damadım Necdet Bey'dir.\\
\textbf{Seha} - Sanki Pakize ile ben birbirine benzeyen ikiz kardeşlerle evlenmişiz gibi bu derece..\\
\textbf{Şemsi} - Şimdi ben söyleyecektim.\\
\textbf{Ka{[}ymakam{]}} - Bereket versin ki biri bıyıklı biri bıyıksız, birini de yüzünde iz var. Yoksa birbirinizin kocasına sahip çıkacaktınız . Sonra düşünün netice neye dönerdi.\\
\textbf{Şemsi} - (Gülerek) Şimdi ben söyleyecektim.\\
\textbf{Seha} -Aman Pakize bu işe ne kadar şaşacaktır. aman sakın sizi işinizden alıkoymayayım kaymakam bey. Bir yere gidecektiniz galiba..\\
\textbf{Ka{[}ymakam{]}} - Sahi bana hatırlattınız. Damadım, bana bahçe kapısına kadar refakat eder misiniz? Size söyleyeceklerim var.\\
\textbf{Şemsi} - Nasıl dayı bey.\\
\textbf{Ka{[}ymakam{]}} - Size birşey diyeceğim. Gelin benimle beraber. Tekra itizar ederim hanımefendi. Akşama kadar Allahaısmarladık. (Şemsi'ye) Buyurun damadım geçin önden.\\
\textbf{Şemsi} - Aman dayı bey! Siz buyurun..\\
\textbf{Ka{[}ymakam{]}} - Haydi canım bu beni biraz daha gençleştirir.\\
\textbf{Şemsi} - (Gülerek) İyi ama beni de ihtiyarlatır. (Kendi) Onu şimdi satar gelirim (Önden çıkar)\\
\textbf{Ka{[}ymakam{]}} - (Çıkmadan evvel arkasına Seha'ya bakar içini çekerek) Ah ne yazık kocasına tapıyor. Ne ise adam sen de.. (Çıkar)\\

\hypertarget{yedinci-meclis-1}{%
\section{Yedinci Meclis}\label{yedinci-meclis-1}}

\begin{verbatim}
 [Seha- Pakize- Necdet]
\end{verbatim}

\textbf{Pakize} - (Girerek kendi) Kardeş gibi onun da benden alacağı olsun. (Seha'yı görerek) Geldin mi..\\
\textbf{Seha} - Evet vapura vakit çok varmış. Heybeli'ye gitmekten vazgeçtim. Kendi kendime yarın dönüşte uğrarım dedim geldim.\\
\textbf{Pakize} - İyi ettin Seha'cığım.\\
\textbf{Seha} - Aman Pakize'ciğim.. Şimdi beni dinle. San şaşılacak bir havadisim var. Sen gelmeden bir dakika evvel zevcinle tanıştım. Aman kardeşim Necdet Bey tıpkı tıpkısına kocama benziyor. Ama nasıl ikiz kardeş gibi.\\
\textbf{Pakize} - İkiz kardeş mi?\\
\textbf{Seha} - Evet bir elmanın yarısı gibi.. Aynı ses, aynı hal, aynı incelik..\\
\textbf{Pakize} - Oo.. İncelik mi?\\
\textbf{Seha} - Ne dedin?\\
\textbf{Pakize} -Hayır anlamadım cicim.. Ben Şemsi Bey'in inceliğine dair birşey söylemedim. Çünkü tanımıyorum. Fakat doğrusu kendi kocamın inceliğinden birşey anlamadım. (Necdet elinde bardakları muhtevi tepsi ile görünür)\\
\textbf{Necdet} - Allah kurtarsın bu meslekten..\\
\textbf{Pakize} - (Necdet'i Seha'ya göstererk) Sen şimdi bunu ince ve zarif mi buluyorsun?\\
\textbf{Seha} - Canım ben deniz kurdundan bahsetmiyorum.\\
\textbf{Pakize} - Deniz kurdu mu?\\
\textbf{Seha} -(Necdet'i göstererek) Demin dayın süt kardeşin için deniz kurdu diyordu..\\
\textbf{Pakize} - Seha'cığım işte benim kocam.\\
\textbf{Seha} - Nasıl.. Siz mi?\\
\textbf{Necdet} - Evet ya..\\
\textbf{Seha} - Ya.. Öyleyse deminki kabalığımdan dolayı beni affedin.\\
\textbf{Necdet} - Ne beis var efendim.. Sabahtan beri yüklendiğim şeylerin yanında bu hiç kalır..\\
\textbf{Seha} - Peki ama dayının bana takdim ettiği Necdet Bey kimdi?\\
\textbf{Pakize} - Hah işte o da süt kardeşim Yaşar. Dayım buraya geldiği zaman bizi Yaşar'la öpüşürken gördüğü için kocam zannediyor..\\
\textbf{Seha} - Demek zevcin Necdet Bey de süt kardeşinin rolünü ifa ediyor öyle mi?\\
\textbf{Pakize} - evet dayımın nazarında saadetimi kurtarmak için.\\
\textbf{Necdet} - (kendi) Ah beşyüzbin liranın hatırı olmasa ben bilirim ya!\\
\textbf{Seha} - (Birden bir şüphe ile kendi kendine) Ah aman Yarabbi acaba?\\
\textbf{Necdet} - (Kendisini dinlemeyen Seha'ya) Size yemin ederim ki hanımefendi ben karımı aldattım ama.. Bu sırf bir kaza eseri idi. Daha doğrusu.. Mesleğimin haysiyetini kurtarmak için..\\
\textbf{Pakize} - Siz hemen lakırdıya başlamayın.. Elinizdeki işi bitirin de mutfağa bulaşıklarınıza koşun.\\
\textbf{Necdet} - Pakize ben artık bu aşcılıktan bıktım ama..\\
\textbf{Pakize} - Haydi diyorum.. Şimdi dayım nerede ise meydana çıkar.\\
\textbf{Necdet} - (Elini önlüğüne silerek) Pakala (Kendi) Aman be nedir bu çektiğim.. Beşyüzbin lirası da onun olsun..\\

(Çıkar)

\hypertarget{sekizinci-meclis-1}{%
\section{Sekizinci Meclis}\label{sekizinci-meclis-1}}

\begin{verbatim}
 [Evvelkiler- sonra Şemsi]
\end{verbatim}

\textbf{Seha} - (Telaş içinde) Peki ama Pakize'ciğim bu süt kardeşin benim midemi bozuyor.\\
\textbf{Pakize} - İşte ben kocamdan intikam elmek için onu intihap etmiştim.\\
\textbf{Seha} - Onu mu?\\
\textbf{Pakize} - Fakat beyefendiye ne oldu bilmem.. İlk önce müsait gibiyken birdenbire çekindi, birtakım vesveselere düştü.\\
\textbf{Seha} - Ya.. Daha önce onun yanında benim geldiğimi ağızından kaçırdın mıydı?\\
\textbf{Pakize} - Evet yemekte dayıma söylerken işitmişti..\\
\textbf{Seha} - Hah peki..\\
\textbf{Pakize} - Ne oluyorsun kuzum?\\
\textbf{Seha} - Hiç hiç.. Söyle bakayım.. Sen bu Yaşar'ı tanıyalı ne kadar oluyor?\\
\textbf{Pakize} - İlk defa olarak bugün görüyorum ayol..\\
\textbf{Seha} - Geldiği zaman nasıldı?\\
\textbf{Pakize} - Bahriye esvabıyla ama pis mundar bir halde, ince kumral bıyıkları vardı.. Burada traş etti.\\
\textbf{Seha} - Hah! Bıyıkları vardı değil mi\ldots{} Burada traş ettirdi. Aman Yarabbi bıyıkları varmış odur..\\
\textbf{Pakize} - Canım ne oluyorsun.. Söylecek misin?\\
\textbf{Seha} - Ne mi oluyorum? Kardeşim zannederim ki ikimiz de çok saf kadınlarız. Daha doğrusu biz bir çift aptalız..\\
\textbf{Pakize} - Aptal mı, niye?\\
\textbf{Seha} - evet.. Kalbi çok iyi, çok saf kadınlara aptaldan başka ne denir? Demek ben Fatih'te köşeciğimde kocacığımı beklerken o buralarda seninle aşıkdaşlık etmekte imiş..\\
\textbf{Pakize} - Kocanı mı? Anlamıyorum Seha'cığım?\\
\textbf{Seha} - Bunda anlaşılmayacak birşey yok kardeşim..Senin süt kardeşin benim kocam..\\
\textbf{Pakize} - Yok canım kabil değil..\\
\textbf{Seha} - (İsyan içinde) Ya\ldots Alacağın olsun.. Çapkın seni..\\
\textbf{Pakize} - Yok canım.. Fakat Seha'cığım onun ismi Yaşar.\\
\textbf{Seha} - Ne safsın Pakize. Sanki kendisine istediği ismi uyduramaz mı?\\
\textbf{Pakize} - Sahi öyle ama..\\
\textbf{Seha} - İlahi kardeşim bu erkekler hıyanete başlayınca neler düşündürmezler ki?\\
\textbf{Pakize} - Öyle ama Seha'cığım erkekler.. Birbirine aynen benzeyen iki adam olamaz mı? Unuttun mu geçen sene Bursa'dan gelen ikiz Cemal Bey vak'alarını. Bütün İstanbul'u ne kadar işgal etmişti.\\
\textbf{Seha} - Evet ama onlar ikiz imiş..\\
\textbf{Pakize} - Doğru.. Ama kocan başkasının ismini takınıp da ta Trabzon'dan buraya kadar gelmek zahmetine niçin katlansın? Evvelce beni tanımazdı..\\
\textbf{Seha} - Bu da doğru.. Sonra anlamadığım bir iki nokta daha var. Bu esrarengiz işi mutlaka halletmeliyiz. Şimdi her şeyden evvel..\\
\textbf{Pakize} - Herşeyden evvel kendisiyle açıktan açığa görüşeceksin değil mi?\\
\textbf{Seha} - Yoo.. Hayır. Bu aptallık olur.. Ona karşı hileden başka silahımız yok. Onun karşısında ondan daha kurnaz davranmalı.. pakizeceğim bu tilkiyi tuzağa düşürmek için bana yardım eder misin?\\
\textbf{Pakize} - Tabii kardeşim.. aldatılmış kadınlar birbirine yardım etmelidir.\\
\textbf{Seha} - (Ellerini tutarak) Demin bilmeden birbirimizin rakibesi idik. Şimdi anlaşarak gel el ele verelim.\\
\textbf{Pakize} - Hay hay..İşte.. (El ele verirler)\\
\textbf{Seha} - Ha.. Çabuk bana kalem kağıt..\\
\textbf{Pakize} - Orada, masada! Ne yapacaksın? (Seha masaya oturur yazı yazar)\\
\textbf{Seha} - Şimdi ona Trabzon'a cevaplı bir telgraf çekeceğim.. Eğer gemide değilse telgraf iade edilir.\\
\textbf{Pakize} - Hay.. Bu suretle hakiakti anlarız.. Ver, postaya göndereyim. ~\\
(telgrafı alır)

\hypertarget{dokuzuncu-meclis-1}{%
\section{Dokuzuncu Meclis}\label{dokuzuncu-meclis-1}}

\begin{verbatim}
 [Evvelkiler- Şemsi]
\end{verbatim}

\textbf{Şemsi} - (Girerken kendi) Beraberler vaziyeti kurtarmalı.\\
\textbf{Seha} - Şemsi.. (Kendi) Dur sen.\\
\textbf{Pakize} - Giriniz kardeşim giriniz.. Ben Seha Hanım'a herşeyi söyledim.
\textbf{Şemsi} - Şu halde hanımefendi, size zevcinizden bahsedebilirim. emsi Bey'den dört gün evvel ayrıldım.\\
\textbf{Seha} - Ya demek ki zevcimle tanışıyorsunuz.\\
\textbf{Şemsi} - Tanımaz olur muyum.. Şemsi Bey'le kardeş gibiyizdir. Hatta birbirimize olan fevkalade müşabehetten {[}benzerlik/benzeyişten{]} dolayı herkes de bizi hakiki ikiz kardeş zanneder\\
\textbf{Seha} - Demek Şemsi ile birbirinize bu derece benzersiniz..\\
\textbf{Şemsi} - Ama ne derece hanımefendi tasavvur edemezsiniz. Eğer benim yüzümde şu gördüğünüz iz olmasa herkez beni Şemsi, Şemsi'yi de ben zannedecek.\\
\textbf{Seha} - Ya.. Şayan-ı hayret.. Hakikaten Şemsi'nin yüzünde böyle bir şey yoktur.\\
\textbf{Şemsi} - Değil mi ya.. Bereket versin bu yara izine.. Bunun sayesinde çabuk ayırd ediliriz.\\
\textbf{Seha} - Evet, sonra Şemsi'nin küçük bıyıkları vardır.\\
\textbf{Şemsi} - Şemsi de kestirecek zannederim. Zira buraya gelirken bundan bahsediyordu. Ah görseniz Şemsi ne cevherli ne candan bir arkadaştır, gemide herkes onu canı gibi sever..\\
\textbf{Seha} - Ya suvarisi?..\\
\textbf{Şemsi} - (Süratle) Ha.. Suvariden maada.. Nedense suvari ile bir türlü bağdaşamadılar. Görseniz o da ne ters adamdır.. Fakat size de perestiş ediyor ha..\\
\textbf{Seha} - A.. Suvari mi?\\
\textbf{Şemsi} - Yok canım.. Oo ne diyorsunuz? Şemsi, zavallı Şemsi size perestiş ediyor. Gece gündüz bana sizden bahseder durur.. Sizi dilinden düşürmez..\\
\textbf{Seha} - Evet evet (Ayağa kalkarak) Şemsi.. Affedersiniz Yaşar Bey belki beni bir parça hoppa bulacaksınız ama zevcim bana hiçbir zaman askerlik tezkeresini göstermedi ben de bir asker tezkeresi görmeyi arzu ediyorum.\\
\textbf{Şemsi} - (Dişlerinin arasından) Dişi tilki seni!!.\\
\textbf{Seha} - Eğer mümkünse bana lütfen tezkerenizi..\\
\textbf{Şemsi} - Hay hay efendim.. Hay hay niye mümkün olmasın.. (Cebinden tezkeresini çıkararak) İşte efendim benimki\\
\textbf{Seha} - (Süratle tezkereyi alır Pakize ile beraberce bakarlar.)\\
\textbf{Pakize} - (Gizlice) Görüyor musun tezkeresi de var.\\
\textbf{Seha} - (Okuyarak) Halil Ömer oğlu Mehmet Yaşar\\

\textbf{Şemsi} - (Tebessümle) Ada 17 tevellüdü\\
\textbf{Seha} - (Pek ziyade mütehayyir kendi) Ada 17 tevellüdlü.. Yarabbi çok gariptir.\\
\textbf{Şemsi} - (Kendi) Yelkenler suya indi mi hanımefendi.\\
\textbf{Seha} - (Tezkereyi muayene ile) Çok tuhaf..\\
\textbf{Şemsi} - Oo orası öyle çok tuhaftır.\\
\textbf{Pakize} - (Yavaşca Seha'ya) Şimdi bunun üstüne yine telgraf çekmek lazım mı?\\
\textbf{Seha} - Tabii, tabii, ne olursa olsun!\\
\textbf{Şemsi} - (Kendi) Yavaş sesle konuşyorlar. Beni birşey anlamıyor zannediyorlar.\\
\textbf{Pakize} - Ben gidip senin odanla meşgul olayım kardeşim..\\
\textbf{Seha} - Sen bilirsin cicim.. Peki git. (Elindeki tezkereyi ortadaki masaya koymak üzere geçerken) Hiç ötesi yok çıldıracağım. Tezkerede Yaşar yazılı (Elindeki tezkereyi bırakır. Şemsi bunu görmez. O çıkan Pakize'ye bakmaktadır)\\
\textbf{Pakize} - Sizi Seha Hanım'la bir kaç dakika yalnız bırakıyorum süt kardeşim.\\
\textbf{Şemsi} - Hay hay süt kardeşim.. Güle güle..\\
\textbf{Seha} - Bunda mutlaka bir iş var..\\
\textbf{Pakize} - Acaba Seha aldanıyor mu?\\
(Çıkar)\\

\hypertarget{onuncu-meclis}{%
\section{Onuncu Meclis}\label{onuncu-meclis}}

\begin{verbatim}
 [**Şemsi**  - Seha]
\end{verbatim}

\textbf{Şemsi} - (Çıkan Pakize'ye bakarak) hepsi birbirinin eşi.. Kocalar?\\
\textbf{Seha} - Eyy.. Yaşar Bey.. Demek ki sonradan çekindiğiniz birtakım kuruntulara düştünüz öyle mi?\\
\textbf{Şemsi} - Ne gibi efendim anlayamadım?\\
\textbf{Şemsi} - Süt kardeşiniz hakkında..\\
\textbf{Şemsi} - Ha.. Size bunu da mı söyledi?\\
\textbf{Pakize} - Evet.. Bizim Pakize ile gizli birşeyimiz yoktur..\\
\textbf{Şemsi} - Vallahi bu meselede, hanımefendi ben Pakize hanımı hiddet ve teessürle yapılan bir işin sebebiyet vereceği vicdan azabından vikaye etmek {[}korumak{]} istedim. Kendileri bugün bana darıldılar.. Fakat ya istediği olsaydı yarın daha ziyade darılmayacakları ne malumdu {[}?{]}\\
\textbf{Seha} - (Müstehzi) Oo.. Tebrik derim. Sizin gibi bir buçuk senedir deniz üstünde gezen bir asker için bu haber doğrusu bir liyakat madalyasına değer.\\
\textbf{Şemsi} - (Müstehzi) Teşekkür ederim hanımefendi. Beni bu madalyaya layık görüyorsanız bunu
bizzat istemem muvafık olmayacağından bu hususta lazım gelen teşebbüste bulunmanızı rica ederim. (Birdenbire şivesini değiştirerek) Şimdi eğer kıymetli arkadaşım eşimŞemsi için bir yapılacak bir siparişiniz varsa emrinize amadeyim hanımefendi..\\
\textbf{Seha} - Çok , çok teşekkür ederim. Acelesi ne? Bundan bahsetmek için daha vaktimiz var. Sizin daha beş gün müsaadeniz var.. Ben de birkaç gün buradayım..\\
\textbf{Şemsi} - Oo.. Hayır! Şüphesiz takdir edersiniz ki Pakize Hanım'la aramızda geçen bu hissi kazadan sonra benim burada daha ziyade durmam doğru olamaz.\\
\textbf{Seha} - Nasıl, gitmek mi istiyorsunuz?\\
\textbf{Şemsi} - Tabii değil mi? Bizi karı koca zanneden Kaymakam Gazanfer Kaptan buradan gider gitmez derhal ben de ayrılacağım.\\
\textbf{Seha} - (Gizli) Aman ne yapsam !!!\\
\textbf{Şemsi} - (Lakayd bir tavırla) Mütebaki beş gün izinimi gidip Beykoz'da!\\
\textbf{Seha} - Beykoz'da mı?\\
\textbf{Şemsi} - Evet tanıdıklarımın birinde geçireceğim.\\
(Bir vakfe)

\textbf{Seha} - (Gizli) Ah Yarabbi ne yapsam..\\
\textbf{Şemsi} - (Masum bir tavırla) Yanılmıyorsam sizi bu haber sinirlendirdi gibi\\
\textbf{Seha} - (Telaşla) A.. Hayır hayır niçin sinirlendirecek. Ben Pakize değilim.Bu evdeki mevcudiyetinizin sizin için ne kadar tehlikeli olduğunu pek iyi takdir ederim.\\
\textbf{Şemsi} - Tabii.. Tabii.. Tabii.. Müsaade ederseniz elbisemi değiştirip torbamı hazırlamaya gidiyorum.\\
\textbf{Seha} - Buyurun Yaşar bey.\\
\textbf{Şemsi} - (Gizli( Aman allah'ım.. Şuradan palamarları bir çözebilsem.\\
(Çıkar)

\hypertarget{on-birinci-meclis}{%
\section{On Birinci Meclis}\label{on-birinci-meclis}}

\begin{verbatim}
 [Kaymakam-Seha]
\end{verbatim}

\textbf{Seha} - (Mütehavvir {[}hiddetten gözü dönmüş olarak{]} ayağa kalkarak) Bir de utanmadan benimle alay ediyor gibi bir hali var.. Beykoz'a gidecekmiş.. Ne yapsam da mani olsam.(Masanın üzerindeki tezkereyi görerek) Hah evvela şunu tezkarasini saklayalım (alır göğsüne saklar)\\
\textbf{Ka{[}ymakam{]}} - (Girerek) A.. Burada mısınız muhterem Seha Hanımefendi. Size şunu ifade edeyim ki bana uğur getirdiniz..\\
\textbf{Seha} - Ne gibi efendim?\\
\textbf{Ka{[}ymakam{]}} - Arz edeyim.. Şimdi telefonda Havuzlar İdaresi'nde bir arkadaşımla konuştum. Ankara'ya davetimin sebebini söyledi: Vekaletten hatırı sayılır bir memuriyet teklif edeceklermiş. Bunu tebşir ettiler {[}müjdelediler{]}. Eczahane'nin telefon numarasını bıraktım. Biraz sonra bana memuriyetin ne olduğunu söyleyecekler.\\
\textbf{Seha} - Ben de tebrik ederim efendim.\\
\textbf{Ka{[}ymakam{]}} - Binaenalyh şimdi artık iki üç gün sonra da gitsem olacak.\\
\textbf{Seha} - İki üç gün sonra mı gideceksiniz? Yani burada mı kalacaksınız? Hemen bu akşam gitmeyecek misiniz?\\
\textbf{Ka{[}ymakam{]}} - Tabii.. -Hayır..\\
\textbf{Seha} - (Sevinçle) Ah.. Kaymakam bey..\\
\textbf{Ka{[}ymakam{]}} - Beyoğlu'na da telefon ettim. Pera Palas'ta bir odacık bulabildim.. Kasımpaşa'ya yakın olabilmek için başka çare yok.\\
\textbf{Seha} - (Telaşla) Oo.. Yok kaymakam bey Pera Palas mera palas yok.. Burada kalacaksınız.\\
\textbf{Ka{[}ymakam{]}} - Vallahi emredersiniz hanımefendi ya.. Hani Pakize'nin de evde hizmetçisi kalmadı.. Çocuklara yük olmayayım diyorum.\\
\textbf{Seha} - Efendim siz bunu düşünmeyin. Bunun çaresine bakılır..\\
\textbf{Ka{[}ymakam{]}} - Peki ama .. Civanım..\\
\textbf{Seha} - (İşvebaz) Benimle beraber burada bir çatı altında kalmaktan niye çekiniyorsunuz, kaymakam bey. Topu topu üç dört gün beraber geçireceğiz. Bu hatırayı çok mu görüyorsunuz?\\
\textbf{Ka{[}ymakam{]}} - Yok.. Onun için değil.. Ah Seha Hanım bakmayın bana öyle..\\
\textbf{Seha} - (Kararını vererek) Kaymakam bey çimdi karşınızdaki kadın vapurda tesadüfen karşısına oturduğunuz kadın değil.. Vapurdaki kadın o vakit kocasının sadakatine inanıyordu.. Şimdi karşınızdaki kadınsa hiç öyle değil..\\
\textbf{Ka{[}ymakam{]}} - Ne yoksa kocanız Şemsi Bey sizi aldatıyor mu?\\
\textbf{Seha} - Ya fena halde yanılıyorum yahut ben küçük beyi Fatih'de beklerken o beni Beykoz'da Boğaziçi'nde aldatıyor..\\
\textbf{Ka{[}ymakam{]}} - (Birşey anlamayarak hazin) Ya!..\\
\textbf{Seha} - Herhalde buna emin olduğum dakika kendimi sizin kollarınızın arasına atacağım.\\
\textbf{Ka{[}ymakam{]}} - Ne? Aman.. Dile benden ne dilersen.. Emrinize amadeyim.\\
\textbf{Seha} - Hayır kaymakam bey. Ben sizin emrinize amadeyim.\\
\textbf{Ka{[}ymakam{]}} - Ah Seha Hanım siz bana pusulamı şaşırttınız..\\
\textbf{Seha} -Peki peki.. Ben gidip Pakize ile odanızı hazırlayacağım. Pera Palas'a telefon etmeyi unutmayın. (Çıkarken yalnız) Şimdi kaç bakalım Şemsi Bey!!!\\

\hypertarget{on-ikinci-meclis-1}{%
\section{On İkinci Meclis}\label{on-ikinci-meclis-1}}

\begin{verbatim}
 [Kaymakam- sonra Şemsi]
\end{verbatim}

\textbf{Ka{[}ymakam{]}} - (Yalnız) Benim benim olacak.. Kendini buraya kollarımın arasına atacak ha..\\
\textbf{Şemsi} - (Elinde torbasıyla girerek kendi) Torban hazır ama esvaplarımı bulamadım.\\
\textbf{Ka{[}ymakam{]}} - Hah damadım sevgili damadım yedi kat göğt havalanmış bir adam görmek istersen bana bak\\
\textbf{Şemsi} - Hayırdır İnşaallah dayı bey iyi bir haber mi aldınız?\\
\textbf{Ka{[}ymakam{]}} - Müjdelerin büyüğü Necdet Bey (Yavaşca sevinç içinde) Benim olacak benim. Kendini buraya kollarımın arasına atacak..\\
\textbf{Şemsi} - Aman aman sahi mi kim dayı bey?\\
\textbf{Ka{[}ymakam{]}} - Seha Hanım..\\
\textbf{Şemsi} - (Sıçrayarak) Kim dediniz..\\
\textbf{Ka{[}ymakam{]}} - Seha Şemsi Hanım canım..\\
\textbf{Şemsi} - Yok canım!\\
\textbf{Ka{[}ymakam{]}} - Ya.. O zavallı Fatih'de kocasını beklerken kocası onu Beykoz'da aldatıyormuş. Herhalde buna emin olduğu dakika kendisini kollarımın arasına atacakmış.\\
\textbf{Şemsi} - (Gizli) Herhalde bu dakikayı bulamayacak.\\
\textbf{Ka{[}ymakam{]}} - Ha.. Sonra bu akşam da gitmiyorum. Daha üç dört gün sizdeyim..\\
\textbf{Şemsi} - (Reçeteymiş gibi) Üç dört gün mü?\\
\textbf{Ka{[}ymakam{]}} - Pera Palas'a telefon edeceğim {[}`Aman Minnoş' şarkısını söyleyerek çıkar)\\

\hypertarget{on-ucuncu-meclis-1}{%
\section{On Üçüncü Meclis}\label{on-ucuncu-meclis-1}}

\begin{verbatim}
 [Şemsi-Necdet- sonra Yaşar]
\end{verbatim}

\textbf{Şemsi} - (Yalnız şaşkın deli gibi) Hay lanet olasıca iş be.. Şuradan kurtulamıyorum bir türlü.. Seha beni aldatsın ha.. (Azim ve şiddetle kalkarak) Hayır aldatamayacak.. Çünkü benim onu alattığıma emin olamayacak (O sırada Seha'nın yazdığı telgrafname elinde giren Necdet'e) Necdet azizim beni dinle..\\
\textbf{Necdet} - Birazdan söylersin Yaşar'cığım.. Postahaneye kadar gitmeye mecburum.\\
\textbf{Şemsi} - Postahaneye mi?\\
\textbf{Necdet} - Evet Seha Hanım'ın bir telgrafı varmış da..\\
\textbf{Şemsi} - (Gizli) Karımın mı? Kime? (Aşikar) Bakayım (Süratle telgrafı kapar)\\
\textbf{Necdet} - Dur canım ne yapıyorsun.\\
\textbf{Şemsi} - (Okuyarak) "Ada'da MAden'de 55 numrolu Necdet Bey'in hanesindeyim. Derhal sağlık haberini bildir.. (Cevaplı da)\\
\textbf{Necdet} - Şişt ver şunu canım gideceğim..\\
\textbf{Şemsi} - (Cebine koyarak) Telgraf yerine vardı\ldots{}\\
\textbf{Necdet} - Nereye vardı?\\
\textbf{Şemsi} - Bırak. Sorma, derinine inme içinde kaybolursun. Yalnız sana sorulursa telgrafı gönderdim dersin.. Sana ne?\\
\textbf{Necdet} - Öyle ya bana ne.. Bir angarya eksik oldun değil mi?\\
\textbf{Şemsi} - Değil mi? (Kapı zili)\\
\textbf{Şemsi} - (Gizli) Oh.. Bu sağanak da geçti.. Yakayı kurtardım.\\
\textbf{Necdet} - (Bahçeye bakarak kendi) Vay bir bahriyeli!!\\
\textbf{Şemsi} - (Düşünceli) Ne yapmalı, ne yapmalı..\\
\textbf{Necdet} - (Camlı kapıyı aralayarak dışarı seslenir) Parmaklıklı kapıyı itin içeriye girin..\\
\textbf{Şemsi} - (Kendi) Gemide bir arkaşıma buraya benim namıma cevap vermesini telgrafla bildirdim.\\
\textbf{Necdet} - (Yine camlı kapıdan dışarıya) Kimi istiyorsunuz?\\
\textbf{Yaşar} - (Dışarıdan bir ses) Ben Yaşar'la görüşmek istiyorum\ldots{}\\
\textbf{Şemsi} - (Sesi duyarak yalnız) Nasıl bu ses?\\
\textbf{Necdet} - (Yine dışarıya) Bu tarafa bu tarafa )Oradan çekilerek Şemsi'ye) Bak Yaşar seni bir bahriyeli arıyor.\\
\textbf{Şemsi} - (Sıçrayıp bakar. Camlı kapının arkasında) Yaşar! Aman Yarabbi Yaşar geldi. İşte şimdi tamamız..\\
Yaşar- (Girerek) Nasıl bravo mu bana, seni buldum ya..\\
\textbf{Şemsi} - (Necdet'in üzerine gider ve hemen onu itmeye başlar) Bizi yalnız bırak..\\
\textbf{Necdet} - Dur ne istiyorsun.\\
\textbf{Şemsi} - Git diyorum. Haydi.. Uç.. Çabuk.. (İterek Necdet'i çıkartır)\\

\hypertarget{on-dorduncu-meclis-1}{%
\section{On Dördüncü Meclis}\label{on-dorduncu-meclis-1}}

\begin{verbatim}
 [Yaşar - Şemsi]
\end{verbatim}

\textbf{Yaşar} - Vay sen bıyıkları kazımışsın.. Ooo hem bu ne şıklık, bu ne tuvalet vay babam vay..\\
\textbf{Şemsi} - Yaşar.. Sen simdi bu gevezelikleri bırak. Buraya ne yapmaya gelsin bakayım? Onu söyle..\\
\textbf{Yaşar} - Dur be.. Şu terimi sileyim.\\
\textbf{Şemsi} - (Sabırsızlık içinde) Peki haydi bakalım.. Şimdi Yaşar söylesene {[}?{]} ne yapmaya geldin?\\
\textbf{Yaşar} - Ne yapmaya olacak.. Seni görmeye. Bilirsin ya. Senden bir dakika ayrılmak istemem..\\
\textbf{Şemsi} - Bunu biliyorum. Teşekkür ederim yavrum {[}?{]} ama gemiden ayrılırken ne demiştik, unuttun mu?\\
\textbf{Yaşar} - Unutur muyum.. Unutur muyum.. Unutmadım ağabeyciğim. Ama ne yapayım.. Buraya nasıl olsa gelip kardeşimle görüşecek değil miyim?..\\
\textbf{Şemsi} - (Etrafına bakarak) Yavaş kardeşim.. Yavaş yavrum..\\
\textbf{Yaşar} - Neden be Şemsi'ciğim?\\
\textbf{Şemsi} - (Ağızını kapayarak) Ne yapıyorsun?\\
\textbf{Yaşar} - (Gülerek) Ha.. Affedersin Şemsi vallahi daldım.. Yoksa\\
\textbf{Şemsi} - (Gider bütün kapıları muayene eder) Buraya gel birşey zannedecekler. Şimdi çabuk söyle bakalım İstanbul'da nereye gideceksin? Ben seni gelir bulurum.\\
\textbf{Yaşar} - Olur ya.. Kartal'a gidip suvarinin emanetlerini bıraktım. Sonra Adalar'a su götüren bir motora atladım. (Gülerek) Bak iskelede başıma geleni anlatayım da sen de gül..\\
\textbf{Şemsi} - Şimdi hikayenin sırası mı Yaşar.\\
\textbf{Yaşar} - Yok yok dinle ama motor iskeleye rampa eder etmez ben haydi dışarıya herifler arkadan bağırır\\
\textbf{Şemsi} - Yaşar. Yaşar'cığım, iki gözüm.\\
\textbf{Yaşar} - Metelik vermedim, sıvıştım.\\
\textbf{Şemsi} -İyi yaptın, iyi yaptın..\\
\textbf{Yaşar} - Darılma ağabeyciğim..\\
\textbf{Şemsi} - Söylesene şimdi ne istiyorsun, paran mı kalmadı?\\
\textbf{Yaşar} - Yok be param var..\\
\textbf{Şemsi} - Ey ne öyle ise\ldots{}\\
\textbf{Yaşar} - Sana birşey soracağımda.. Senden birşey isteyeceğimde\ldots{}\\
\textbf{Yaşar} - (Yanına gider yılışık) Şemsi Ağabey'ciğim be..\\
\textbf{Şemsi} - Yahu benim ismim Yaşar değil mi Yaşar'cığım?\\
\textbf{Yaşar} - Yok canım bırak şimdi Şemsi'yi Yaşar'ı. Ben senden mühim bir iyilik isteyeceğim..\\
\textbf{Şemsi} - Canım söylesene, seni bekliyorum işte..\\
\textbf{Yaşar} - Peki ağabeyciğim (Yanına gider yılışık bir tarzda gülerek) Ben benim canım..\\
\textbf{Şemsi} - (Gülerek) Ey senin canının.. Söylesene be..\\
\textbf{Yaşar} - Benim canım eğlenti istiyor. Beni bir yere götürsene be..\\
\textbf{Şemsi} - Ne.. Nereye?..\\
\textbf{Yaşar} - Nereye mi? Ayağım karaya değdi be.. Anlamadın mı? Bir yere işte.. (Şemsi'nin kulağına gidip birşeyler söyler)\\
\textbf{Şemsi} - Ne?..\\
\textbf{Yaşar} - Darılma be ağabeyciğim..\\
\textbf{Şemsi} - Seni yaramaz seni.. Herşey bitti de şimdi onu mu düşünüyorsun..\\
\textbf{Yaşar} - Aman be ağabeyciğim.. Gözünü seveyim.. Vallahi halimyaman beni götür sen dön.. Ha.. Olur mu?\\
\textbf{Şemsi} - Peki olur, olur, daha başka ne istiyorsun?\\
\textbf{Yaşar} - Başka hiç mi hiç.. (Şemsi'ye sarılarak) Ah ağabeyciğim benim yaşa be.. Haydi öyleyse.. Hemen çıkalım yola..\\
\textbf{Şemsi} - Ne.. Hemen şimdi mi.. Sen kızdın mı oğlum biraz sabret.. Yarın öbür gün gideriz.\\
\textbf{Yaşar} - Yarın öbür gün mü?.. Ben yarın, öbür gün döneceğim..\\
\textbf{Şemsi} - Döneceğin mi {[}Dönecek misin{]} nereye?..\\
\textbf{Yaşar} - Nereye olacak, gemiye..\\
\textbf{Şemsi} - Gemiye mi? Neden?\\
\textbf{Yaşar} - Öyle ya ne yapayım İstanbul'da. Benim yapyalnız içim sıkılıyor.\\
\textbf{Şemsi} - Burada senin hiç tanıdığın, arkadaşın yok mu? Git onlarla konuş, eğlenirsin. onlar seni he yere götürürler..\\
\textbf{Yaşar} - Birkaç aşcı var ama.. Beyazıt'a kahveye gitmeli..\\
\textbf{Şemsi} - Bunun için mi gemide bir hafta izin diye el etek öptün? Şimdi de dönmek istiyorsun öyle mi?\\
\textbf{Yaşar} - Haa.. Ne yapayım burada? Yiyip içip nefsimi körlettikten sonra artık işim kalmaz.\\
\textbf{Şemsi} - Demek İstanbul'a geldiğine pişman oldun?\\
\textbf{Yaşar} - Buraya geldiğime pişman edip etmemek senin elinde ağabeyciğim (Bir vakfe)\\
\textbf{Şemsi} - Aman Yarabbi canım bir haftada insan sıkılır mı? ..\\
\textbf{Yaşar} - Belki bir hafta kalabilirim ama.. Sakın yine darılma ağabeyciğim\ldots{} Onsuz yapamayacağım..\\
\textbf{Şemsi} - Pekala.. Pekala.. Yaşar'cığım.. Sen şimdi gider beni aşağıdaki yolu dönerken bir büyük kahve var hani, orada beklersin. Ben yarım saate kadar gelirim, seni orada bulurum.. Beklersin anladın mı? (Yaşar düşünceli)\\
\textbf{Yaşar} - Anladım..\\
\textbf{Şemsi} - Neyi anladın?\\
\textbf{Yaşar} - Beni atlatacağını..\\
\textbf{Şemsi} - Yok.. Yok iki gözüm emin ol. Haydi sen dediğimi yap emi. haydi benim güzel Yaşar'cığım, aşcıbaşıların aşcıbaşısı Yaşar'cığım. Haydi..\\
\textbf{Yaşar} - (Mütevekkil çekilir) Peki ama kandı zannetme ha.. Bilmiş ol ki eğer yarım saate kadar gelmezsen yine gelir balta olurum..\\
\textbf{Şemsi} - Peki.. Peki.. Hadi şimdi uğurlar olsun..(Dışarı iterek çıkarmak ister)\\
\textbf{Yaşar} - (Birden aklına birşey gelmiş gibi dönerek) Ha.. Aman Şemsi Ağabey'ciğim..\\
\textbf{Şemsi} - Yine mi.. Hay Allah.. Unut Şemsi Ağabey'i..\\
\textbf{Yaşar} - Olur.. Olur.. Az kaldı unutuyordum.\\
\textbf{Şemsi} - Ne var? ..\\
\textbf{Yaşar} - Tezkerem, benim tezkerem.. İşi bittiyse artık ver bana onu..\\
\textbf{Şemsi} - Niçin yavrum? Unuttun mu gemiye girerken değiştirecek değil miydik?\\
\textbf{Yaşar} - Öyle ama bunun burası İstanbul kuzum.. Ada'dan tezkere soruyorlar. Ver benimkini ben de seninkini vereyim.\\
\textbf{Şemsi} - Benimki sende ya..\\
\textbf{Yaşar} - Olsun ben korkuyorum.. Hem ben gemiye senden evvel gideceğim. Onun için benim tezkerem benim olmalı.. Korkulu rüya görmektense..\\
\textbf{Şemsi} - Canım Yaşar'cığım şurada üç dört gün daha kaldı beraberce gidelim.. Ne olur..\\
\textbf{Yaşar} - Vallahi yapamam ağabeyciğim.Ben bir kerecikte şu süt kardeşimin yüzünü görsem artık bir işim kalmaz! Yarın akşam Selamet Vapuru kalkıyor. Aşcıbaşısı da tanıdık. Ben Selamet'le gideceğim. Sonra senin tezkerenle gemiye nasıl girebilirim? Haydi bakalım çık bizim tezkereyi..\\
\textbf{Şemsi} - Canım bir iki gün kal..\\
\textbf{Yaşar} - Kalamam.. Ağabeyciğim.. Kalamam.. Sen de gel de biraz gidelim ha ne olur? Sen şu tezkereyi unutmadan..\\
\textbf{Şemsi} - Bırak Allahaşkına şu inadı..\\
\textbf{Yaşar} - (alayla) Ha.. Sahi sen gelemezsin. Artık başın dumanlı değil mi? nasıl bizim süt hemşire ile mercimeği fırına verdiniz mi?\\
\textbf{Şemsi} - Sonra anlatırım.. Sonra.. Sonra.. Şimdi ben.\\
\textbf{Yaşar} - Ha.. Sen şu tezkereyi çık da unutmadan..\\
\textbf{Şemsi} - Demek mutlaka gideceksin öyle mi? (Canı sıkılmıştır)\\
\textbf{Yaşar} - Evet mutlaka..\\
\textbf{Şemsi} - (Seri ve titiz) Pekala ne olursa olsun al bakalım.. (Cebini arayarak) Eyvah.. Tezkere?\\
\textbf{Yaşar} - Ulan ağabey yoksa tezkeremi kayıp mı ettin?\\
\textbf{Şemsi} - Yok canım.. Merak etme.. Nerede olur. Esvabımı değiştirirken ceketimin cebine koymuştum ama (Kendi) Mutlaka Seha bir yere bırakmıştır. (Aramaya koyulur sonra birden şüphe ile) Ha~yoksa şeytan karı sakladı mı tezkereyi..\\
\textbf{Yaşar} - Bulamadın mı ağabey? Ne düşünüyorsun?\\
\textbf{Şemsi} - Yaşar'cığım ben onu şimdi bulurum. Mutlaka yukarıda asker esvabımın cebinde bırakmışımdır. Sen şimdi hiç durma git beni o dediğim kahvede bekle yarım saate kadar gelir seni bulurum. Ne olursa olsun..\\
\textbf{Yaşar} - Canım şurada oturur beklerim.. Hadi git getir be ağabey.. (Oturur)\\
\textbf{Şemsi} - Hay Allah senin galiba niyetin bozuk. Benim işimi bozmak istiyorsun..\\
\textbf{Yaşar} - Hangi işini..\\
\textbf{Şemsi} - Ben burada sen değil miyim? Seni burada görseler benim işim bozulmaz mı? Hani bana ne diyordun? Yirmi senedir görmediğim bir süt kardeşimi görüp de ne yapacağım. Benim onlara hıncım var.. Küçükken beni başlarından atmışlar.. Neme lazım.. Yüzlerini şeytan görsün, demiyor muydun?\\
\textbf{Yaşar} - Yine de öyle ya.. Ama merakımdan hani..\\
\textbf{Şemsi} - Adam sen de. Verdiğin sözden geri mi döneceksin? Benim gibi bir arkadaşını, ağabeyini ele mi vereceksin!\\
\textbf{Yaşar} - Yok canım ağabeyciğim.. Ben zaten İstanbul'a senin hatırın için geldim. Sen burada işini koymuşsun. Ben pişmiş aşa su katan aşcılardan değilim. İşte gidiyorum. Tezkeremi oraya getiresin ha.. Getireceksin değil mi ağabey..\\
\textbf{Şemsi} - Getireceğim dedim çocuğum haydi..\\
\textbf{Yaşar} - Yemin et.. Ağabey\\
\textbf{Şemsi} - Senin başına yemin ederim. Haydi bakalım şimdi..\\
\textbf{Yaşar} - Şemsi Ağabey kahve paralarım senden ha..\\
\textbf{Şemsi} - Hay, hay, (Yaşar çıkar)\\

\hypertarget{on-besinci-meclis-1}{%
\section{On Beşinci Meclis}\label{on-besinci-meclis-1}}

\begin{verbatim}
 [Şemsi-Kaymakam-Necdet- Seha- Pakize]
\end{verbatim}

\textbf{Şemsi} - Ya.. Seha beni böyşe kıskıvrak bağlarsın ha.. Pekala bakalım kim galip geliyor. Sakın bi,r yere düşmüş olamsın.. (Eğilir aramaya koyulur) Ama{[}n{]} tezkere göreyim seni. (Şeytan aldı götürdü satamadan getirdi)\\
\textbf{Ka{[}ymakam{]}} - (Girer) (Şemsi'yi o halde görür tanımaz. Kendi) Bu ne demek?\\
\textbf{Şemsi} - (Başı masanın altında) Şeytan aldı götürdü satamadan getirdi.. Nereye gider.. Şuraya koydumsu..\\
\textbf{Ka{[}ymakam{]}} - (Kendi) Fakat bu bizim damat..\\
\textbf{Şemsi} - (Kaymakamın girdiğini görerek kalkarken) Yok, yok, tezkere yok! (Kaymakamı görerek kendi) Vay canına, acaba duydu mu?\\
\textbf{Ka{[}ymakam{]}} - Damat tezkerenizi mi arıyorsunuz?\\
\textbf{Şemsi} - Ha evet efendim.. Demin iuralara bırakmıştım ama..\\
\textbf{Ka{[}ymakam{]}} - Durun siz e yardım edeyim.. Beraber arayalım..\\
\textbf{Şemsi} - (Telaşla) Aman efendim zahmet etmeyin. Nereye koyduğumu ben şimdi bulurum. (Necdet girer)\\
\textbf{Ka{[}ymakam{]}} - Pekala.. (Necdet'i görerek) Ha Yaşar buraya gel..\\
\textbf{Necdet} - (İlerlerken kendi) Efendim.. Bereket versin ki bu akşam gidiyor.\\
\textbf{Ka{[}ymakam{]}} - Bu akşam yemekte burada değiliz. Ucuz kurtuldun fakat yarın abah yemeğinde bana yine bugünkü gibi rezaletler yaparsan senin aşık kemiğini kırarım.. Bak sana bugünden haber veriyorum..\\
\textbf{Necdet} - (Şemsi'ye) nasıl yarın da mı gitmiyor?\\
\textbf{Şemsi} - Üç dört gün daha burada..\\
\textbf{Necdet} - (Kaymakama) Nasıl hani gidiyordunuz?\\
\textbf{Ka{[}ymakam{]}} - Ne dedin?\\
\textbf{Necdet} - Hayır yani benim için ne şeref ki.. Daha birkaç gün..\\
\textbf{Ka{[}ymakam{]}} - Yeter! Sen hakikaten çekilmez, ne söylediğini, ne istediğini bilmez bir herifmişsin ya.. (Şemsi'ye) Ona o kadar fazla yüz vermişsiniz ki..\\
\textbf{Şemsi} - Sizi temin ederim dayıcığım..\\
\textbf{Ka{[}ymakam{]}} - Pek fazla ma pek fazla.. (Necdet'e) Sen bulaşık yıkamaktan başka elinden bir şey gelmeyen bir herifsin.\\
\textbf{Necdet} - (Dişlerinin arasında) Ah benim elimden neler geliyor bilsen..\\
\textbf{Ka{[}ymakam{]}} - Çıkar bakayım tezkereni!\\
\textbf{Necdet} - (Süratle cebinden çıkararak) Buyurun kaymakam bey..\\
\textbf{Şemsi} - (Kendi) Ah beceriksiz (İşaret)\\
\textbf{Necdet} - (Yaptığının farkına vararak) Ah ne halt ettim..\\
\textbf{Ka{[}ymakam{]}} - (Tezkereye bir göz atarak hayretle) Oo bu ne?\\
\textbf{Necdet} -(yaptiginin farkinda olarak) aman ne halt etdin\\
\textbf{Ka{[}ymakam{]}} - (Okuyarak) Necdet İbrahim. Dersa'adet..\\
\textbf{Necdet} - (Kendi) Mahvoldum..\\
\textbf{Ka{[}ymakam{]}} - (Şemsi'ye) Ha işte tezkereniz damat.. Deminden beri arıyordunuz!\\
\textbf{Şemsi} - (Telaş ve Necdet'e işaretle) Nasıl Yaşar.. Sen mi bulmuştun tezkeremi?\\
\textbf{Necdet} - (Şaşkın) Evet evet demin burayı süpürürken..\\
\textbf{Ka{[}ymakam{]}} - (Tezkereyi Şemsi'ye vererek) Bulmuş sonrada cebinde unutmuş. Kafa değil lahana..\\
\textbf{Şemsi} - Bunak efendim.. Bu yaşta bunak\\
\textbf{Ka{[}ymakam{]}} - (Necdet'e) Ver seninkini..\\
\textbf{Necdet} - Benimki yok kaymakam bey.\\
\textbf{Ka{[}ymakam{]}} - Ne demek benimki yok?\\
\textbf{Necdet} - Yani üstümde değil, efendim.. Galiba aşağıda bir tencerenin içine bırakmıştım..\\
\textbf{Ka{[}ymakam{]}} - Aman Yarabbi tezkeresini tencerenin içinde bırakmış. Dünyada bu kadar salak bir gemici daha yoktur. Sen bir askerin tezkeresiz gezemeyeceğini bilmiyor musun?\\
\textbf{Necdet} - Hayır efendim.\\
\textbf{Ka{[}ymakam{]}} - Hayır efendim? (Şemsi'ye) Hakkın var damat bunak bir şey bilmeyen beyinsiz bunak!. Bana bak oğlum.. madem ki kafanda rakamdan başka bir şey yer tutmuyor..Aç kulağını iyi dinle: Mezunen burada bulunduğun müddet zarfında eğer en ufak bir münasebetsizliğini görecek olursam seni Kasımpaşa gönderiri sekiz gün kömür taşıtırım anladın mı sekiz gün..\\
\textbf{Necdet} - Evet kaymakam bey.\\
\textbf{Ka{[}ymakam{]}} - Hem dua et ki bir gün elime düşmeyesin, maiyetimde bulunmayasın\\
\textbf{Necdet} - Ederim efendim (Kendi) Bir de bu adam akrabam dayım ha!.. Ah ne aileyiz..\\
\textbf{Şemsi} -- (Kendi) Yaşar da kahvede dokuz doğuruyordur.\\
\textbf{Seha} - (Girer) Kaymakam bey.. Odanız hazır.. Pakize ile beraber istirahatınız için ne mümkünse yaptık..\\
\textbf{Ka{[}ymakam{]}} - Ah hanımefendi. Size bu kadar zahmetler ettirdiğim için affedersiniz..\\
\textbf{Seha} - Aman efendim size bu kadarcık bir hizmette bulunabildiğim için bahtiyarım! (Şemsi'nin yanına gelerek) tasavvur ediniz Necdet Bey dayınız mutlaka otele gitmek istiyorlardı. Kendilerini burada alıkoyabilmek için bilhassa ısrara mecbur kaldım (çapkın ve yaramaz bir nazarla kocasına bakarak) İyi yaptım mı?\\
\textbf{Şemsi} - (Cebrî bir tebessümle) Tabii hanıefendi bu muvaffakiyetinizden dolayı altın madalyaya layıksınız..\\
\textbf{Necdet} - (Seha için kendi) Bu ne acar karıymış Allah'ım..\\
\textbf{Seha} - (Şemsi hakkında kendi) Dümenin elimde Şemsi Bey\ldots{}\\
\textbf{Pakize} -(Elinde bir kağıtla gelerek) Dayıcığım eczajaneden çırağı gçndermişler.. Gaibe sizin için telefona söylenen bir haber var..\\
\textbf{Ka{[}ymakam{]}} - Ha evet eczahanenin telefon numarasını vermiştim.. Ver kızım (Zarfı yırtarken) Müsaade ederseniz şunu okuyayım..\\
\textbf{Seha} - Rica ederim efendim.\\
\textbf{Pakize} - Rica ederiz dayıcığım.\\
\textbf{Şemsi} - Rica ederiz dayıcığım.\\
\textbf{Necdet} - Ben de rica ederim.\\
\textbf{Ka{[}ymakam{]}} - (Ters ters ona bakarak) İn'allah ma-sabarin. Ulan sana ne oldu.. (Okuyarak) " Beyefendi size bihude zahmet vermemek üzere biraz evvel telefonla söylenen şeyi aynen yazıyorum. Vekalet müşteşarı beyefendi bizzat telefonda sizin Karadeniz Bahriye Kumandanlığı'na tayininizi söyleyerek yarın behemahal kendisini görmenizi rica etti. Hürmetler"\\
\textbf{Ka{[}ymakam{]}} - Ey.. Yaşar.. Okudum canına.. Karadeniz'e kumandan mı?\\
\textbf{Seha} - Aman Yarabbi kocamın da amiri oldu..\\
\textbf{Necdet} - (Şaşkın hirâs {[}korku; korkma,ürkme{]} içinde) Aman.. Çarpıntım tutacak.\\
\textbf{Pakize} -Aman Yarabbi.\\
\textbf{Şemsi} - İşte bir bu eksikti..\\
\textbf{Ka{[}ymakam{]}} - (Necdet'e giderek) Oğlum helalleş. Seni gemiden maiyetime alacağım. Bu saniyeden itibaren emrimdesin. Haydi bakalım hazır ol.. Aç başı {[}?{]}..\\
\textbf{Necdet} - (kaymakamın kolları arasına yığılarak) Aman çarpıntım.\\
\textbf{Ka{[}ymakam{]}} - Nasıl.. Oğlana fenalık geldi galiba..\\
\textbf{Pakize} - (Bir sinir buhranı içinde) Ah fena oluyorum.\\
\textbf{Seha} - (Koşarak) Pakize'ciğim.. Kardeşim.. Ne oluyorsun. (Masanın yanındaki sandalyeye düşerek oturan Pakize'nin yanına\\
\textbf{Ka{[}ymakam{]}} - (Pakize'ye bakarak) Ne de olsa süt kardeşi. Şişt Yaşar oğlum\\
\textbf{Şemsi} - (Nevmîd {[}umitsiz{]} kendi) Aman Yarabbi Yaşar da kahvede dokuz doğuruyordur.\\

PERDE İNER

\hypertarget{perde-2}{%
\chapter{3. PERDE}\label{perde-2}}

\hypertarget{birinci-meclis-2}{%
\section{Birinci Meclis}\label{birinci-meclis-2}}

\begin{verbatim}
 [**Yaşar** - Şemsi]
 [Perde açıldığı zaman sokak kapısının çalındığı işitilir. Az sonra bahçe kapısı tarafından Yaşar gözükür..]
\end{verbatim}

Aynı dekor

\textbf{Yaşar} - Kapıyı bacak kadar bir arap kızı açtı.. Ne bahçede ne burada kimseler yok.. Sakın hepsi de sözbirliği edip sıvışmış olmasınlar (Orta masaya gider su içer) Oh.. Susamışım.. Şemsi Ağabey de şimdi beni burada görünce kızacak ama ne yapalım.. Ben ona söyledim vesikamı getirmezsen yine damlarım dedim.. (Bir yere oturur)\\
\textbf{Şemsi} - (Girerek) Her kapı çalındıkça Yaşar diye ödüm kopuyor (Görerek) Ay.. İşte Yaşar.. Emindim zaten..\\
\textbf{Yaşar} - (Sıçrayarak) Ha.. Şemsi Ağabey.. Hemen darılma..\\
\textbf{Şemsi} - Niye geldin bakayım? Ben demedim mi ki?..\\
\textbf{Yaşar} - Dur söyleyim ağabeyciğim. Ne yapayım bir saatten ziyade bekleyemedim. Anlaşıldı ki sen gelmeyeceksin. Tezkeremi buldunsa ver de gideyim.. Vaz geçtim.\\
\textbf{Şemsi} - Yaşar'cığım daha bulamadım.. Çünkü vesikan ceketimin..şeyinde.. Çeketimin şeyinin içinde kalmış da..\\
\textbf{Yaşar} - Ne ? Yoksa vesikamı kayıp mı ettin.\\
\textbf{Şemsi} - (Ağızını kapayarak) Sus diyorum hemen islim koyverme..\\
\textbf{Yaşar} - kayıp mı oldu? Aman sahi ha.. Şemsi Ağabey vesikam kayıp mı oldu..\\
\textbf{Şemsi} - Hayır yavrucuğum hayır, hayır. Şimdi git o kahvede biraz otur arkandan geleceğim. Tabii vesikanla beraber..\\
\textbf{Yaşar} - Ben o kahvede sıkılıyorum. Orada bir Rum oğlanı var yüzüme bakıp sırıtıyor.. Oraya gidemem. Hem bu sefer vesikamı almadan şuradan şuraya gitmeyeceğim..\\
\textbf{Şemsi} - Yaşar'cığım, canım, ciğerim\ldots{}\\
\textbf{Yaşar} - (Kafa tutarak) Canım, ciğerim yok. Yüz sene beklemek lazım gelse bekleyeceğim.. Nah işte buraya demir attım..\\
\textbf{Şemsi} - Üç dakika. Üç dakika sonra arkandan geleceğim diyorum kardeşim..\\
\textbf{Yaşar} - Pişin {[}?{]} sen ! O bir defa olur.\\
\textbf{Şemsi} - Ahh.. Katır inadı.. (Aklına birşey gelmiş gibi) Peki peki (Sağ tarafta birinci plandaki küçük kapıyı göstererek) Gir şu küçük salona ama hiç kımıldama.. Bir yere çıkma..\\
\textbf{Yaşar} - Ne yapacağım orada?\\
\textbf{Şemsi} - Beş dakika oturacaksın a iki gözüm. Bak orada masanın üzerinde tavla var..\\
\textbf{Yaşar} - (Yılışarak) Bir beş mi atacağız?..\\
\textbf{Şemsi} - Yaşar'cığım vesikanı arayacağım. Hem seninle tavla oynayayım hem de vesikanı mı arayayım? Nasıl olur?\\
\textbf{Yaşar} - Hakkın var Şemsi Ağabey. Peki.. Peki\\
\textbf{Şemsi} - Bana Şemsi deme be, ne davul kafalısın!\\
\textbf{Yaşar} - Ha sahi.. Sahi ben senim sen de bensin değil mi?\\
\textbf{Şemsi} - Ya.. Aman geliyorlar çabuk ol.\\
\textbf{Yaşar} - Ah Şemsi Ağabey sen yok mu? (Girer) ({[}Şemsi{]} Yaşar'ın üstünden kapıyı kilitler anahtarı üstünde bırakır..)\\
\textbf{Şemsi} - Ha şöyle sen biraz bekle bakalım Yaşar'cık!.\\

\hypertarget{ikinci-meclis-2}{%
\section{İkinci Meclis}\label{ikinci-meclis-2}}

\begin{verbatim}
 [Şemsi-sonra Necdet-sonra Pakize-sonra Kaymakam]
\end{verbatim}

\textbf{Şemsi} - Ah.. Gözümü açsam bir de baksam ki bütün bunlar rüya imiş..\\
\textbf{Necdet} - (Pakize ile girerek) Demin kapı çalınmıştı kimmiş acaba?\\
\textbf{Şemsi} - (Başından savma) Bilmem! Kapıyı şaşırmış birisi galiba.. Gitti..\\
\textbf{Necdet} - (Kendi vaziyetini düşünerek) Karadeniz Sahil Muhafızlığı Müfettiş-i Umumisi ha !.\\
\textbf{Pakize} - (Kendi vaziyetini düşünerek) Acaba herşeyi güzellikle dayıma söylesem mi?\\
\textbf{Şemsi} - (Birdenbire) Hah.. Buldum ! Bu içinden çıkılmaz berzahtan kurtulmak için bir çare var.~
\textbf{Necdet} - Oh çok şükür! Ey nedir Şemsi Bey?\\
\textbf{Pakize} - Aman çabuk söyleyin..\\
\textbf{Şemsi} - (Başını kaşır düşünür) Ama.. Olamaz.. Bir işe yaramaz ki\\
\textbf{Necdet} - Canım söylesenize.. Zarar yok..\\
\textbf{Şemsi} - Hayır.. Bahriye Vekaletinde büyüklerden bir tanıdığım olsaydı da Gazanfer Kaptan'ın yeni memuriyetin tebdil ettirsek diyordum.\\
\textbf{Pakize} - A.. Ben de bir çare buldunuz zannediyordum.\\
\textbf{Necdet} - (Ona kızgın bakarak) Allah kimseyi şaşırtmasın\\
\textbf{Şemsi} - Latife zannetmeyin. Hiç olmazsa kaymakam beyin bu hafta Trabzon'a gitmesini tehir edecek bir sebeb, bir hile olmalı.. (Aklına birşey gelerek( Ha.. Ben biraz telefona kadar gidip geleyim tanıdık bir şeytanla konuşacağım ve eğer bu adamı yerinde bulursam hileyi de buldum demektir.\\
\textbf{Necdet} - Acaba?\\
\textbf{Şemsi} - Emin olun. Hile bulmaya çekinme {[}?{]} {[}gelince{]} bunda güçlük çekmeyiz\\
\textbf{Pakize} - (Mütebessim) Bundan eminim Yaşar..\\
\textbf{Ka{[}ymakam{]}} - (Dışardan) Yaşar\\
\textbf{Necdet} - Hah.. Eyidir {[}epeydir{]} beni unutmuştu..\\
\textbf{Pakize} - Necdet dayım bağırıyor..\\
\textbf{Ka{[}ymakam{]}} - (Dışardan) Yaşar, Yaşar.\\
\textbf{Şemsi} - (Süratle cebinden Necdet'in vesikasını çıkararak ona verir)\\
\textbf{Şemsi} - Şist.. Bana bak al şu tezkereyi koş yoksa şimdi buraya damlar.\\
\textbf{Ka{[}ymakam{]}} - (Dışardan) Yaşar diyorum\ldots{}\\
\textbf{Necdet} - Geldim kaymakam bey.. Geldim efendim.. (Giderken) hay yaşamaz olayım..\\

\hypertarget{ucuncu-meclis-2}{%
\section{Üçüncü Meclis}\label{ucuncu-meclis-2}}

\begin{verbatim}
 [Şemsi- Pakize]
\end{verbatim}

\textbf{Şemsi} - (Necdet'in gittiğine kani olduktan sonra) Pakize Hanım sizinle bir saniye yalnız kalmak için bahane arıyordum, işte bahane ayağımıza geldi..\\
{[}\textbf{Pakize} -{]} Ne söylemek istiyorsunuz.\\
\textbf{Şemsi} - Pakize Hanım.. Vaziyetimiz gittikçe daralıyor.. (Tevkif) Pakize Hanım.. Mevkiimiz.. (Karar vererek) Size hakikati itiraf edeceğim..\\
\textbf{Pakize} - Hangi hakikati!..\\
\textbf{Şemsi} - Evet hangi hakikati.. Evet söyleyeceğim.. Fakat daha evvel bana vaad ediniz de emin olayım..\\
\textbf{Pakize} - Vaad mi! Neyi?\\
\textbf{Şemsi} - Söyleyim efendim.. Seha Hanım'a hiçbir şeyi söylememenizi rica ederim.. Bunu bana vaad ediyor musunuz?\\
\textbf{Pakize} - Peki- Ama niçin?\\
\textbf{Şemsi} -Ya niçin? Çünkü Seha Hanım (küçük tereddüd) benim karım Seha'dır.\\
\textbf{Pakize} - Haa\ldots{} Seha aldanmamış.. Siz Şemsi Bey'siniz değil mi?\\
\textbf{Şemsi} - (Mahcup, mütebessim) Vallahi doğrusunu söylemek lazım gelirse kim olduğumu doğru dürüst ben de bilmiyorum.. Şemsi idim.. Yaşar oldum.. Yaşar'ken Necdet oldum. Şimdi yine Şemsi oldum.. Ama sizin için. Bütün bu şahsiyetlere iki içinde girdim. Bir kişi için bu oldukça ağır bir vazife değil midir?\\
\textbf{Pakize} - Demek Şemsi Bey siz de zevcenizi aldatmak istiyordunuz öyle mi?\\
\textbf{Şemsi} - Kabahat kimin? Sakın sizin olmasın\\
\textbf{Pakize} - Benim mi?\\
\textbf{Şemsi} - Süt kardeşinize o kadar güzel o kadar nevazişkar{[}gönül alan; iltifat eden{]} mektuplar yazıyordunuz ki zavallı Yaşar.. O da cevap yazmam için ısrar edip duruyordu..\\
\textbf{Pakize} - Ne o mektuplar..\\
\textbf{Şemsi} - Evet o mektupların hepsi benim.. Zavallı Yaşar'cığımın o okuyup yazması o kadar kısa ki ister istemez ben yazardım ve..\\
\textbf{Pakize} - Ve benim mektuplarımı da..\\
\textbf{Şemsi} - Tabii ister istemez ben alırdım. Günün birinde yazılarınızdan hissettiğim cazibenize tahammül edemedim, sizden resminizi istedim. Gönderdiniz..\\
\textbf{Pakize} - Size değil.. Ona..\\
\textbf{Şemsi} - Farz edelim ki ikimize birden.. Madem ki yazan ben yazdıran ise o.. Hayır artık yazdıran da sizdiniz..\\
\textbf{Pakize} - Evet evet şimdi hepsini anladım fakat.. Siz zannediyor musunuz ki.. Oynadığınız bu komediyi biz Seha ile alkışlayacağız? affedersiniz Şemsi Bey..\\
\textbf{Şemsi} - Siz affedersiniz Pakize Hanım. Müsaade ediniz de ikmal edeyim.. ben gayrı-iradi bir surette girdiğim o komedinin kahramanı iken şimdi kurbanı olmaktan korkuyorum.. Bakın demin ne diyordum?.. Emin olun ki bu maceraya irademin haricinde bir surette atıldım. Hadisatın cereyanı seni, bizi hepimizi bu berzaha attı. Şimdi buradan el birliği ile kurtulmaya bakacağımıza birbirimize girecek olursak emin olun kördüğümü oluruz. Ve sonra unutmayınız ki bütün bütün bunların yegane sebebi sizin güzelliğinizdi. Şimdi beni kurtarmak istemez misiniz?\\
\textbf{Pakize} - Zevcenizi seviyorsunuz! Ben bir maceradan başka birşey değilim öyle mi Şemsi Bey..\\
\textbf{Şemsi} - Rica ederim bunu demedim..\\
\textbf{Pakize} - Tabii.. Bunu demediniz.. Çünkü siz terbiyeli bir gençsiniz (İçini çekerek) Haydi canım.. Kendiniz sıkmayın benim Seha benim eski arkadaşım.. Onu sizin kadar ben de severim. Siz de onu tercih etmekte haklısınız.. Çünkü zevceniz.. Benden istediğiniz affa gelince, Şemsi Bey sizi affediyorum hatta size karşı bir parça medyun-u şükranım {[}teşekkür borçluyum{]}.\\
\textbf{Şemsi} - Medyun-u şükran mı?\\
\textbf{Pakize} - Evet sizin sayenizde hayatımda mühim bir tecrübe geçirmiş oldum. Tıpkı ilk defa olarak memnu bir meyveden yiyen bir adam gibi tuhaf bir hiz, bir baş dönmesi içinde kaldım. Fakat işte bugün bu tehlikeli yoldan h,çbir vicdan azabı hissetmeden dönüyorum. Yalnız unutulmaz acı bir hatıra saklıyorum. ;şte bu kadar.. Şimdi söyleyin bakalım Yaşar için nasıl bir iyilik yapabilirim.\\
\textbf{Şemsi} - Pek sade bir şeyi.. Yaşar'ın vesikasını geri alacaksınız.. İşte bu kadar.\\
\textbf{Pakize} - Demek ki Seha onu geri vermedi.\\
\textbf{Şemsi} - Maalesef hayır!.. Yaşar da vesikasını bekleyip duruyor. Şunu vesikasını ele geçiriniz üst tarafını ben halledeceğim Pakize Hanım.\\
\textbf{Pakize} - Nasıl! Hakiki Yaşar.. Asıl süt kardeşim burada mı? Ama nerededir görmek isterdim..\\
\textbf{Şemsi} - Evet ama; şimdi değil..\\
\textbf{Pakize} - Ne vakit..\\
\textbf{Şemsi} - Ne vakit onun vesikasını bulursanız o vakit onu sizin kardeşinize çıkartırım.\\
\textbf{Pakize} - Pekala ben de size vaad ediyorum.\\
\textbf{Şemsi} -Ah Pakize Hanım size ne surette teşekkür edeceğimi bilemiyorum\\
\textbf{Pakize} - Ne suretle mi\_ Şimdiye kadar yaptıklarınızı bir daha yapmamak suretiyle..\\
\textbf{Şemsi} - O.. Buna emin ol süt kardeşim..\\
\textbf{Pakize} - (Gülerek) Haydi öyle ise şimdi işinizin başına..\\
\textbf{Şemsi} - Ne gibi?\\
\textbf{Pakize} - Hani arkadaşınız şeytana telefon edip dayımın Trabzon'a hareketini tehir için hile bulacak değil miydiniz?\\
\textbf{Şemsi} - Evet! Öyle ya gidiyorum (Küçük salon tarafına bakarak bir küçük tereddütten sonra kendi) Allah vere de Yaşar'ın olduğu yere kimse girmese.. (Çıkar)\\

\hypertarget{dorduncu-meclis-2}{%
\section{Dördüncü Meclis}\label{dorduncu-meclis-2}}

\textbf{Pakize} - (Yalnız) Evet onu kutarmak vazifemdir.\\
\textbf{Seha} - (Gelerek) Kuzum sahte Yaşar nerelerde?\\
\textbf{Pakize} - Biraz evvel sokağa çıktı.\\
\textbf{Seha} - Sokağa mı çıktı? İşte benim de kortuğum bu idi. Sen de mani olmadın öyle mi?\\
\textbf{Pakize} - A.. Seha'cığım onu burada zorla alıkoyamazdım ya!\\
\textbf{Seha} - Peki hiç olmazsa nereye gittiğini sormadın mı?\\
\textbf{Pakize} - Aman Seha'cığım, süt kardeşim burada karantina altında değil.. Canı istediği zaman sokağa çıkamaz mı?\\
\textbf{Seha} - Aman Pakize! Sür kardeşin, süt kardeşin.. Düşünmüyorsun ki Yaşar Efendi karantina altında değil ama gayet şüpheli bir vaziyettedir. galiba pek yakında teslim bayrağını çekmek lazım geldiğini hissetiği için sıvıştı. Eminim ki cezadan kurtulmak için yine mutlaka kendine mahsus birtakım hileli dolaplar kurmaya gitmiştir.\\
\textbf{Pakize} - Hayır canım bilakis buradan çıktığı zaman gayet tabii idi\\
\textbf{Seha} -Ah.. Bana haber vermek yok muydu kardeşim ? Bak ben nasıl onun yolunu keserdim.\\
\textbf{Pakize} - Oo.. Demek ki senin uzaktan uzağa bir şüphen var diye..\\
\textbf{Seha} - (Bağırarak) Ne, uzaktan uzağa mı?\\
\textbf{Pakize} - Öyle ya.. Bari elinde ciddi ispatlar olsa..\\
\textbf{Seha} - Ya..\\
\textbf{Pakize} - Değil mi ya.. Nereden nereye,tesadüfen kocanla aynı gemiye düşmüş zavallı bir aşcıyı iki sene görmediğin kocana benzetip türlü türlü şüphelere düşüyorsun.\\
\textbf{Seha} - Vay, vay, vay..\\
\textbf{Pakize} - Bu çılgınlıktan başka birşey değil Seha'cığım..\\
\textbf{Seha} - (pakize hakkında şüpheler içinde kendi) Acaba, sakın\\
\textbf{Pakize} - Canım sana vesikasını gösterdi ya..\\
\textbf{Seha} - Evet öyle vesikasını gördüm.\\
\textbf{Pakize} - Pekala bu bir ispat değil mi? Vesikasını göstermemiş olsa idi neyse. Sahi onun vesikasını iade ettin miydi?\\
\textbf{Seha} - (Masum bir tavırla) Ne peki{[}Dizgi hatası-Nesini{]} cicim\\
\textbf{Pakize} - Askerlik vesikasını..\\
\textbf{Seha} - Hayır.. Alıkoydum. Ta emin oluncaya kadar da vermeyeceğim. O bana lazım..\\
\textbf{Pakize} - (Yalandan gülerek) Ya.. Pekala edersin. Nereye koymuştun onu?\\
\textbf{Seha} - Yattığım odada camlı dolabın alt gözüne koymuştum, Pakize'ciğim.\\

\hypertarget{besinci-meclis-2}{%
\section{Beşinci Meclis}\label{besinci-meclis-2}}

\begin{verbatim}
 [Kaymakam - Pakize-Seha]
\end{verbatim}

\textbf{Kaymakam} - Eyy.. Pakize kızım rahatsızlığın geçti ya?.. Nasılsın?\\
\textbf{Pakize} - Teşekkür ederim dayıcığım.. Şimdi iyiyim.. Demin balkonda oturdum. Biraz hava aldım geçti.. Bir parça da odama gidip uzanmak istiyorum.\\
Kaymakam - Hah.. Ya iyi olur.. Git istirahat et yavrum.\\
\textbf{Pakize} - (Seha'ya) Beni mazur görüyorsun ya Seha'cığım\\
\textbf{Seha} - Aman kardeşciğim ben yabancı mıyım? Allahaşkına sıkılma\\
\textbf{Pakize} - (kendi, giderken) Camlı dolabın alt gözünde\\
\textbf{Kaymakam} - Bak Pakize'ciğim Seha Hanım da izin verdiler. artık odana çıkıp iyice istirahat et hatta biraz uyursan fena etmezsin.. Bizi hiç düşünme.. Sen iyice istirahat et kızım..\\
\textbf{Seha} - (Kendi) Git camlı dolabı iyice karıştır bakalım. (Vesikayı göğsünden çıkarıp bakarak) Vesikanın yerini bulabilir misin? (Vesikayı yine göğsüne koyup ayağa kalkar)\\

\hypertarget{altinci-meclis-2}{%
\section{Altıncı Meclis}\label{altinci-meclis-2}}

\textbf{kaymakam} - (Yaklaşarak) Şu Pakize ne narin, ne ince bir kadındır değil mi, seha Hanım? Bakın bizi başbaşa bıraktı..\\
\textbf{Seha} - Tabii değil mi efendim! Pakize pek samimi bir arkadaşımdır.\\
\textbf{Ka{[}ymakam{]}} - Ah Seha'cığım.. Ben, ben ki hayatın bütün dalgalarına bi-perva göğüs germiş bir adamım. Bir saatten beri kasırgaya tutulmuş tekne gibi zangır zangır titriyorum\\
\textbf{Seha} - Titriyor musunuz? A.. Niçin?\\
\textbf{Ka{[}ymakam{]}} - Söyleyim civanım. Ya kocanız size sadıksa.. Benim halim ne olacak diye..\\
\textbf{Seha} - Ha.. eğer sizi korkutan, düşündüren şey bu ise emin olun..\\
\textbf{Ka{[}ymakam{]}} - Nasıl, sahi mi söylüyorsunuz? (Titreyerek)\\
\textbf{Seha} - Evet! hem biraz evvel bundan bir parça şüphe edebiliyordum. Fakat şimdi artık o şüphem de kalmadı.\\
\textbf{Ka{[}ymakam{]}} - Ya.. Sakın yeni bir şey mi öğrendiniz..\\
\textbf{Seha} - Evet! Öğrendiğim şu ki: Bir kadın zevcininin sadakatine olduğu kadar muhibbelerinin samimiyetine de itimat etmemeli imiş..\\
\textbf{Ka{[}ymakam{]}} - Acaip zevciniz size muhibbelerinizden birisiyle mi hiyanet ediyormuş.\\
\textbf{Seha} - Hem de nasıl muhibbem!! Ne ise kaymakam bey.. Anladım ki hasmım bir değil iki imiş. Herhalde muzafferiyetim de iki misli olacaktır.\\
\textbf{Ka{[}ymakam{]}} - Yaşar \ldots{} Seha Hanım siz insanı yalnız kalben değil fikren de tesih ediyorsunuz {[}büyülüyorsunuz{]}\\
\textbf{Seha} - Sonra eminim ki hissiyatım beni aldatmıyor. İçim öyle diyor ki zevcim şimdi İstanbul'dadır. Fakat niye sabahleyin evine gelmedi? Bunu mutlaka öğrenmeliyim. evet, bunu mutlaka öğrenmeliyim. katiyen eminim ki yarım saate kadar öğrenmek istediğimi öğreneceğim. Ve belki de kendisini ele geçireceğim..\\
\textbf{Ka{[}ymakam{]}} - Vay işte bu olmadı.. hemen gidiyor musunuz?\\
\textbf{Seha} - Onu ne kadar çabuk ele geçirebilirsem o kadar çabuk.. Sizin olacağım..
\textbf{Ka{[}ymakam{]}} - Öyle ise durmayın Seha Hanım, gidin.. Ne yapacaksanız yapın. Yalnız unutmayınız ki Allah göstermesin zevciniz size sadıksa benim de işim bitiktir. Bu inkisar-i hayale {[}hayal kırıklığına{]} dayanamam ölürüm.\\
\textbf{Seha} - Hayır, hayır, tehl,ke yok.. Yaşayacaksınız.. Benim için yaşayacaksınız.. (İçeri girer. kaymakam o giderken öpücük gönderir ve sonradan)\\
\textbf{Ka{[}ymakam{]}} - (Kendi) Oh bu kadın bana herşeyi yaptırabilir. (``Sen tazeledin ömrümü ey taze civanım'' şarkısını mırıldanır..)\\

\hypertarget{yedinci-meclis-2}{%
\section{Yedinci Meclis}\label{yedinci-meclis-2}}

\begin{verbatim}
 [Kaymakam - Yaşar- sonra Necdet]
\end{verbatim}

(Evvela Yaşar'ın bulunduğu odanın kapısı vurulur.. Sonra ses duyulur)
\textbf{Yaşar} - (İçeriden) Ağabey be.. Beni buraya ne diye hapsettin? açsana be.. Ağabey Allahaşkına aç..\\
\textbf{Ka{[}ymakam{]}} - (Kendi) Bu da kim (Gidip kapıyı açar)\\
\textbf{Yaşar} - (Birden çıkarak hiddetle kaymakamın üzerine yürür) Alay mı ediyorsun Allahaşkına, ne diye beni hapsettin (Kaymakamı görerek birdenbire selam vaziyeti takınır) Vay geçmişini\ldots{} Bir kaymakam be!!!\\
\textbf{Ka{[}ymakam{]}} - Bu da nesi sen nereden çıktın.. Kimin nesisin?\\
\textbf{Yaşar} - Ben mi efendim, ben iyi bir ailedenim.. Babamın biricik oğluyum. Benden evvelkilerin hepsi öldüğü için\\
\textbf{Ka{[}ymakam{]}} - Yok be.. Birdenbire böyle nereden çıktın, diyorum.. Rahat dur.\\
\textbf{Yaşar} - Gördünüz ya bu odadan, kaymakam bey..\\
\textbf{Ka{[}ymakam{]}} - Peki bu odada ne yapıyordun..\\
\textbf{Yaşar} - (Elindeki tavla zarlarını işaret ederek) Ben mi? Hiç efendim. kendi kendime oynuyordum.\\
\textbf{Ka{[}ymakam{]}} - Kendi kendine oynuyor musun?\\
\textbf{Yaşar} - Evet. Çift getirebilir miyim diye.. Kendi kendime. Ne yaparsınız.. (Sırıtarak)~
\textbf{Ka{[}ymakam{]}} - Peki orada kimi bekliyordun?\\
\textbf{Yaşar} - Ben mi? Arkadaşımı kaymakam bey..\\
\textbf{Ka{[}ymakam{]}} - Arkadaşını mı? (serpuşundaki Aytaş .. ismine bakarak) Vay sen Aytaş Gemisi'nden misin?\\
\textbf{Yaşar} - Evet efendim Karadeniz Sevahil Bahriy Kumandanlığı maiyetinde Aytaş Karakol Gemisi'nden.\\
\textbf{Ka{[}ymakam{]}} - Anladım.. Sen Yaşar'ı bekliyorsun oğlum.\\
\textbf{Yaşar} - Nasıl da bildiniz kaymakam bey. Evet onu bekliyorum\\
\textbf{Ka{[}ymakam{]}} - Pekala.. Dur bakalım (İçeriye bağırarak) Yaşar.. Aşağı gel seni istiyorlar.\\
\textbf{Yaşar} - (kendi) Vay canına keşke odadan çıkmasa idim kendi kendimi mars eder dururdum.\\
\textbf{Ka{[}ymakam{]}} - (Dönerek) Ey adın nedir bakalım senin?\\
\textbf{Yaşar} - Benim mi? Efendim.. Şemsi.\\
\textbf{Ka{[}ymakam{]}} - (Hayretle) Ne ! Şemsi mi? Şemsi Nuri?. İstanbullu?\\
\textbf{Yaşar} - Evet kaymakam bey İstanbullu Şemsi Nuri\\
\textbf{Ka{[}ymakam{]}} - (Kendi) Ne seha hanım'ın kocası bu mu? haydi canım mümkün değil! (Yüksek) Göster bakayım tezkereni bahriyeli..\\
\textbf{Yaşar} - (Tezkereyi cebinden çıkarıp verir) Benim mi, buyurun kaymakam bey..\\
\textbf{Ka{[}ymakam{]}} - (Okuyarak) Aytaş.. Şemsi Nuri, İstanbul.. Telsiz memurunun maiyetindesin öyle mi?\\
\textbf{Yaşar} - Ben mi? evet efendim sayenizde\\
\textbf{Ka{[}ymakam{]}} - (Yaşar'a bakarak kendi) Yuha.. Bu yobaz onun kocası!. Öyle gül gibi kadına bu saksı suratlı herif düşmüş ha.. Ne yazık.. Üste bir de kadına hiyanet ediyor.\\
\textbf{Yaşar} - (Kendi) Niye bana öyle ters ters bakıyor? kaptanı kızdırdık mı yoksa?\\
\textbf{Ka{[}ymakam{]}} - (Tezkere'yi vererek) ey şöyle otur bakalım Şemsi Efendi, ayakta kaldın..\\
\textbf{Yaşar} - Ben mi? Aman kaymakam bey sizin huzurunuzda\\
\textbf{Ka{[}ymakam{]}} - Otur canım ben öyle istiyorum haydi.. (Yaşar oturur) İstanbul'a ne vakit geldin..\\
\textbf{Yaşar} - Ben mi? Bu sabah kaymakam bey (Tezkereyi cebine koyar)\\
\textbf{Ka{[}ymakam{]}} - Nereye indin?\\
\textbf{Yaşar} - Ben mi? Rıhtıma efendim.\\
\textbf{Ka{[}ymakam{]}} -Tabii onu sormuyorum sonra doğruca nereye gittin\\
\textbf{Yaşar} - Ben mi? Doğruca kartal'a, bizim suvarinin evine. Emanetler vardıda..\\
\textbf{Ka{[}ymakam{]}} -Ya.. Sonra..\\
\textbf{Yaşar} - Sonra hiç efendim buraya geldim..\\
\textbf{Ka{[}ymakam{]}} - Hiç mi? Yalan söylüyorsun..\\
\textbf{Yaşar} - Vallahi Billahi iki gözüm önüme aksın ki kaymakam bey\\
\textbf{Ka{[}ymakam{]}} - Dur yemin etme (Yaklaşıp biraz laubaliyane bir tavırla) Oturun. Eminim ki.. Sen bu sabah değil dün akşam gelmissindir.\\
\textbf{Yaşar} - Ben mi? Vallahi değil, kaymakam bey.. Bu sabah geldim ama\\
\textbf{Ka{[}ymakam{]}} - (Daha mürim {[}?{]} ) haydi canım sana izin veriyorum bana söyle bakayım.. Madem ki bu sabah geldim diyorsun, inandım.. Şimdi ötesini de söyle bakayım, kendine güzel bir ziyafet çekilmişsindir ya..\\
\textbf{Yaşar} - Ben mi? Oo.. Evet..\\
\textbf{Ka{[}ymakam{]}} - Ha..\\
\textbf{Yaşar} - Köprüden çıkar çıkmaz doğru muhallebiciye gittim üç tabak kaymaklı kadayıf yedim.\\
\textbf{Ka{[}ymakam{]}} - Onu sormuyorum canım.. Başka bir şeyi.. Eski bir aşna fişnaya gittin mi? Mutlaka bir yere uğramış yaramazlık yapmışsındır. Anlarsın ya haydi söyle bakalım..\\
\textbf{Yaşar} - (sırıta sırıta) Aman kaymakam bey Vallahi beni mahcup ediyorsunuz.\\
\textbf{Ka{[}ymakam{]}} - Ne var mahcup olacak be ben de erkeğim haydi.. Açıl. Ben öyle istiyorum. evli bir kadın mıydı ha..\\
\textbf{Yaşar} - (Sırıtarak) Yok canım kaymakam bey bir çamaşırcı idi\\
\textbf{Ka{[}ymakam{]}} - Ne bir çamaşırcı mı?\\
\textbf{Yaşar} - Evet.. hani suvarinin evine gitmiştim ya.. Ama darılmaca yok kaymakam bey..\\
\textbf{Ka{[}ymakam{]}} - Yok yok haydi (Kendi) Aman Yarabbi şu dünyaya bakın.\\
\textbf{Yaşar} - Oraya gitmiştim.. Ondan sonra efendim beni mutfağa aldılar. Orada bekledim. Beklerken de bu çamaşırcı orada iken koca bir leğen içinde çamaşır çitiliyordu. Bir yandan da yüzüme bakıp bakıp, gevrek gevrek gülüyordu. dayanamadım kaymakam bey tabakayı çıkardım bir hazır cigara uzattım, nah dedim beraber tellendirelim biraz soluk al..Aldı. Ondan sonra efendim bana elin titriyor dedi. Ben de ona yalnız elim değil her tarafım titriyor dedim. Artık sonrasını bilmiyorum.. Gözüm kızmış çamaşırcının beline sarılmışım.. Sonra efendim o da ıslak ıslak elleriyle, kollarımı yakaladı var kuvveti ile çekmeye başladı. Aman dedim, cankurtaran..\\
\textbf{Ka{[}ymakam{]}} - (Keserek) Pekala bitir, bitir artık..\\
\textbf{Yaşar} - Nerede efendim bitiremedim.. Sonra üstümüze geldiler. Demir de tutmadı..\\
\textbf{Ka{[}ymakam{]}} - Kes dedim sana (Kendi) Süphanallah Süphanallah\\
\textbf{Yaşar} - Efendim siz emretmiştiniz de..\\
\textbf{Ka{[}ymakam{]}} - Pekala.. Bitir artık..\\
\textbf{Yaşar} - Başüstüne kaymakam bey..\\
\textbf{Ka{[}ymakam{]}} - (Kendi) Seha hanım'ın muhibbelerinden birisiyle değil.. Bir çamaşırcı karısıyla..
Ha.. (Necdet görünür)\\
\textbf{Necdet} - Kaymakam bey beni mi istemişler?\\
\textbf{Ka{[}ymakam{]}} - Evet yaklaş bakalım Yaşar..\\
\textbf{Yaşar} - (Kendi hayretle) Vay bizim adaşmış bu da Yaşar ha..\\
\textbf{Ka{[}ymakam{]}} - (Yaşar'ı göstererek) Bak kim gelmiş iyi değil mi? (Gülerek aralarından çekilir)\\
\textbf{Necdet} - (Kendi) Deminki bahriyeli (Birbirlerine garip garip bakışırlar)\\
\textbf{Ka{[}ymakam{]}} - Eyy.. Niye öyle birbirinizn yüzüne baka kaldınız. Gemi arkadaşın Şemsi'yi tanıdın mı?\\
\textbf{Necdet} - Şemsi mi?\\
\textbf{Ka{[}ymakam{]}} - Ulan arkadaşını tanımıyor musun?\\
\textbf{Necdet} - Ha.. Tanımaz olur muyum efendim? Bizim koca Şemsi'yi tanımaz olur muyum.. (Yaşar'a yaklaşarak) vay Şemsi'ciğim! Safa geldin azizim. Nasılsın? (Kucaklaşırlar)\\
\textbf{Yaşar} - İyiyim iyiyim.. (Kendi) Bu da kim.. Sanki kırk yıllık ahbap gibi beni kucaklıyor..\\
\textbf{Necdet} - Vallahi seni gördüğüme ne memnun oldum, biliyor musun? Ey.. Ötekiler, gemidekiler nasıl? İyilerdir ya\\
\textbf{Yaşar} - Gemidekiler mi? İyi.. İyi hepsi iyidir.\\
\textbf{Ka{[}ymakam{]}} - Yaşar..\\
\textbf{Necdet} - Kaymakam bey.\\
\textbf{Ka{[}ymakam{]}} - Git derhal haber ver..(Fikrini değiştirerek) Yok yok.. Ben kendim giderim. (Girerken bir daha dönüp dışarı bakarak kendi) Bir de Seha Hanım bizim damadı bu herife benzetiyor..\\

\hypertarget{sekizinci-meclis-2}{%
\section{Sekizinci Meclis}\label{sekizinci-meclis-2}}

\begin{verbatim}
 [**Yaşar** - **Necdet** - sonra Kaymakam]
\end{verbatim}

\textbf{Yaşar} - Ne oluyoruz.. Bu evde bir alaya aldılar galiba. Sahi senin adın Yaşar mı hemşeri..\\
\textbf{Necdet} - Evet!.\\
\textbf{Yaşar} - Hem de Aytaş'da öyle mi?\\
\textbf{Necdet} - (Endişe ile) Evet !.\\
\textbf{Yaşar} - Haydi canım.. Sen aç da.. başınla alay et. Kendinin Yaşar, Aşcı Yaşar olduğuna emin misin?\\
\textbf{Necdet} - Emin olmaz olur muyum azizim.. Yani bak sana anlatayım. benim adım Yaşar ama kendim değil.. Yani hem Yaşar'ım hem değilim doğrusu.. İsmim Yaşar, Kendim Necdet'im..\\
\textbf{Yaşar} - Ne.. Sen Necdet'sin ha.. Necdet.. Sensin ha (Gülmeye başlar)\\
\textbf{Necdet} - Ne gülüyorsun ben Necdet'im.. Yaşar olan öteki onun adı Yaşar..\\
\textbf{Yaşar} - Öteki mi?\\
\textbf{Necdet} - Evet ama.. Şimdi artık o da Yaşar değil!.. Binaenaleyh ben Necdet olduğum halde Yaşar'ım..\\
\textbf{Yaşar} - (Meselenin içinden çıkamayıp gülmeye başlar) Ulan amma zihnim karıştı be.. şimdi yaşar'la üç oldu ha..\\
Necdet- Hayır üç değil iki..\\
\textbf{Yaşar} - Ne ikisi be? Öyle ise ben kimim? Zorla bana kendimi kayıp ettireceksiniz..\\
\textbf{Necdet} - Nasıl sen Şemsi değil misin?\\
\textbf{Yaşar} - Yoo..\\
\textbf{Necdet} - Sahi mi azizim\\
\textbf{Yaşar} - Nah istersen yemin ederim: Vallahi, Billahi, iki gözüm önüme aksın ki..\\
\textbf{Necdet} - Öyle ise karın Seha Hanım da karın değil\\
\textbf{Yaşar} - Nasıl benim karım mı varmış, iyi be.. Ben Yaşar'ım ayol.. Aşcıbaşı Yaşar. Ezelden bekar. Ben siz değilim ki karım olsun.\\
\textbf{Necdet} - Peki ama.. Burada bizim evdeki Yaşar kim?\\
\textbf{Yaşar} - Buradaki yaşar mı? Senin anlayacağın o da Şemsi'dir\\
\textbf{Necdet} - Ne Şemsi mi?\\
\textbf{Yaşar} - Evet, Yaşar.. O benim ben de oyum (Sırıtır) Çaktın mı dalavereyi?\\
\textbf{Necdet} - (Şaşkın, başını ellerinin içine alarak) Dur, dur iki gözüm sne Yaşar osun, Necdet Yaşar da sensin hayır o da Yaşar. Yani Necdet de Yaşar.. Aman Yarabbi hesabın içinden çıkamadım ama meseleyi anlamaya başlıyorum galiba..\\
\textbf{Yaşar} - Aferin sana. Ben hesapta dahil değilim ama yine meselenin içine karıştım. Yine birşey anlamıyorum. iyi ama şunu söyle arkadaş: Bu kaymakam neci oluyor..\\
\textbf{Necdet} - Kaymakam Gazanfer Kaptan, yeni Karadeniz Sevahili Bahriye kumandanı\\
\textbf{Yaşar} - (Sıçrayarak) Ne dedin ne dedin?\\
\textbf{Necdet} - Karadeniz Sevahil Umum Bahriye kumandanı.. Yeni tayin edildi..\\
\textbf{Yaşar} - Hadi canım şerm-ağızlığın sırası mı, yoksa bana dolma mı yutturacaksın..\\
\textbf{Necdet} - Dolma molma yutturduğum yok.. Alayın sırası değil arkadaş..\\
\textbf{Yaşar} - Eyvah desene ki bir çuval inciri berbat ettik. Ben buradan palamarı çözüyorum. Bana bak Necdet olan Yaşar, Yaşar olan Şemsi'ye söyle: Şemsi olan Yaşar, Yaşar olarak gidiyor. (Dibe doğru gider)\\
\textbf{Necdet} - (Birşey anlamadan) Bir daha söyle, bir daha söyle bakayım..\\
\textbf{Yaşar} - Sana söyledim Necdet olan Yaşar dedim.Yaşar olan Şemsi'ye söyle: Şemsi olan Yaşar, Yaşar olarak.. (Birden keserek) Oo.. Aman Yarabbi mümkünnü yok, sıvışamam.. Daha tezkeremi, elde edemedim.\\
\textbf{Necdet} - Tezkereni mi? Tezkeren yok mu?\\
\textbf{Yaşar} - Var ama Şemsi olan Yaşar alıp getiremedi..(Kaymakamın sesi işitilir)\\
\textbf{Ka{[}ymakam{]}} - (Dışarıdan) Aşağıki salonda..\\
\textbf{Yaşar} - (Korku içinde) Aman Şemsi kaymakam.. Ben yine deliğe tıkılıyorum.. (eliyle gösterir) Benim için gitti dersin\\
\textbf{Necdet} - Peki..\\
\textbf{Yaşar} - (Girerken( Ah babm.. İzinliyim ama.. Haydi deliğe..\\

\hypertarget{dokuzuncu-meclis-2}{%
\section{Dokuzuncu Meclis}\label{dokuzuncu-meclis-2}}

\begin{verbatim}
 [Necdet-sonra Kaymakam-sonra Seha]
\end{verbatim}

\textbf{Necdet} - (Yalnız) Tekrarladığım şeylerden birşey anlayamadım..\\
\textbf{Ka{[}ymakam{]}} - (Girdiği yerden) Gelin gelin hanımefendi çabuk vakitte (?) müjdemi isterim ama (Sahneye bakıp yaşar'ı göstererek) Eyy hani ya.. Şemsi nerede?.\\
\textbf{Necdet} - Şemsi çıktı kaymakam bey..\\
\textbf{Ka{[}ymakam{]}} - Çıktı mı? Gitti mi?\\
\textbf{Necdet} - Evet efendim\\
\textbf{Seha} - (Girerek) Müjdeniz ne imiş bakalım kaymakam bey\\
\textbf{Ka{[}ymakam{]}} - Müjdem şu ki.. Zevciniz geldi.. Bir dakika evvel de burada idi..\\
\textbf{Seha} - (Pek mütehayyir) Zevcim mi? Sahi mi kaymakam bey? Siz gözünüzle gördünüz mü?\\
\textbf{Ka{[}ymakam{]}} - Böyle sizi nasıl görüyorsam onu da böylece görmüştüm\\
\textbf{Seha} - Yanılmadığınıza emin misiniz?\\
\textbf{Ka{[}ymakam{]}} - Nasıl emin olamam ki tezkeresine baktım\\
\textbf{Seha} - Nasıl tezkeresi de..\\
\textbf{Ka{[}ymakam{]}} - Gördüm efendiciğim zaten.. Hah.. Durun herhalde uzak olmamalıdır. (Necdet'e) Koş arkasından Yaşar.. Ve onu icabederse cidden {[}dizgi hatası- cebren olmalı{]} geir buraya\\
\textbf{Necdet} - Peki ama kaymakam bey..\\
\textbf{Ka{[}ymakam{]}} - Sana koş diyorum behey şamandıra kafalı herif, hala duruyor..\\
\textbf{Necdet} -Koşuyorum, koşuyorum (Kendi) Ah keşke bütün angarya böyle olsaydı.. (Çıkar)\\

\hypertarget{onuncu-meclis-1}{%
\section{Onuncu Meclis}\label{onuncu-meclis-1}}

\begin{verbatim}
 [Evvelkiler - Necdet'ten maada]
\end{verbatim}

\textbf{Seha} - Peki kaymakam bey zevcim buraya ne yapmaya gelmiş ?.\\
\textbf{Ka{[}ymakam{]}} - Arkadaşı Yaşar'ı görüp kucaklamaya gelmiş..\\
\textbf{Seha} - Kendisine benim burada olduğumu söylediniz mi?\\
\textbf{Ka{[}ymakam{]}} - Ne.. O kadar enayi miyim ben? Aman. Affederssiniz ağızımdan kaçtı.. Bunu söyler miyim hiç.. Once onun güzelce ağızını aradım. Ama hiç oralarda olmadan şöyle kurnazca\ldots{} bana herşeyi itiraf etti.\\
\textbf{Seha} - Herşeyi itiraf etti mi?\\
\textbf{Ka{[}ymakam{]}} - Evet.. Yalnız o dediğiniz gibi sizi muhibbelerinizden birisiyle değil bir çamaşırcı karısıyla aldatmış..\\
\textbf{Seha} - Bir çamaşırcı ile mi?\\
\textbf{Ka{[}ymakam{]}} - Evet suvarisinin evinde mutfakta bulduğu bir çamaşırcı.\\
\textbf{Seha} - Aman Yarabbi.. Kocam bir çamaşırcı karısıyla ha? İşte bu yalandır.\\
\textbf{Ka{[}ymakam{]}} - Orasını bilmem.. Fakat bana anlattığı bu idi. Hem bana kalırsa doğru idi.. Çünkü Seha Hanım, affedersiniz ama, genç bir adam birçok senelik perhizden sonra.. Bu derece gözleri dönebilir..\\
\textbf{Seha} - Şu halde demek ki Yaşar hakiki Yaşar'mış..Öyle mi herhalde ne olursa olsun bana hiyanet ettiği muhakkak ya\ldots{}\\
\textbf{Ka{[}ymakam{]}} - Hah.. Siz ona bakım.. Hem sonra zevcinizin zevcinizin bizim damatla olan müşabahatına gelince: Darılmayın ama Seha Hanım siz biraz mübalağayı seviyorsunuz..\\
\textbf{Seha} - Aman kaymakam bey onun su götürüryeri yok.. Necdet Bey kocamın bir eşi..\\
\textbf{Ka{[}ymakam{]}} - Herhalde ben gözlerime aldanmadığıma eminim.. Nerede Necdet Bey, nerede buraya gelen Şemsi..\\
\textbf{Seha} - Aman Yarabbi şimdi onu görebilmek için ne isteseler verirdim.. Allah vere de Necdet Bey arkasından yetişmiş olsun.\\
\textbf{Ka{[}ymakam{]}} - Necdet mi? Yaşar demek istiyorsunuz değil mi?\\
\textbf{Seha} - Ha evet.. Yaşar, Yaşar (kendi) Birbirine karıştırıyorum.\\
\textbf{Ka{[}ymakam{]}} - O da öyle hımbıl birşey ki.. Durun bir de ben bakayım\\
\textbf{Seha} - A.. Rica ederim zahmet etmeyin kaymakam bey\\
\textbf{Ka{[}ymakam{]}} - Zararı yok efendim. Zevcinizi buraya mutlaka getireceğim. ya dirisini ya ölüsünü.. (Çıkar)\\

\hypertarget{on-birinci-meclis-1}{%
\section{On Birinci Meclis}\label{on-birinci-meclis-1}}

\begin{verbatim}
 [**Seha** - sonra Yaşar- sonra Kaymakam ve Necdet]
\end{verbatim}

\textbf{Seha} - (Yalnız) Allah düşmanı bu kadar karışık bir berzaha düşürmesin..\\
\textbf{Yaşar} - (Kapıdan başını çıkararak) Oh çok şükür.. Kaymakam fayrap etmiş.. Oo bir kadın.\\
\textbf{Seha} - (Göğsünde sakladığı tezkereyi çıkarark) Demek bu vesika hakiki Yaşar'ın tezkeresi imiş.\\
\textbf{Yaşar} - (Seha'nın elindeki tezkereyi görerek) Benim tezkerem.. (Birden Seha'nın arkasından tezkereyi kapar)\\
\textbf{Seha} - (Korku ile haykırarak) Ay..\\
\textbf{Yaşar} - Ha şöyle ele geçirdim ya..\\
\textbf{Seha} - Siz kimsiniz? Verin şu tezkereyi bana..\\
\textbf{Yaşar} - Yoo kusuruma bakmayın ama işte onu yapamam. Çünkü o benim izin tezkerrem..\\
\textbf{Seha} - Sizin mi?\\
\textbf{Yaşar} - Benim ya.. Benim malım. Benim Yaşar olduğumu bu kağıt tasdik ediyor.~
\textbf{Seha} - Yaşar mı?\\
\textbf{Yaşar} - Evet.. KAradeniz Umum Bahriye Kumandanlığı maiyetinde Aytaş Karakol Gemisi Aşcıbaşısı Yaşar..\\
\textbf{Seha} - (Mütehayyir kendi) Ya.. Demek ki..\\
\textbf{Ka{[}ymakam{]}} - (Necdet'i bir kulağından tavlamış olduğu halde girer) Hayvan sen yetişemedin ha..\\
\textbf{Necdet} - Efendim bir merkebe atladığını gördüm ama arkasından koşamadım\\
\textbf{Yaşar} - Al bakalım yine geldi..\\
\textbf{Ka{[}ymakam{]}} - (Ortaya ilerlerken Yaşar'ı görüp hayretinden haykırarak) Ooo.. İşte burada ya..\\
\textbf{Seha} - (Kendi) Ne..\\
\textbf{Necdet} - (Kendi) Eyvah şimdi ne yapacağız.\\
\textbf{Ka{[}ymakam{]}} - (Necdet'e) Bir de bana aval kandırır gibi merkebe atladığını gördüm dersin değil mi?\\
\textbf{Necdet} - (Şaşkın) Öyle gibi gördümdü kaymakam bey\\
\textbf{Ka{[}ymakam{]}} - Gibi gördündü ha.. Bana bak çabuk defol gözümün önünden..Yoksa şimdi kaburga kemiklerini parça parça ederim.\\
\textbf{Necdet} - (Çekilirken) Başüstüne efendim (Kendi) O.. Tavanarasına çekilipte sekiz gün kimseye görünmem (Şiddetle çıkar)\\

\hypertarget{on-ikinci-meclis-2}{%
\section{On İkinci Meclis}\label{on-ikinci-meclis-2}}

\begin{verbatim}
 [**Yaşar** - Kaymakam - Seha]
\end{verbatim}

\textbf{Ka{[}ymakam{]}} - (Dönüp Seha'nın karşısına gelerek) Eyy.. Hanımefendi nihayet zevciniz Şemsi Bey'e kavuştunuz ya. Ben sizi yalnız..\\
\textbf{Seha} - fakat kaymakam bey.. Bu adam benim zevcim değil..\\
\textbf{Ka{[}ymakam{]}} - O değil mi?\\
\textbf{Yaşar} - (masum bir tavırla) Hanımın zevci Şemsi mi? Ben ni? Bu işte bir yanlışlık var.. Kaymakam bey..\\
\textbf{Ka{[}ymakam{]}} -Nasıl yanlışlık var? Demin adını sordum Şemsi dedin ya !.\\
\textbf{Yaşar} - (Soğukkanlılıkla) Nasıl olur kaymakam bey? Nasıl diyebilirim? Ben daha şimdi geldim..\\
\textbf{Ka{[}ymakam{]}} - Şimdi mi geldin?.. Allah Allah demin de esrar çektim zannederim (Kendi) Bu andavallı benimle alay ediyor galiba.. (Yüksek) Pekala kimsin bakalım? adın ne?\\
\textbf{Yaşar} - Yaşar kaymakam bey !.\\
\textbf{Ka{[}ymakam{]}} - (Sıçrayarak) Yaşar mı ?.\\
\textbf{Yaşar} - Evet efendim. Karadeniz Umum Bahriye Kumandanlığı maiyetine memur Aytaş KArakol Gemisi aşcıbaşısı\\
\textbf{Ka{[}ymakam{]}} - Ne halt eder ağanın beygiriabuk vesikanı çıkart..\\
\textbf{Yaşar} - (Seha'dan kaptığı vesikayı verip) Buyurun efendim\\
\textbf{Ka{[}ymakam{]}} - (Vesikaya göz atarak) Süphanallah.. Rüya mı görüyorum acaba?\\
\textbf{Yaşar} -(apar) kaymakamda hoşafın yağı kesildi\\
\textbf{Ka{[}ymakam{]}} - (Kendi) Allah Allah bu vesika Yaşar'ın tezkeresi.. O halde öteki Yaşar kim.. Bu kadar benzeyiş..Bakalım bu çorbanın içinden nasıl çıkacağız..(Tezkereyiiade ederek) Bu sefer artık yanlışlık filan yok. Sen Yaşar'sın değil mi?\\
\textbf{Yaşar} - Evet efendim\\
\textbf{Ka{[}ymakam{]}} - Tekrar et bakalım: Sen kimsin !.\\
\textbf{Yaşar} - Yaşar kaymakam bey. Vallahi, Billahi, iki gözüm önüme aksın\\
\textbf{Ka{[}ymakam{]}} - Peki peki.. Sus (kendi) Ben bu dalaverayı çakarım ya.. (Çıkar)\\

\hypertarget{on-ucuncu-meclis-2}{%
\section{On Üçüncü Meclis}\label{on-ucuncu-meclis-2}}

\begin{verbatim}
 [Seha- Yaşar)
\end{verbatim}

\textbf{Seha} - (Deminki sahneyi bütün kuvvetiyle dinlemiştir.Şiddetle Yaşar'ın yanına giderek kendi) Zannederim ki herşeyi anlamaya başlıyorum.\\
\textbf{Yaşar} - Adam gemisini kurtaran kaptandır.. Öteli Yaşar umurumda mı.. Onlar da kendi başlarının çaresine baksınlar..\\
\textbf{Seha} - Tebrik ederim Yaşar Bey. Pek yere bakan yürek yakanmışsınız..\\
\textbf{Yaşar} - Ben mi? eh sayenizde bir parça öyleyimdir !. Ama belli etmem. Eskiden daha fena idim. Şöyle pek içimin çektiği bir kadın elimden zor kurtulurdu.\\
\textbf{Seha} - hayır onun için demiyorum. Göründüğünüz gibi değil pek kurnazmışsınız demek istiyorum.\\
\textbf{Yaşar} - Ben mi? Ha, bu mu? Size yalnız şunu söyleyim ki gemide üçüncü o {[}düşmüş bir kelime var{]} bizim candan ahbaptır.Bana her zaman: Ulan Yaşar şeytana dümencilik ediyorsun derdi.\\
\textbf{Seha} - (Gülerek) Hah hah hiç şaşmam.. Fakat deminki işte size jarşı pek ziyade nazik davrandığımı inkar etmezsiniz ya! Tezkereyi elimden kaptığınızı kaymakam beye pekala da söyleyebilirdim. Ama.. Yapmadım..\\
\textbf{Yaşar} - Aman yavaştan ha.. Geçtim olsun duyar da..\\
\textbf{Seha} - Merak etmeyin benim ağızım pektir.. Zaten bu tezkereyi size vermemi söyleyen Şemsi Bey'in kendisi idi.\\
\textbf{Yaşar} - Ya! Beni çürük tahtaya bastırıp ağızımdan lakırdı kapacaksınız öyle mi? Bize aytaş'lı derler hanım. Biz ser veririz de sır vermeyiz. . Peki madem ki benim tezkeremi vermenizi size Şemsi Ağabey'im söyledi, öyledir de niçin tezkeremi kaptığım zaman tekrar elimden almak istediniz?\\
\textbf{Seha} - Bu sizin hakikaten Yaşar olup olmadığınızı anlamak içindi..\\
\textbf{Yaşar} - Doğru be.. Ey Vallahi hani siz de şeytana kaptanlık edebilir bir kadınmışsınız.\\
\textbf{Seha} - Ben değil bütün kadınlar isterlerse öyledir (Sandalyeye oturarak) Şemsi Ağabey'iniz çok sevimli bir çocuk. Yalnız bir kusuru var. O da fazla çapkınlığı seviyor.\\
\textbf{Yaşar} - Oo.. Çapkınlığına diyecek yoktur. Köküne kadar çapkın bir erkektir.\\
\textbf{Seha} - iyi ama.. Herhalde birgün karısına yakayı ele vereceği muhakkak..\\
\textbf{Yaşar} - Ben ona zaten söyledim. Şemsi Ağabey dedim, sen de yakayı ele vermezsen ben bu bıyıkları dibinden keserim dedim. Ama o aptal değilse gözünü dört açmalıdır. Zaten bir insan evvelinden her cihetce ayağını tetik almalıdır. Malum ya.. Kadın bir, silah iki, beygir üç.. Bu üç şeye inan olmaz..\\
\textbf{Seha} - (Gülerek) Oo.. Kadınlar için bilmem ama.. Ötekiler için hakkınız var.. Maateessüf de ekseri evvela erkekler de zevcelerine komşu veya ahbaplarının karılarıyla hiyanet ederler.\\
\textbf{Yaşar} - (Gülerek, arkadaş gibi) Öyle öyle. Komşunun tavuğu komşuya kaz.. Karısı kız..\\
\textbf{Seha} - Evet öyle ama.. Şemsi Ağabey'iniz{[}in{]} de Necdet Bey'in karısıyla olan münasebeti kendisine pek pahalı mal olacak galiba..\\
\textbf{Yaşar} - Ama bu öyle birşey ki, çok şaşılacak birşey.. Kendisi Necdet Bey'in karısına uzaktan tutuldu.. Resminden. Resmini gördüğü günden beri onunla oynaş için can atıyor. Geçen hafta ikimizin birden iznimiz çıktı. Bana Yaşar'cığım ikimizin de iznimiz var.. İstanbul'a varınca vesikanı bana verirsin, ben de kendiminkini sana veririm kim kime dum duma üç beş gün süt kardeşinin evinde senin yerine kalırım, dedi. Aman Şemsi Ağabey tanıdığımız kimseleri böyle mandepsiye bastırmak ayıp değil mi?" dedim. "Yaşar'cığım sen beni bu kadar mı seversin? Ben de sana bir daha kardeşim dersem, bana da Fatih'li Şemsi demesinler..\\
\textbf{Seha} - Siz de razı oluverdiniz değil mi?\\
\textbf{Yaşar} - Ne yapayım.. Benim kalbim yumuşaktır. Efendim hiç kimsenin bana darılmasını istemem..\\
\textbf{Seha} - (Kalkarak kendi) Zannederim ki aldığım malumat kafidir.\\
\textbf{Yaşar} - (Kendi) Ah.. Böyle bir kadın insanı baştan çıkarabilir.\\
\textbf{Seha} - Aşcıbaşı bana bu kadar malumatı verdiğinizden dolayı size ne kadar teşekkür etsem azdır.\\
\textbf{Yaşar} - (Endişeli) Malumat mı? Estağfurullah ben bunları size malumat diye söylemedim; hani sözün gelişi..\\
\textbf{Seha} - Ne suretle olursa olsun.. Bu kadına iyilik ettiniz. Verin elinizi sıkayım..\\
\textbf{Yaşar} -(Elini uzatarak) Buyrun..\\
\textbf{Seha} - (Elini burakmadan) HA.. Şimdi size birşey daha soracağım\\
\textbf{Yaşar} - Sorun sorun efendim\\
\textbf{Seha} - Şemsi Bey'in yüzündeki çizgi nedir kuzum..\\
\textbf{Yaşar} - Ne mi.. Ha o mu.. Birgün Şemsi Ağabey'im güvertede ayağı kalkıp {[}kayıp{]} düşmüş yanağını da bir demire çarpmış. Bir de kamarasına gittim ki: Al kan içinde..\\
\textbf{Seha} - Kafi Yaşar.. Siz çok saf kalpli bir bahriyelisiniz. Askerden çıktıktan sonra bize uğrarsanız ben de size iyilik edebilirim. Kocam Şemsi size bizim evi gösterir.\\
\textbf{Yaşar} -(Afallayarak) Kim dediniz? Kim dediniz? Kocanız.. Demek ki siz .. Sizin kocanız.\\
\textbf{Seha} - Evet Şemsi Ağabey'iniz benim kocam..\\
Yaşar- Aman Yarabbi.. Ben ne halt ettim\\
\textbf{Seha} - Teşekkür ederim Yaşar.. Çok teşekkür ederim\\
\textbf{Yaşar} - Tüü.. Yoksa bir çuval inciri berbat mı ettim! (Seha çıkar)\\

\hypertarget{on-dorduncu-meclis-2}{%
\section{On Dördüncü Meclis}\label{on-dorduncu-meclis-2}}

\begin{verbatim}
 [**Yaşar** - sonra Şemsi]
\end{verbatim}

\textbf{Yaşar} - (Bitik bir halde) kadın değil şeytan.. Kurnazlıkta beni fersah fersah geçti..Meğerse beni iskanil ediyormuş ha..\\
\textbf{Şemsi} - (İçeri girerken) Eczacı bizim bugünkü telefon muhaveratımızdan usanmıştır.\\
\textbf{Yaşar} - Eyvahlar olsun Şemsi Ağabey'im de geldi.\\
\textbf{Şemsi} - (Yaşar'ı görerek) Ne.. Sen buradan niye çıktın bakayım..\\
\textbf{Yaşar} - Şemsi Ağabey'ciğim beni dinle (Birden tevakkufla {[}durarak{]}) Aman dur.. Evvela.. Şu vesikayı al.. (Verir)\\
\textbf{Şemsi} - Demek ki seninkini ele geçirdin..\\
\textbf{Yaşar} - (Gözleri parlayarak) Hem öyle kurnazlıkla ki.. (Tezkereyi pantalonunun cebinden çıkarır şeraketinin {[}ortaklık, arkadaşlık(?){]} cebine koyar)\\
\textbf{Şemsi} - Ne gibi ?..\\
\textbf{Yaşar} - Sonra söylerim. Şimdi beni dinle\\
\textbf{Şemsi} - Gördün ya tezkeren kaybolmamış..\\
\textbf{Yaşar} - Canım beni dinle diyorum be\\
\textbf{Şemsi} - Çabuk söyle..\\
\textbf{Yaşar} - Aldırma, metin ol.. Ey gözüm al bakalım planın sunturlusu.. Şemsi Ağabey sana kötü bir haberim var\\
\textbf{Şemsi} - Kötü bir haber mi?\\
\textbf{Yaşar} - Evet Şemsi Ağabey.. karın burada bu dam altında\\
\textbf{Şemsi} - (Gülerek) Amma yaptın ha.. Bu kadar mı kötü haberin\\
\textbf{Yaşar} - Vay.. Sen bunu daha evvel biliyor muydun?\\
\textbf{Şemsi} - Elbette herşeyi biliyorum yavrum !\\
\textbf{Yaşar} - İyi ama öyle ise işin bitik Şemsi Ağabey\\
\textbf{Şemsi} - İşim bitik mi? Aklına şaşayım..\\
\textbf{Yaşar} - İşin bitik diyorum, ben dinle\\
\textbf{Şemsi} - Sus. Kapa şimdi usturanı traş sırası değil. Merak etme be herşeyi halledeceğim..\\
\textbf{Yaşar} - Sahi mi Ağabey..\\
\textbf{Şemsi} - Ben herşeyi halledeceğim dedim ya.. Şimdi sen rahat rahat gidebilirsin\\
\textbf{Yaşar} - Yaşa be Şemsi Ağabey, sen yaman bir tilki imişsin\\
\textbf{Şemsi} - (Zaten meşgul) Şimdi mesele büyük bir cesaret daha göstermekte\\
\textbf{Yaşar} - Öyle ise ben gemiciliğe dönerim babam.. Allahaısmarladık Ağabey. Bak senin hatırın için bizim süt hemşireyi görmeden dönüyorum..\\
\textbf{Şemsi} -(Ona hiç dikkat etmeyerek) Hah.. Bu iyi hatırıma geldi. Bu iyi bir çare..\\
\textbf{Yaşar} - Eyy gel bir öpüşelim de ben gideyim. Hişt Şemsi Ağabey! Tavşan ıykusu mu uyuyorssun\\
\textbf{Şemsi} - Yaşar iki gözüğm dur daha gitmiyorsun. Sana ihtiyacım var..\\
\textbf{Yaşar} -Pış sen.. Öyle yağma yok.. Ben faka bir defa basarım..\\
\textbf{Şemsi} - Yaşar\\
\textbf{Yaşar} - Yaşar Yaşar.. Yok Şemsi Aağabey.. Aman be.. Oyuncak oldum yahu.. Ben bu akşam Selamet'e biniyorum..\\
\textbf{Şemsi} - Yaşar'cığım..\\
\textbf{Yaşar} - Vallahi olmaz..\\
\textbf{Şemsi} - Ya.. Yaşar ben de sana bir daha kardeş dersem bak. Görürsün sen\\
\textbf{Yaşar} - (Gevşeyerek) Etme be Şemsi Ağabey. Darılacak ne var bunda\\
\textbf{Şemsi} - Haydi oradan sen de. Senin gibi kardeş olmaz olsun. İskele babası..\\
\textbf{Yaşar} - Aman Yarabbi.. Azar başladı..\\
\textbf{Şemsi} - (Amirane) Haydi çıkar ceketini çabuk\\
\textbf{Yaşar} - Ceketi mi?\\
\textbf{Şemsi} - Sana çabuk ceketini ver diyorum (Ceketi çıkarır)\\
\textbf{Yaşar} - Peki Ağabey'ciğim al.~Darılma, peki ama ne yapacaksın ?..\\
\textbf{Şemsi} - Ne yapacağımı görürsün. Haydi ver şunu.. (Yaşar verir) Şimdi bana cesaret lazım..\\
\textbf{Yaşar} - Vallahi olsa ondan da verirdim Ağabey'ciğim)\\
\textbf{Şemsi} - Çok söyleme başlığını da ver (Şemsi'nin {[}Yaşar'ın{]} attığı ceketi giyer)\\
\textbf{Yaşar} - Peki ama.. Ne olur, ne yapacaksın söyle de\\
\textbf{Şemsi} - Sonra sonra. Şimdi vaktim yok..\\
\textbf{Yaşar} - Pekala buyurun.. Daha başka.. El torbamı da..\\
\textbf{Şemsi} - Şimdi hiç durmadan haydi deliğine..\\
\textbf{Yaşar} - Öyle ya hadi bakalım deliğe (Geçerken) Gemide ceza yiyen arkadaşınızı {[}arkadaşlarımızı{]} izinle İstanbul'a göndermeli. Bir daha mı (Yakasını ısırır) Allah yazdıysa bozsun..\\
\textbf{Şemsi} - (Arkasından itierek) Hakkın var yavrum..\\

\hypertarget{on-besinci-meclis-2}{%
\section{On Beşinci Meclis}\label{on-besinci-meclis-2}}

\begin{verbatim}
 [**Şemsi** - sonra Kaymakam -sonra Yaşar]
\end{verbatim}

\textbf{Ka{[}ymakam{]}} - (İçerden) Allah Allah bu iş saklambaç oyununa döndü. Bu herif nereye gitmiş..\\
\textbf{Şemsi} - Kaymakam cesaret Şemsi..\\
\textbf{Ka{[}ymakam{]}} - (Girerek) Bu Yaşar olacak sahtekarı ele geçirmek mümkün olmayacak galiba.. (Şemsi'yi görerek) Oo.. Bizim damat..Çulu da değiştirmiş {[}İlerler) Necdet Bey oğlum Yaşar'ı gördünüz mü buralarda\\
\textbf{Şemsi} - (Oralarda değil) Affedersiniz Kaymakam Bey bir yanlışlığım mı var..\\
\textbf{Şemsi} - Evet efendim ben Yaşar değilim !\\
\textbf{Ka{[}ymakam{]}} - (Afallayarak) Ne! Siz benim damadım Necdet Bey değil misiniz?.\\
\textbf{Şemsi} - Aman Kaymakam Bey!\\
\textbf{Ka{[}ymakam{]}} - Hoppala.. Peki o halde kimsin?\\
\textbf{Şemsi} - Ben Şemsi'yim efendim.\\
\textbf{Ka{[}ymakam{]}} - (Sıçrayarak) Ne dediniz Şemsi mi?\\
\textbf{Şemsi} - Şemsi'yim Beyefendi.\\
\textbf{Ka{[}ymakam{]}} - Şemsi.. Siz.. Emin misiniz?\\
\textbf{Şemsi} - Aman Kaymakam Bey kendimi bilmez miyim?\\
\textbf{Ka{[}ymakam{]}} - Buraya ne yapmaya geldiniz..\\
\textbf{Şemsi} - Efendim.. Bizim tanıdık bir aşcı{[}başı{]}sı Yaşar var. bana burada süt kardeşi Pakize Necdet Hanım'ın olduğunu söylemişti. Ben iş için Ada'ya gelmiştim. Uğradım. Biraz evvel de tesadüfen zevcemin burada misafir olduğunu ve benim de fevkalade bir surette Necdet Bey'e benzediğimi haber aldım.\\
\textbf{Ka{[}ymakam{]}} - (Kendisini zorlayarak fakat yine pek hedîd {[}hiddetli{]}) Siz bana bakın efendi.. Galiba topunuz benimle alay etmeye karar vermişsiniz..\\
\textbf{Şemsi} - Rica ederim Kaymakam Bey\\
\textbf{Ka{[}ymakam{]}} - Ayol ben daha biraz evvel Şemsi'yi burada gördüm\\
\textbf{Şemsi} - (Mütehayyır) Şemsi'yi mi gördünüz?\\
\textbf{Ka{[}ymakam{]}} - Evet bu gözlerimle, demin, burada..\\
\textbf{Şemsi} - Kimi? Şemsi'yi mi?\\
\textbf{Ka{[}ymakam{]}} - Evet Şemsi'yi !\\
\textbf{Şemsi} - (Mütehayyir) Şemsi'yi mi görmüş? {[}?{]}\\
\textbf{Ka{[}ymakam{]}} - Siz de mi ikileştiniz? Öyle ise iki Şemsi'den birsi sahtekar. İkinizden biri sahtekarsınız!\\
\textbf{Şemsi} - (Şiddetle) Her halde ben değilim Kaymakam Bey.. Sahtekar olan öteki olsa gerek. Asıl Şemsi benim\\
\textbf{Ka{[}ymakam{]}} - Ya öyle ise göster bakalım tezkereni..\\
\textbf{Şemsi} - (Tezkereyi çıkarıp vererek) Buyurun Beyefendi\\
\textbf{Ka{[}ymakam{]}} - (Vesikaya göz atarak) Bu ne?\\
\textbf{Şemsi} - Görüyorsunuz ya Kaymakam Bey, vesikam..\\
\textbf{Ka{[}ymakam{]}} - Vesikan mı? Bu Yaşar'ın vesikası be..\\
\textbf{Şemsi} - (Kendi) Eyvah Bu sefer sağlam şapa oturdum..\\
\textbf{Ka{[}ymakam{]}} - Bir de Şemsi'yim diyordun ha..\\
\textbf{Şemsi} - Kaymakam Bey\\
\textbf{Ka{[}ymakam{]}} - Kaymakam Bey Kaymakam Bey.. İyi kaç saattir burada benimle oynayan bu çeteyi yakalayacağım.. Çabuk söyleyin: Yaşar'ın tezkeresi sizin elinizde ne geziyor? Burada Yaşar yazıyor. Yaşar. Siz Yaşar mısınız?\\
\textbf{Yaşar} - (Küçük salondan çıkarak) Beni çağırıyorlar galaiba\\
\textbf{Ka{[}ymakam{]}} - (Yaşar'i görerek) Hah..\\
\textbf{Şemsi} - (Kendi ) Buyurun cenaze namazına..\\
\textbf{Ka{[}ymakam{]}} - Sen yine çıktın mı meydana aşcıbaşı..\\
\textbf{Yaşar} - Kaymakam Bey (Şemsi Yaşar'a işaret eder oluyor gibi)\\
\textbf{Ka{[}ymakam{]}} - Sen kimsin bakayım?\\
\textbf{Yaşar} - (Soğukkanlı) Evet Kaymakam Bey daha Yaşar'ım. Yaşar da kalacağım..\\
\textbf{Ka{[}ymakam{]}} - Ya.. Demek emr-i ahire değin Yaşar ha.. Ver bakalım vesikanı çabuk..\\
\textbf{Yaşar} - Buyurun Kaymakam Bey\\
\textbf{Ka{[}ymakam{]}} - (Vesikaya bakarak) Şemsi Nuri..\\
\textbf{Şemsi} - Hah..\\
\textbf{Yaşar} - (Şaşkın kendi) Vay anasını be vesikaları değiştirmeyi unutmuşuz..\\
\textbf{Ka{[}ymakam{]}} - Bu da Şemsi'nin vesikası..\\

\textbf{Yaşar} -\\
\} Kaymakam Bey\\
\textbf{Şemsi} -\\
\textbf{Ka{[}ymakam{]}} - (Masaya bir yumrük vurarak) Hay canına yandığımın işi! Şimdi kaçıracağım.. Söyleyin diyorum size: İkinizden hanginiz Şemsi'siniz\\
\textbf{Şemsi} -\\
\} Benim\\
\textbf{Yaşar} -\\
\textbf{Ka{[}ymakam{]}} - Aman Yarabbi Şimdi ikiniz de mi Şemsi oldunuz\\
\textbf{Şemsi} - Hayır Beyefendi. Hayır efendim. Bakın arz edeyim\\
\textbf{Ka{[}ymakam{]}} - Tiz {[}?{]}.. Her ikiniz de götürüp deliğe tıkacağım anlaşıldı mı?\\

\hypertarget{on-altinci-meclis-1}{%
\section{On Altıncı Meclis}\label{on-altinci-meclis-1}}

\begin{verbatim}
 [Evvelkiler-Seha*]
\end{verbatim}

*Baştan beri Seha olan isim buradan itibaren ``Seza'' olarak dizilmiş\\

\textbf{Seha} - (Birden girerek) Bihude üzülmeyiniz Kaymakam Bey. İşte zevcim..(Şemsi'yi gösterir)\\
\textbf{Şemsi} - (Kendi) Çok şükür..\\
\textbf{Ka{[}ymakam{]}} - Bundan emin misiniz? Bundan aldanmıyorsunuz ya!\\
\textbf{Seha} - Heyhat Kaymakam Bey mateessüf bu sefer eminim..\\
\textbf{Şemsi} - Mateessüf mü? Birçok senelik hasretten sonra beni böyle mi karşılıyorsun Seha\\
\textbf{Seha} - (Kocasına bakıp, birşey söylemeye lüzüm görmeden) Affınızı rica ederim Kaymakam Bey; zevcimle beni bir dakika başbaşa bırakırsanız memnun olurum\\
\textbf{Ka{[}ymakam{]}} - Emredersiniz Seha Hanım (Çıkarken kendi) Hiç şüphe yok kocası.. Öyle ise naıl oluyor da izin tezkeresinde Yaşar yazılı..\\
\textbf{Şemsi} - (Cebri bir şetaretle {[}zorlama bir neşe ile{]}) Ah Seha'cığım.. Seni iyice şaşırttım değil mi? İtiraf et ki sana bu latifeyi yapacağımı hiç hatırına getirmiyordun ya.. Ha anlıyorum.. Bıyıklarımı niçin kestirdiğim sormak istiyorsun, değil mi? Bak anlatayım.. tasavvur et ki sana bu latifeyi hazırladığım zaman birdenbire..\\
\textbf{Seha} - (Gayet sakin) Nafile sıkıntıya girmeyiniz.. Çenenize yazık.. (Yaşar'ı göstererek) Yaşar bana herşeyi anlat{[}tı{]}.\\
\textbf{Şemsi} - Ne !\\
\textbf{Yaşar} - Sana ben işin bitik diye haber vermemiş miydim?\\
\textbf{Şemsi} - Bana hiyanet ettin ha\\
\textbf{Yaşar} - Vallahi.. Benmim kabahatim yok. Sizinki beni faka bastırdı. Ben de alık gibi bir bir anlattım.\\
\textbf{Seha} - Yaşar Efendi azizim siz de biraz Şemsi ile beş dakika yalnız bırakır mısınız?\\
\textbf{Yaşar} - Peki efendim (Kendi) Ah şu kadınların dili; insana her şeyi yaptırır.\\

\hypertarget{on-yedinci-meclis-1}{%
\section{On Yedinci Meclis}\label{on-yedinci-meclis-1}}

\begin{verbatim}
 [Şemsi-Seha]
\end{verbatim}

\textbf{Şemsi} - (Karısının üstüne giderek) Seha'cığım benim..\\
\textbf{Seha} - (Şiddetle) A?. Yoo o kadar acele etmeyin Necdet Beyefendi\\
\textbf{Şemsi} - Necdet rolüne makerahe, mecburiyetle girdiğimi anlaman icap eder\\
\textbf{Seha} - Acaba yaşar rolüne girmeniz de yine bir mecburiyetle mi? Öyle mi? Yarabbi demek ki eğer Pakize'yi ziyarete gelmeseydim, çocukluk arkadaşımın aşkı olacaktınız ha..\\
\textbf{Şemsi} - Ben bunun böyle olduğunu ne bileydim Seha'cığım..\\
\textbf{Seha} - Yarabbi kendini müdafaa için de bulduğu şeye bakın\\
\textbf{Şemsi} - Karıcığım işte itiraf ediyorum: Ben mücrimim. Senin en son mektubundaki hırçın sözlerin tesiri altında ne yapacağımı bilemiyordum. Aklımı kaybetmiştim. Fakat yalnız aklım, kalbimi değil.. O yine eskisi gibi yalnız senin\\
\textbf{Seha} - (Keserek) Beni bütün bu yaldızlı kelimelere kanar mı zannediyorsunuz. Katiyen. Bir buçuk senelik bir ayrılıktan sonra bana bunu mu yapacaktınız? Katiyen, anlıyor musunuz?\\
\textbf{Şemsi} - Affet karıcığım. Bak af diliyorum. Gemide düşüp yanağımı patlattığım zaman hasta olmuştum.. Bu sırada bir hafta mektubunu geciktirdim diye mektubunda bana kocalık haysiyetine dokunacak kelimeler yazıyordun. Beni üzmekten kastın neydi? Ben de seni üzmek için u latifeyi tertip ettim. Ödeştik.. Haydi affettim de de öpüşelim..\\
\textbf{Seha} - Hayır. Siz bu hiyanete latife mi diyorsunuz? Pakize'yi latife olsun diye mi öptünüz? Niçin sükut ediyorsunuz..\\
\textbf{Şemsi} - Şunun için ki.. Seha.. Ne söylesen haklısın. Sana karşı bir hiyanet olduğunu düşünemeyecek kadar kendimi maceraya kaptırmıştım..\\
\textbf{Seha} - Bir de utanmadan buna katır {[}?{]} diyorsun.. İtiraf ediyorsun değil mi?\\
\textbf{Şemsi} - Hayır utanarak itiraf ediyorum Seha..\\
\textbf{Seha} - Bu derece yüzsüz olduğunuzu bilmezdim.. Bir dakika daha fazla yüzünüzü görmeye tahammül edemeyeceğim\\
\textbf{Şemsi} - Nereya gitmek istiyorsun deliliğin lüzumu yok.\\
\textbf{Seha} - Neresi canım isterse oraya gidiyorum ve bu dakikadan itibaren kendimi serbest addediyorum..Anladınız mı? Madem ki siz istediğiniz maceraya kapılmakta hürsünüz ben de aynı derecede hür olmak istiyorum. Süt kardeşiniz size mübarek olsun. Bu evde bir dakika daha durmak bana ağır geliyor. Saat beş vapuruyla gidiyorum.. Haydi bana bir araba çağırın..\\
\textbf{Şemsi} - Seha'cığım beni dinle.. Latifeyi bırak..\\
\textbf{Seha} - Ya gitmiyor musunuz? Zahmet etmeyin ben kendim giderim\\
\textbf{Şemsi} - Gidiyorum (Kendi) Zaten benim de istediğim o. Bir araba çağırır iskeleye beraberce gider oradan da haydi Bayezit'e..\\
\textbf{Seha} - Ne duruyorsunuz?\\
\textbf{Şemsi} - Görüyorum {[}Gidiyorum{]} karıcığım (Kendi) Ben sana kendimi affetiririm ya\\
(Çıkar)

\hypertarget{on-sekizinci-meclis}{%
\section{On Sekizinci Meclis}\label{on-sekizinci-meclis}}

\begin{verbatim}
 [**Seha** - Kaymakam - Necdet- Pakize- Yaşar- Şemsi]
\end{verbatim}

\textbf{Seha} - Aptal bana kendisini affettirecek kelimeyi bulamadı.\\
Kaymakam - (Girerek) Ey Seha Hanım kocanıza hiyanetini itiraf ettirebildiniz mi?\\
\textbf{Seha} - Evet.. hepsini itiraf etti. Kaymakam Bey eğer beni elan seviyorsanız\\
\textbf{Ka{[}ymakam{]}} - (Keserek) Ah hanımefendi seviyor muyum? Yanıyorum, tutuşuyorum, tam istim bekliyorum..\\
\textbf{Seha} - (Bir tereddütle) Öyle ise Kaymakam Bey ben sizinim..\\
\textbf{Ka{[}ymakam{]}} - (Kollarını açar) Ah Seha, Seha, Seha'cığım\\
\textbf{Seha} - (Şiddetle silkinerek) Hayır hayır bırakınız (Kendini bir iskemleye atar ve ağlar) Aman Yarabbi ne felaket, ne bedbahtım..\\
\textbf{Ka{[}ymakam{]}} - (Kendi) Hoppala.. Hala kocasını seviyor.\\
\textbf{Seha} - (Mahcup) Kaymakam Bey affediniz. Ne yapacağımı bilemiyorum.\\
\textbf{Ka{[}ymakam{]}} - İtizara {[}özür dilemeye{]} lüzum yok.. Seha Hanım, ben anladım. Şimdi anladım.\\
\textbf{Pakize} - (Necdet'le girerek) A.. Seha'cığım ağlıyor musun?\\
\textbf{Seha} - Hayır Pakize'ciğim.. Demin bir baş dönmesi geçirdim de.. Ah kardeşim biz kadınlar fenalıkta erkeklerle boy ölçüşemeyiz.. çapkın kocalarımızdan intikam almaya yemin ederiz fakat son dakikada iffetimiz isyan eder, kalbimiz kirlenmeye tahammül edemez. Eminim ki sen de böylesin. Sen de affedeceksin..\\
\textbf{Pakize} - Evet ben de affettim Seha'cığım.\\
\textbf{Seha} - Nasıl..\\
\textbf{Pakize} - Beş dakikadan beri.\\
\textbf{Ka{[}ymakam{]}} - Ne? Senin kocan da mı aldatmıştı?\\
\textbf{Necdet} - Topu, topu bir defacık Kaymakam Bey. ama size yemin ederim ki bu sonu olacak..\\
\textbf{Ka{[}ymakam{]}} - (Şaşkın) Nasıl sen Yaşar?.\\
\textbf{Necdet} - Hayır Kaymakam Bey Yaşar değil, Necdet..\\
\textbf{Ka{[}ymakam{]}} - (Daha ziyade şaşkın) Bu nasıl şey\\
\textbf{Yaşar} - Şimdi Necdet oldu.\\
\textbf{Pakize} , \textbf{Necdet} - Evet! Ya..\\
\textbf{Ka{[}ymakam{]}} - Ey öyle ise aşcı Yaşar nerede, aşcı Yaşar!\\
\textbf{Yaşar} - (Temenna ederek) Mevcut efendim, Yaşar mevcut.. Şeytanın art ayağı, dümencisi..\\
\textbf{Şemsi} - (Girerek) Arabayı çağırdım\\
\textbf{Necdet} - (Şemsi'ye) Azizim kardeşim Necdet'ciğim.. Hayır Yaşar'cığım..Aman Şemsi'ciğim, karının ayaklarına kapan. seni affediyor\\
\textbf{Şemsi} - (Seha'ya koşarak) Ah.. Karıcığım.. Biliyordum zaten.. Affettin ha !.\\
\textbf{Ka{[}ymakam{]}} - Durun be.. Şaşkına döndüm. Mesele kördüğüm oldu..İçinden çıkamıyorum.
\textbf{Şemsi} - kaymakam Bey size kördüğümü bu akşam yemeğinde açarız. Ve eminim ki siz de affedeceksiniz..\\
\textbf{Ka{[}ymakam{]}} - Sizi af mı etmek.. Beni oyuncağa çevirdiğiniz halde mi? Allah göstermesin\\
\textbf{Necdet}, \textbf{Pakize} - Ah benim güzel dayıcığım affet..\\
\textbf{Şemsi}, \textbf{Seha} - Bizm hatırımız için Kaymakam Bey.. (Etrafını alırlar)\\
\textbf{Ka{[}ymakam{]}} - Peki, peki, madem ki hepiniz yalvarıyorsunuz öyle olsun!. Affediyorum..\\
\textbf{Necdet}, \textbf{Pakize} - (Müteşekkir) Ah dayıcığım..\\
\textbf{Şemsi}, \textbf{Seha} - Oh Kaymakam Bey ..\\
\textbf{Ka{[}ymakam{]}} - Ama bir şartla, bir tek şartla: Bana dokuz aya kadar iki küçük Türk bahriyelisi yetiştireceksiniz..\\
\textbf{Yaşar} - (kuvvetle) Üç olsun Kaymakam Bey.. Bir de benden..\\
\textbf{Herkes} - Nasıl bir de senden?\\
\textbf{Yaşar} - (Sırıtarak) Evet.. Acizane..\\

PERDE İNER

\bibliography{book.bib,packages.bib}


\end{document}
